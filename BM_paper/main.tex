\documentclass[lean]{Draft}

\usepackage{newunicodechar}
\usepackage{xspace}
\usepackage{colonequals}
\usepackage[colorlinks=true]{hyperref}

\def\lstlanguagefiles{lstlean.sty}
\lstset{language=lean}

%%%Mathematical abbreviations
\newcommand*{\N}{\mathbb{N}}
\newcommand*{\Z}{\mathbb{Z}}
\newcommand*{\Q}{\mathbb{Q}}
\newcommand*{\R}{\mathbb{R}}
\newcommand*{\C}{\mathbb{C}}
\newcommand*{\mathlib}{\textsc{Mathlib}\xspace}
\newcommand*{\Lean}{\textsc{Lean}\xspace}
\newcommand*{\Leant}{\textsc{Lean3}\xspace}
\newcommand*{\Leanf}{\textsc{Lean4}\xspace}


\addbibresource{biblio.bib}
\usepackage{tikz}
\usetikzlibrary{arrows, arrows.meta}
\definecolor{ghpurple}{rgb}{0.4, 0.22, 0.73}

% \hypersetup{
%   colorlinks = true,
%   citecolor = teal,
%   linkcolor = teal,
%   urlcolor = ghpurple
% }

\newcommand{\changeurlcolor}[1]{\hypersetup{urlcolor=#1}}
\newcommand{\lsthref}[2]{\changeurlcolor{ghpurple}{\ttfamily\href{#1}{#2}}}
\newcommand{\leanok}{\qed}
\newcommand{\uses}[1]{Uses #1}
\newcommand{\mathlibok}{}
\renewcommand{\lean}[1]{\lstinline| #1|}

\makeatletter
\newcommand\leanlink{\begingroup\catcode`\#=12\relax\@leanlink}
\newcommand\@leanlink[2]{\endgroup
\href{#1}
{\texttt{\detokenize{#2}}}}

\newcommand{\docs}[1]{%
\leanlink{https://leanprover-community.github.io/mathlib4_docs/find/?pattern=#1\#doc}
{#1}}

\lstset{escapeinside={(*}{*)}, language=lean}

\title[Formalization of Brownian motion in Lean]{Formalization of Brownian motion in Lean}
\author[R. Degenne, M. Himmel, D. Ledvinka, E. Marion, P. Pfaffelhuber]{
  Rémy Degenne, Markus Himmel, David Ledvinka, Etienne Marion, Peter Pfaffelhuber}


\authorinfo[R. Degenne]{Univ. Lille, Inria, CNRS, Centrale Lille, UMR 9189-CRIStAL, F-59000 Lille, France}{remy.degenne@inria.fr}
\authorinfo[M. Himmel]{University of XYZ, Country}{markus.himmel@example.com}
\authorinfo[D. Ledvinka]{University of XYZ, Country}{david.ledvinka@example.com}
\authorinfo[E. Marion]{University of XYZ, Country}{etienne.marion@example.com}
\authorinfo[P. Pfaffelhuber]{University of Freiburg, Germany}{p.p@stochastik.uni-freiburg.de}

% please fill in
\msc{TODO}
\keywords{Formalization, Mathlib, Lean, Probability, Brownian motion}

%update
\VOLUME{TODO}
\YEAR{TODO}
\NUMBER{TODO}
\firstpage{1}
\DOI{TODO}
\receiveddate{TODO}
\finaldate{TODO}
\accepteddate{TODO}


\begin{abstract}
Brownian motion is a building block in modern probability theory. In this paper, we describe a formalization of Brownian motion using the Lean theorem prover. We build on the existing measure-theoretic foundations in Lean's mathematical library, Mathlib, and we develop several key components needed for the construction of Brownian motion, including the Carathéodory and Kolmogorov extension theorems, Gaussian measures in Banach spaces, and the Kolmogorov-Chentsov theorem for path continuity.
\end{abstract}

\begin{document}

\href{https://github.com/RemyDegenne/brownian-motion}{github}

% Some useful commands:
% \mathlib{}
% \leanlink{url}{this is a link to Lean code}

% Inline code: \lstinline|sorry|

% Multi-line code:
% \begin{lstlisting}
% /-- A process `X : T → Ω → E` has independent increments if for any `n ≥ 2` and `t₁ ≤ ... ≤ tₙ`,
% the random variables `X t₂ - X t₁, X t₃ - X t₂, ...` are independent. -/
% def HasIndepIncrements {Ω T E : Type*} {mΩ : MeasurableSpace Ω} [Sub E]
%     [Preorder T] [MeasurableSpace E] (X : T → Ω → E) (P : Measure Ω) : Prop :=
%   ∀ n, ∀ t : Fin (n + 2) → T, Monotone t →
%     iIndepFun (fun i : Fin (n + 1) ↦ X (t i.succ) - X (t i.castSucc)) P
% \end{lstlisting}

\section{Introduction}

\subsection{Mathematical background}
\sloppy Brownian motion is arguably one of the most important stochastic processes (e.g.\ \cite{karatzas1991brownian, morters2010brownian}), and is used as a modeling tool across all sciences (physics: e.g.\ \cite{einstein1906theorie, bian2016111}; biology: e.g.\ \cite{erban2014molecular}; finance: e.g.\ \cite{davis2006louis}). Mathematically, Brownian Motion led to Wiener measure \cite{wiener1923differential}, which is the first instance of a probability measure on a function space, but also to developments such as Stochastic (Partial) Differential equations (see e.g.\ \cite{hairer2009introduction}).

Let us briefly describe Brownian motion $X = (X_t)_{t\geq 0}$. There are several interesting features (considering the one-dimensional version and if $X_0=0$):
\begin{itemize}
\item It has a normal distirbution at all times; more precisely, for any $t_1,...,t_n > 0$, we have that $(X_{t_1},...,X_{t_n})$ has a multivariate normal distribution, with $\mathbf E[X_{t_i}] = 0$ and $\mathbf{COV}(X_{t_i}, X_{t_j}) = t_i \wedge t_j$ (in other words, $X$ is a Gaussian process);
\item Its states after time $t$ are independent of states before $t$ given $X_t$, or $\mathbb P(X_{t+s} \in . | (X_r)_{r\leq t}) = \mathbb P(X_{t+s} \in . | X_t)$ for $s,t \ge 0$, i.e.\ Brownian Motion is a Markov process;
\item It has independent increments, which only depend on the evolved time, i.e.\ $X_s$ and $X_t - X_s$ are independent, and $X_t - X_s$ is a function of $t-s$, i.e.\ it is a Lévy process;
\item It describes a fair game (usually denoted a martingale), i.e.\ $\mathbb E[X_{t+s} | (X_r)_{r\leq t}] = X_t$ for $s,t \geq 0$; interestingly, it is the only stochastic process with continuous paths such that $(X_t)_{t\geq 0}$ and $(X_t^2 - t)_{t\geq 0}$ are martingales;
\item It has continuous paths (almost surely); more precisely, it has Hölder-continuous paths for any coefficient $\beta < \tfrac 12$; as a consequence, its paths are nowhere differentiable, almost surely;
\end{itemize}
The goal of the present paper is to describe a formalization project of Brownian Motion using Lean \cite{lean}, building on its mathematical library. Since Mathlib \cite{mathlib} has a decent amount of measure theory already included, but is still lacking some fundamentals, we describe our way to a full formalization using the following steps:
\begin{itemize}
\item {\em The Carathéodory and Kolmogorov Extension Theorems:} Constructing a probability measure on an uncountable infinite product (i.e.\ the distribution of a stochastic process) is not straight-forward, since the corresponding product of $\sigma$-algebras is too large. Rather, one uses the $\sigma$-algebra defined by all finite (or countably infinite) projections, and constructs a probability measure by extension from all finite dimensional possibilities, which is the content of the Kolmogorov Extension Theorem. This result is in fact an application of a more general result, the Carathèodory Extension Theorem. In this step, we need that the family of finite dimensional distirbutions has a consistency property (i.e.\ it forms a projective family).
\item {\em Gaussian Measures and characteristic functions:} Defining a one-dimensional normal distribution is straight-forward using its density with respect to Lebesgue measure, and concrete calulations are often possible using the density. In the multi-dimensional case, which we have to work with extensively, it is often much more convenient to use characteristic functions, which are known to characterize probability measures uniquely. In particular, they can be used to show that linear maps (e.g.\ projections) of Gaussian measures are Gaussian. This approach is fundamental to show that finite dimensional distirbutions of Brownian Motion indeed form a projective family.
\item {\em The Kolmogorov-Chentsov Theorem:} We are aiming for a stochastic process with continuous paths. The classical Kolmogorov Chentsov theorem gives a criterion in terms of some moment bounds if the set of times is a subset of $\mathbb R^d$; see e.g.\ \cite{kallenberg2021}. Here, we use a modern version for existence of a modification of a stochastic process with continuous paths, based on a more general set of times \cite{kratschmer2023kolmogorov}.
\item {\em Construction of Brownian motion and Wiener measure on $\R_+$:} Putting all of the above together gives a continuous Gaussian stochastic process with the correct distribution, i.e.\ Brownian Motion. From this, we can also define a probability measure on (the Polish space) of continuous functions $\mathbb R_+ \to \mathbb R$. This is usually called the Wiener measure.
\end{itemize}
It is important to realize that probability theory comes with a duality between random variables (or stochastic processes) and their distributions (or laws), which are probability measures. In the case of stochastic processes, the latter are defined on the (often uncountable infinite) product on the state space. While some properties of a stochastic process are only dependent on their distribution (e.g.\ being a Gaussian process), other properties depend on more refined properties (e.g.\ having continuous paths). We account for this duality in our code by introducing in the assumptions that a certain law is required.

\subsection{Related work}
In the literature, there are some formalizations of stochastic processes. Within Isabelle/HOL, the Kolmogorov extension theorem has been previously formalized \cite{Immler2012}. This formalization only works on Polish spaces (rather than on spaces where every finite measure is inner regular with respect to compact sets, see below), and only in the case where the state spaces for all times are identical. As for the Kolmogorov-Chentson-Theorem (showing continuity of paths), there is an Isabelle/HOL formalization as well \cite{Kolmogorov_Chentsov-AFP}. It uses the index set  $\R_+$ (rather than a more general type) with the usual dyadics proof, as e.g. outlined in \cite{kallenberg2021}. Last,
Brownian motion in implemented in Isabelle/HOL at \cite{laursen2024brownian} (the code is on \href{https://github.com/cplaursen/Brownian_Motion}{github})\todorm{the formalization does not look complete}. As the authors say, dependent type theory would have facilitated their work in places.

For the Lean theorem prover, \cite{ying2023formalization} formalized martingales (see e.g.\ \cite{kallenberg2021}) and was the first implementation of stochastic processes. In our work, however, we do not touch martingales, and rather build on implementations of measure theory, as implemented in \cite{mathlib}.


\subsection{Project organization}

This project started as a traditional collaboration between researchers, in which two of the authors (R.D. and P.P.) formalized the Kolmogorov extension theorem.
In a second phase however, we decided to transform the project into a collaborative effort, open to anyone interested in contributing.
We wrote first a succinct description of the proof we intended to follow for the Brownian motion definition, before expanding it into a detailed \emph{blueprint} (TODO cite leanblueprint, using the github template \cite{Monticone_LeanProject_2025}).
This is a website that accompanies the github repository on which the code is hosted, and that contains pages with detailed descriptions of the sequence of lemmas and definitions needed to prove the main theorems.
The dependency of each lemma on the previous ones is made explicit, and the \texttt{leanblueprint} tool automatically generates a color-coded graph of dependencies.
The graph is useful to visualize the structure of the proof, and to identify which lemmas are proved, which are ready to be proved, and which are still missing prerequisites.
Before making the project public, we additionally implemented in \Lean several key definitions and statements (without proofs).
The project was announced on the Lean Zulip website, and anyone interested was invited to contribute.
M.H, D.L., and E.M. joined the project at that time and then contributed significantly to the formalization.
The coordination of the project was done through discussions on Zulip: lists of tasks and progress updates were regularly posted, and volunteers could claim the tasks and report on their progress.
The correctness of the code was ensured by continuous integration scripts on github, which automatically check that the code compiles without errors, and by manual review of the proposed additions.

\todor{remark on the blueprint bottleneck?}


\section{The formalization}
Let us start with our notation. Any stochastic process has an index set $\iota$ (usually denoted {\em time}, usually uncountably infinite), and a state space. Since we use dependent type theory, the state space may depend on the time, i.e.\ the state space at time $t$ is $\alpha t$. Probability measures are denoted $P_.$.

All references to results in \mathlib are accurate for commit {\tt xxx}, August 22, 2025.

\subsection{The Carathéodory and Kolmogorov Extension Theorems}
The usual approach to construct (the distribution of) a stochastic process works as follows: describe properties of the distribution of the stochastic process $P_J$ at some arbitrary but finite number of times $J = \{t_1,...,t_n\} \subseteq \iota$.
The resulting family of probability measures $(P_J)_{J \subseteq \iota \text{ finite}}$ has to be {\em projective} in the sense that the projection of $P_J$ to $H\subseteq J$ has to be equal to $P_H$. In other words, when describing the distribution of the stochastic process at all times in $J$, and then forgetting all properties for times in $J\setminus H$, results in the description of properties at times in $H$. One may then ask if this already gives a complete description of the process for all times.

It is the achievement of Kolmogorov that these finite-dimensional distributions in fact provide a unique description of the distribution of a stochastic process, even if $\iota$ is uncountable, as long as the underlying family of state spaces $(\alpha_t)_{t\in\iota}$ is nice enough (Polish, i.e.\ a separable topological space which can be metrized by a complete metric, for example) \cite{kolmogoroff1933grundbegriffe}.
The resulting measure is defined on the product-$\sigma$-field $\mathcal F :=\bigotimes_{t\in\iota} \mathcal B(\alpha_t)$ (where $\mathcal B(\alpha_t)$ is the Borel $\sigma$-algebra on $\alpha_t$). Here, $\mathcal F$ is generated by finite projections and hence any element of $\mathcal F$ may only depend on at most countably many $t\in \iota$, making this a rather coarse $\sigma$-algebra. (In particular, note that this is not the Borel $\sigma$-algebra of the product topology for infinite $\iota$.)

The Kolmogorov extension theorem is on the interface between measure theory and probability theory. Here, we rely on a decent amount of formalized mathematics in the measure-theory part of {\tt mathlib} (outer measures, above all), while not requiring any specific previous formalization of probability theory. (In fact, most of our results are formulated in terms of finite rather than probability measures.)

We are going to formulate the main result in a modern fashion, as e.g.\ found in Theorem 2.2 of \cite{rao1971projective}, Theorem 7.7.1 of Volume~2 of \cite{bogachev2007measure}, Theorem 15.26 of \cite{guide2006infinite}, or \cite{border1998expository}. Note that these formulations split general assumptions on the underlying space(s) (e.g.\ a metric property) from the property which is needed in the proof (inner regularity with respect to compact sets). Other -- highly readable -- references such as \cite{Billingsley1995} state the extension theorem only in special cases such as $\alpha_t = \mathbb R$ for all $t$.


\subsection{Gaussian Measures and characteristic functions}
\label{ss:char}
Our goal in this subsection is to define the finite-dimensional distributions of Brownian Motion. For times $t := (t_1, ..., t_n)$, this is given by the multi-dimensional Gaussian distribution $N(0, C_t)$, where $0$ is the vector of expectations, and $C_t$ is the covariance matrix, given by $C_{ij} = t_i \wedge t_j$. In order to do so, we rely on (i) the implementation of the one-dimensional normal distribution (xxx ref to mathlib) and characteristic functions of probability (or finite) measures.

For any probability measure $P$ on $\mathbb R^n$, its characteristic function is given by $\psi: t \mapsto \int e^{it^\top x} P(dx)$, where the integral takes values in $\mathbb C$. We use the fact that $\psi$ characterizes $P$ uniquely (xxx ref to mathlib). For the standard normal distribution $N(0,1)$ this is $\psi_{N(0,1)}(t) = \exp(-t^2/2)$. Moreover, by independence, we can as well define the $n$-fold product measure $N(0, I_n)$, where $I_n$ is the unit matrix with characteristic function $\psi_{N(0,I_n)}(t) = \exp\big(-\tfrac 12 t^\top I_n t\big)$. In addition, we call a probability measure $P$ on $\R^n$ Gaussian if there is some non-negative definite matrix $C$ with $\psi_P(t) = \exp\big( - \tfrac 12 t^\top C t\big)$. For any such $C$, such a measure exists, since there is $A$ with $C = A^\top A$ (xxx ref to mathlib) and there is the image of $N(0,I_n)$ under the map $f : x\mapsto Ax$. It has the caracteristic function
\begin{align} \label{eq:gausslin}
\psi_{f_\ast N(0,I_n)}(t) = \int e^{it^\top x} f_\ast N(0,I_n) dx = \int e^{it^\top A x} N(0,I_n) dx = \exp\big( - \tfrac 12 t^\top C t\big).
\end{align}
Using the same transformation, we can show that Gaussian measures are closed under linear maps.

So, we can define the finite dimensional distributions of Brownian motion given that we show that $C \in \R^n$ with entries $C_{ij} = t_i \wedge t_j$ is non-negative definite. For this, we rely on Gram matrices, which are based on inner product spaces. In such a space $E$ (with scalar product $\langle .,. \rangle$), and $v_i,...,v_n \in E$, define $C_{ij} := \langle v_i, v_j\rangle$. Then, for $t :=(t_1,...,t_n) \in \R^n$,
$$ t^\top C t = \sum_{i,j} t_i \langle v_i, v_j\rangle t_j = \Big\langle \sum_i t_i v_i, \sum_j t_j v_j\Big\rangle \geq 0.$$
Let $v_i = 1_{[0,t_i]}$ in the space of $L^2$ integrable functions with respect to Lebesgue integral. Then,
$$ \langle v_i, v_j \rangle = \int 1_{[0,t_i]}(x) 1_{[0,t_j]}(x) dx = \int 1_{[0,t_i \wedge t_j]}(x) dx = t_i \wedge t_j.$$
So, $C$ from above is a Gram matrix, which is non-negative definite by general theory, and we have constructed the finite dimensional distributions for Brownian motion.

\cite{hairer2009introduction}

\subsection{The Kolmogorov-Chentsov Theorem}
Let us describe briefly the Kolmogorov-Chentsov Theorem in a simple case here, before we dive deeper in the next sections. The result states that for a stochastic process $X$ (with $\iota = [0,1])$, assume we find $\alpha, \beta, C > 0$ satisfying
\begin{align}
\label{eq:cs}
  \mathbf E[r(X_s, X_t)^\alpha] \leq C|t-s|^{\beta + 1}, \qquad 0\leq s,t\leq 1.
\end{align}
Then there exists $Y = (Y_t)_{t\in [0,1]}$ with $\mathbf P(X_t = Y_t) = 1$ for all $t\in [0,1]$ and $Y$ has almost surely Hölder continuous paths with coefficient $\gamma$ for any $\gamma < \tfrac \beta \alpha$.

In order to see this, set $D_n := \{k/2^n: k=0,...,2^n\}$ and $D := \bigcup_{n\in\mathbb N} D_n$. Start by showing summability of $\mathbf P\Big( \sup_{s,t\in D_n, |t-s| = 2^{-n}} r(X_s, X_t) \geq 2^{-\gamma n} \Big)$ using \eqref{eq:cs}. By the Borel-Cantelli Lemma, this shows that $\sup_{s,t\in D_n, |t-s| = 2^{-n}} r(X_s, X_t) \leq 2^{-\gamma n}$ holds for $n$ large enough. From this, we see that $X$ is locally Hölder-$\gamma$ continuous on $D$. From here, define some Hölder continuous $Y$ coinciding with $X$ on $D$. Finally, fix $t \in [0,1]$ and $t_1, t_2,...\in D$ with $t_n \xrightarrow{n\to\infty} t$. Using \eqref{eq:cs}, we see that $X_{t_n} \xrightarrow{n\to\infty} X_t$ in probability, as well as $Y_{t_n} \xrightarrow{n\to\infty} Y_t$ almost surely due to continuity of $Y$. Therefore, $X_t = Y_t$ almost surely.

We will use a more general version of this statement, replacing $\iota = [0,1]$ by a metric space with a property restricting the number of balls needed to cover $\iota$. This is based on recent work of \cite{talagrand2022upper} and \cite{kratschmer2023kolmogorov}.

\subsection{Construction of Brownian motion and Wiener measure on $\R_+$}
In order to finally construct Brownian Motion, we need to put everything together. First, the projectivity property of the finite-dimensional distributions as defined in Section~\ref{ss:char} follows from the fact that Gaussian measures are closed under linear maps, and -- as in \eqref{eq:gausslin} -- the characteristic function of the image of $N(0,C)$ under the linear map $\Pi : x \mapsto \pi x$ is $\Psi_{\Pi_\ast N(0,C)}(t) = \exp\Big( -\tfrac 12 t^\top \pi^\top C \pi t \Big)$. So, for $t = (t_1,...,t_n)$ and $C(t) \in \mathbb R^{n\times n}$ with $C(t)_{ij} = t_i \wedge t_j$ and the projection $\Pi: \mathbb R^n \to \mathbb R^m$ with $m < n$ and $\Pi(t) = (s_1,...,s_m)$, this shows that $\Pi_\ast N(0,C(t)) = N(0,C(s))$, which is the desired projectivity. Therefore, we have constructed a probability measure on $\mathbb R_+^{\mathbb R}$, the law of Brownian motion.\\
Finally, the assumption in the Kolmogorov-Chentsov Theorem can e.g.\ be verified by using that
\[ \mathbf E[|X_t - X_s|^4] = \mathbf E[X_{t-s}^4] = (t-s)^2 \mathbf E[X_{1}^4] < \infty.\]
So, we have constructed a process with continuous paths and the correct finite dimensional distributions, which we call Brownian Motion. Since this also gives a distribution on $\mathcal C(\mathbb R_+, \mathbb R)$, we call this distribution Wiener measure.



\section{The Carathéodory and Kolmogorov Extension Theorems}
\label{S:extension}
We describe our implementation of the Kolmogorov Extension Theorem, and rely on basic notions from topology, such as a metric space, and the Borel $\sigma$-algebra. Recall, that a pseudo-metric $r(.,.)$ can have $r(x,y) = 0$ for $x\neq y$.

We begin by defining set systems we will need; see also \docs{MeasureTheory.IsSetSemiring}  and \docs{MeasureTheory.IsSetRing}.

\begin{definition}[Semi-ring, ring]\label{def:semi}
  Let $\alpha$ be some set. We call $\mathcal H \subseteq 2^\alpha$ a
  \emph{semi-ring}, if it is (i) a $\pi$-system (i.e.\ closed under
  $\cap$) and (ii) for all $A, B \in\mathcal H$ there is\footnote{We write $A \subseteq_f B$ iff $A$ is a finite subset of $B$.} $\mathcal K
  \subseteq_f \mathcal H$ with\footnote{We write $A\uplus B$ for
    $A\cup B$ if $A\cap B=\emptyset$.}  $B\setminus A = \biguplus_{K
    \in \mathcal K} K$.  \\ We call $\mathcal H \subseteq 2^\alpha$ a
  \emph{ring}, if it is closed under $\cup$ and under set-differences.
\end{definition}

\noindent
Any ring is a semi-ring since $A\cap B = A \setminus (A \setminus B)$, i.e.\ every ring is a $\pi$-system. Let us state two important lemmas on semi-rings; see and
\docs{MeasureTheory.IsSetSemiring.disjointOfDiffUnion} and \docs{MeasureTheory.IsSetSemiring.disjointOfUnion}.

\begin{lemma}\label{l1}
  Let $\mathcal H$ be a semi-ring, $\mathcal I \subseteq_f \mathcal H$,
  $A \in \mathcal H$. Then, there is $\mathcal K \subseteq_f \mathcal
  H$ such that $\mathcal K$ contains pairwise disjoint sets and $A
  \setminus \bigcup_{I \in \mathcal I} I = \biguplus_{K\in \mathcal K}
  K$.
\end{lemma}

\begin{lemma}\label{l2}
  Let $\mathcal H$ be a semi-ring and $A_1,...,A_m \in \mathcal
  H$. Then, there are $\mathcal K_1,...,\mathcal K_m \subseteq_f
  \mathcal H$ disjoint such that $\bigcup_{n=1}^m \mathcal K_n$
  contains disjoint sets and $\bigcup_{m=1}^n A_m = \biguplus_{m=1}^n
  \biguplus_{K \in \mathcal K_n} K$.
\end{lemma}

Given an additive content $m : \mathcal H \to [0,\infty]$ for some semi-ring $\mathcal H$ (see Definition~\ref{def:content} as well as \docs{MeasureTheory.AddContent}), the goal of Carathéodory's extension theorem is to define a measure $\mu : \sigma(\mathcal H) \to [0,\infty]$ extending $m$ to the $\sigma$-algebra $\sigma(\mathcal H)$ generated by $\mathcal H$. More precisely, the Carathéodory extension gives a measure on an even larger $\sigma$-algebra; see Theorem~\ref{T:cara} in conjunction with Theorem~\ref{T:masseind} below. We will follow this abstract construction, and start by stating some basic concepts; see \docs{MeasureTheory.Measure}, \docs{MeasureTheory.OuterMeasure}.

\begin{definition}\label{def:content}
  For some set $\alpha$, let $\mathcal H \subseteq 2^\alpha$
  and call any $m : \mathcal H \to [0,\infty]$ a content.
  \begin{enumerate}
  \item $m$ is called additive if for $\mathcal K \subseteq_f \mathcal H$ pairwise disjoint and $\bigcup_{K \in \mathcal K} K \in \mathcal H$, we have $m \Big(\bigcup_{K \in \mathcal K} K \Big) = \sum_{K \in \mathcal K} m(K)$. If the same holds for\footnote {We write $A \subseteq_c B$ if $A$ is a countable subset of $B$.}$\mathcal K \subseteq_c \mathcal H$ pairwise disjoint, we say that $m$ is $\sigma$-additive.
  \item The set-function $m$ is called sub-additive if for $\mathcal K \subseteq_f \mathcal H$ and $\bigcup_{K \in \mathcal K} K \in \mathcal H$, we have $m \Big(\bigcup_{K \in \mathcal K} K \Big) \leq \sum_{K \in \mathcal K} m(K)$. (Note that elements of $\mathcal K$ need not be disjoint.) Here, $\sigma$-sub-additivity is defined in the obvious way using $\mathcal K\subseteq_c \mathcal H$.
  \item If $m(A) \leq m(B)$ for $A\subseteq B$ and $A,B\in\mathcal H$, we say that $m$ is monotone.
  \item If $\mathcal H$ is a $\sigma$-algebra and $m$ is $\sigma$-additive with $m(\emptyset) = 0$, we call
    $m$ a measure.
  \item If $\mathcal H = 2^\alpha$, $m$ is monotone and $\sigma$-sub-additive with $m(\emptyset)=0$, we call $m$ an outer measure.
  \end{enumerate}
\end{definition}

\subsection{Carathéodory's Extension Theorem}
An additive, $\sigma$-sub-additive content on a semi-ring induces an outer measure by approximating sets from above (see \docs{MeasureTheory.inducedOuterMeasure_eq}):

\begin{proposition}[Outer measure induced by a set function on a semi-ring] %\mbox{}
  Let \label{P:auss} $\mathcal H$ be a semi-ring and $m: \mathcal H\to\mathbb R_+$ additive and $\sigma$-sub-additive. For $A\subseteq E$ let
  $$ \mu(A) := \inf_{\mathcal G \in \mathcal U(A)} \sum_{G\in\mathcal G} m(G)$$ where
  $$ \mathcal U(A) := \big\{\mathcal G \subseteq_c \mathcal H, A\subseteq \bigcup_{G\in\mathcal G} G\big\}$$
  is the set of countable coverings of $A$. Then, $\mu$ is an outer measure.
\end{proposition}

Moreover, we call a set Carathéodory wrt some outer measure, if it is measurable in the following sense (see \docs{MeasureTheory.OuterMeasure.IsCaratheodory}):

\begin{theorem}[\boldmath $\mu$-measurable sets are a
    $\sigma$-algebra]\label{T:cara} Let $\mu$ be an outer measure on
  $E$ and $\mathcal F$ the set of $\mu$-measurable sets,
  i.e.\ the set of sets $A$ satisfying
  \begin{align*}
    \mu(B) = \mu(B\cap A) + \mu(B\cap A^c), \qquad B \subseteq E.
  \end{align*}
  Then, $\mathcal F$ is a $\sigma$-Algebra and $\mu|_{\mathcal F}$ is
  a measure.
\end{theorem}

In the construction above, since all sets in the semi-ring $\mathcal H$ are Carathéodory (see \docs{MeasureTheory.AddContent.isCaratheodory_ofFunction_of_mem}), the $\sigma$-algebra from Theorem~\ref{T:cara} is at least as large as the $\sigma$-algebra generated by $\mathcal H$. The resulting measure is \docs{MeasureTheory.AddContent.measureCaratheodory} and agrees with the additive content on $\mathcal H$ (see \docs{MeasureTheory.AddContent.measure_eq}). Let us summarize this:

\begin{theorem}[Carathéodory extension]\label{T:masseind}
  Let $\mathcal H$ be a semi-ring and $m: \mathcal H\to\mathbb R_+$
  $\sigma$-finite and $\sigma$-additive. Furthermore, let $\mu$ be the
  induced outer measure from Proposition~\ref{P:auss} and $\mathcal F$
  the $\sigma$-algebra from Theorem~\ref{T:cara}. Then,
  $\sigma(\mathcal H)\subseteq\mathcal F$ and $\mu$ coincides with $m$
  on $\mathcal H$.
\end{theorem}


Apparently, in order to apply Theorem~\ref{T:masseind}, we need to show that the additive content is $\sigma$-additive. We do this on the level of rings rather than semi-rings. Note that we can extend an additive content on a semi-ring to an additive content on the ring generated by the semi-ring.

\begin{lemma}
Let $\mathcal H$ be a semi-ring. Then,
\begin{align*}
\mathcal R := \Big\{ \bigcup_{j \in J} A_j : J \text{ finite, } A_j \in \mathcal H, j\in J \text{ disjoint}\Big\} \supseteq \mathcal H
\end{align*}
is a ring. Moreover, if $m$ is an additive content on $\mathcal H$, $m' : \mathcal R \to \mathbb R_+$, defined, for disjoint $(A_j)_{j\in J}$ in $\mathcal H$
$$ m'\Big( \bigcup_{j \in J} A_j \Big) := \sum_{j\in J} m(A_j),$$
extends $m$ to $\mathcal R$.
\end{lemma}

Actually, $\sigma$-additivity and $\sigma$-sub-additivity are close relatives. We have \docs{MeasureTheory.addContent_iUnion_eq_sum_of_tendsto_zero}, which states that $\sigma$-additivity is implied by continuity in $\emptyset$, and \docs{MeasureTheory.isSigmaSubadditive_of_addContent_iUnion_eq_tsum}, which states that -- if the additive content is defined on a ring -- it is $\sigma$-sub-additive if it is $\sigma$-additive. Therefore, we need to show continuity in $\emptyset$. For this, we require the notion of inner regularity with respect to a compact system.

We define inner (and outer) regularity of set functions; see \docs{MeasureTheory.Measure.InnerRegularWRT}.

\begin{definition}[Inner regularity] \label{def:innerreg}
  Let $\alpha$ be some set, equipped with a topology, and $m$ be a
  set-function on some $\mathcal H \subseteq 2^\alpha$.
  \begin{enumerate}
  \item Let $p, q : 2^\alpha \to \{\text{true, false}\}$. Then, $m$ is
    called inner regular with respect to $p$ and $q$, if
    $$ m(A) = \sup\{m(F) : p(F) = \text{true}, F \subseteq A\}$$ for
    all $A \in \mathcal H$ with $q(A) = \text{true}$.
  \item If $q(A) = \text{true}$ iff $A$ is measurable, we neglect the
    {\em and $q$}. If $p(A) = \text{true}$ iff $A$ is closed (compact,
    closed and compact), we say that $m$ is inner regular with respect
    to closed (compact, compact and closed) sets.
  \end{enumerate}
\end{definition}

For the next result, recall that for compact sets $C_1, C_2,...$ with
$\bigcap_{n=1}^\infty C_n = \emptyset$, there is some $N$ with
$\bigcap_{n=1}^N C_n = \emptyset$. More generally, compact sets form a
compact system, which is defined as follows; see \leanlink{https://github.com/RemyDegenne/kolmogorov_extension4/blob/a9e2c52f6e2178fc109f3c68d7732627181888b3/KolmogorovExtension4/CompactSystem.lean#L15-L18}{compactSystem}.

\begin{definition}
  Let $\mathcal C \subseteq 2^\alpha$. If, for all $C_1, C_2,...$ with
  $\bigcap_{n=1}^\infty C_n = \emptyset$, there is some $N$ with
  $\bigcap_{n=1}^N C_n = \emptyset$, we call $\mathcal C$ a {\em
    compact system}.
\end{definition}

\noindent
Such compact systems are important since they allow for a proof of continuity at $\emptyset$ (hence $\sigma$-additivity) of a content on a ring, which is the missing piece for applying the Carathéodory Theorem in the proof of the Kolmogorov extension theorem; see \leanlink{https://github.com/RemyDegenne/kolmogorov_extension4/blob/a9e2c52f6e2178fc109f3c68d7732627181888b3/KolmogorovExtension4/RegularContent.lean#L17}{MeasureTheory.tendsto_zero_of_regular_addContent}.

\begin{lemma}\label{l:stetigcompact}
  Let $\mu$ be an additive set function on a ring $\mathcal R$, which contains the compact system $\mathcal C$. If $\mu$ is inner regular with respect to $\mathcal C$, then, $\mu$ is continuous at $\emptyset$.
\end{lemma}

\subsection{Kolmogorov's Extension Theorem}
We aim to apply Theorems~\ref{T:cara} and~\ref{T:masseind} on product spaces. The resulting Kolmogorov's extension theorem is a statement about extending a set-function on a product space to a (finite) measure, where the product space can come with an arbitrary index set.  The next definition covers the important concept of a projective family of measures. In short, we define measures on any finite subset of indices in a consistent way; see \docs{MeasureTheory.IsProjectiveMeasureFamily}.

\begin{definition}[Projective family and projective limit] \mbox{}
  \begin{enumerate}
  % \item For some set $\iota$, we will write $J\subseteq_f \iota$ if $J\subseteq \iota$ and $J$ is finite.
  \item Let $\iota$ be some (index) set and $(\alpha_i)_{i\in\iota}$ a
    family of sets, such that $\mathcal F_i$ is a $\sigma$-algebra on $\alpha_i, i \in \iota$. For $J\subseteq \iota$, we denote $\alpha_J :=
    \prod_{j \in J} \alpha_j$ and $\pi_J : \alpha_\iota \to \alpha_J$
    the projection. For $H\subseteq J \subseteq \iota$, we write
    $\pi_H^J$ for the projection $\alpha_J \to \alpha_H$.
  \item Let $\mathcal F_i$ be a $\sigma$-algebra on $\alpha_i$,
    $i\in\iota$. For $J\subseteq_f \iota$, let $\mathcal F_J$ be the
    (finite) product-$\sigma$-algebra on $\alpha_J$, and $\mathcal F_\iota$ be
    the $\sigma$-algebra generated by cylinder sets
    $\{\pi_J^{-1}\prod_{j\in J} A_j: J \subseteq_f \iota, A_j \in
    \mathcal F_j, j\in J\}$.
  \item A family $(P_J)_{J\subseteq_f I}$, where $P_J$ is a finite
    measure on $\mathcal F_J$, is called projective if
    $$P_H = (\pi_H^{J})_\ast P_J$$ for all
    $H\subseteq J \subseteq_f I$. (Recall that $A \mapsto
    (\pi_H^{J})_\ast P_J(A) := P_J((\pi_H^{J})^{-1}A)$
    is called the image measure of $P_J$ under $\pi_H^{J}$.)
  \item If, for some projective family $(P_J)_{J\subseteq_f \iota}$,
    there is a finite measure $P_\iota$ on $\mathcal F_\iota$ with
    $P_J = (\pi_J)_\ast P_\iota$ for all $J\subseteq_f \iota$, then we
    call $P_\iota$ projective limit of $(P_J)_{J\subseteq_f \iota}$.
  \end{enumerate}
\end{definition}

In order to apply this result and show the extension theorem, we need to show that $\{\pi_J^{-1} C: C \in \prod_{j\in J} \alpha_j \text{ compact and closed}\}$ is a compact system. Note that compact sets are closed in Hausdorff spaces, but we do not have this property since we are working with pseudo-metric spaces, which do not have this property. Since Lemma~\ref{l:stetigcompact} gives the $\sigma$-additivity of a \leanlink{https://github.com/leanprover-community/mathlib4/blob/cfb46701db8bface26df278373eb0a72879273d4/Mathlib/MeasureTheory/Constructions/ProjectiveFamilyContent.lean#L110-L116}{projectiveFamilyContent}, which is defined through the projective family $P$, we have a proof of the Kolmogorov Extension Theorem \leanlink{https://github.com/RemyDegenne/kolmogorov_extension4/blob/a9e2c52f6e2178fc109f3c68d7732627181888b3/KolmogorovExtension4/KolmogorovExtension.lean#L95-L103}{MeasureTheory.projectiveLimitWithWeakestHypotheses}.

\begin{theorem}[Kolmogorov extension]\label{T1}
  For all $t\in\iota$, let $\alpha_t$ be a separable, complete
  pseudo-extended-metric space and $\mathcal F_t$ the Borel
  $\sigma$-algebra generated by its topology. Let $(P_J)_{J\subseteq_f
    \iota}$ be a projective family of finite measures and $P$ be
  defined on $\mathcal A := \bigcup_{J \subseteq_f \iota} \mathcal
  F_J$ given by $P(A) = P_J(\pi_J A)$ for $A\in\mathcal F_J$. Then,
  there is a unique extension of $P$ to $\sigma(\mathcal A)$.
\end{theorem}

Note that we extend standard assumptions (see e.g.\ \cite{Billingsley1995,bogachev2007measure}) in two directions. First, we allow that $\alpha$ is a dependent type, i.e.\ for every index $t \in \iota$, the state space $\alpha_t$ might be different. Second, all $\alpha_t$ are not necessarily separable, complete metric spaces (or Polish, i.e.\ separable and metrizable through a complete metric), but extended pseudo-metric spaces. Such spaces do not satisfy the frequently used Hausdorff (or t2) property, i.e.\ there can be $x\neq y$ such that all open balls around $x$ and $y$ overlap. While the firstgeneralization did not require any change in the proof, the second generalization was only possible since underlying results in mathlib were already provided on the same level of generality.  More precisely, \docs{isCompact_iff_totallyBounded_isComplete}, which shows that a set $A \subseteq \alpha$ is compact iff it is complete and totally bounded, requires $\alpha$ to be a uniform space (recall that every metric space is uniform).  We also require the underlying space(s) to be second-countable. A second-countable uniform space can be made into an (extended) pseudo-metric space (\docs{UniformSpace.pseudoMetrizableSpace}).



\section{Gaussian Measures and characteristic functions}
\label{S:gaussian}

Prior to our work, \mathlib contained the definition of the one-dimensional normal distribution \docs{ProbabilityTheory.gaussianReal}, but not more general Gaussian measures.
We will denote by $N(m, \sigma^2)$ the normal distribution on $\mathbb{R}$ with mean $m$ and variance $\sigma^2$.
In order to define the law of a Brownian motion on $\mathbb{R}_+$, we need definitions and properties of mutivariate Gaussian distributions.
We did not however define Gaussians only for that setting, but first introduced a general definition for Banach spaces, and then specialized to Hilbert and finally finite dimensional spaces.
Our formalization uses many results from \mathlib and we relied on characteristic functions \docs{MeasureTheory.charFunDual}, as well as the fact that they uniquely determine a probability measure \docs{MeasureTheory.Measure.ext_of_charFunDual}\todorm{treat as existing thing or talk about is as something we developped?}.

\todor{Everywhere: be clear about what we formalized and what was already in mathlib.}

We start by defining the characteristic function of a general measure. Fix $E$ a real Banach space, and denote by $E'$ its topological dual, i.e. the set of continuous linear forms on $E$. Consider $\mu$ a measure over $E$, and denote by $\mathbf E_\mu$ and $\mathrm{var}_\mu$ the associated expectation and variance operators.

\begin{definition}
	The \emph{dual characteristic function} (\docs{MeasureTheory.charFunDual}) of $\mu$ is the map $\phi_\mu : E' \to \R$ defined by
	$$\phi_\mu(L) = \int_E e^{i L(x)} \mu(\mathrm dx),$$
	where $i$ is the imaginary unit.
\end{definition}

This function is quite useful because if $E$ is separable and equipped with the Borel $\sigma$-algebra, and if $\mu$ is finite, then it is characterized by $\phi_\mu$ (\docs{MeasureTheory.Measure.ext_of_charFunDual}). Thus it is a great tool to deduce certain properties of the measure. For instance, two random variables are independent if and only if the dual characteristic function of their joint law is the product of the dual characteristic functions of the marginal laws (\docs{ProbabilityTheory.indepFun_iff_charFunDual_prod}).

In the special case where $E$ is a Hilbert space, Riesz's representation theorem implies that each $L \in E'$ can be uniquely represented by an element $t \in E$. This motivates the following definition.

\begin{definition}
  If $E$ is a Hilbert space, the \emph{characteristic function} (\docs{MeasureTheory.charFun}) of $\mu$ is the map $\psi_\mu : E \to \R$ defined by
  $$\psi_\mu(t) = \int_E e^{i \langle t, x \rangle} \mu(\mathrm dx).$$
\end{definition}

This in some sense is a more convenient way to use the dual characteristic function, as it makes computations easier, and ensures the same characterization property.

Let us now turn to Gaussian measures, which are defined through their one-dimensional projections.

\begin{definition}\label{def:gaussian}
  The measure $\mu$ is called \emph{Gaussian} (\docs{ProbabilityTheory.IsGaussian}) if for each $L \in E'$, the pushforward measure $L_* \mu$ is the Gaussian measure $N\left(\mathbf E_\mu[L], \mathrm{var}_\mu(L)\right)$.
\end{definition}

In the case of a real Gaussian measure $N(m, \sigma^2)$, the characteristic function is given by
$$\psi_{N(m, \sigma^2)}(t) = \exp(im - \sigma^2/2).$$
This proves in particular that a real Gaussian measure is determined by its mean and covariance. The same statement is actually true for any Gaussian measure, if we generalize the notion of covariance.
As our work primarily focuses on the case where $E$ is a Euclidean space, in what follows we assume that $E$ is a separable Hilbert space equipped with its Borel $\sigma$-algebra. However most results remain true if $E$ is a Banach space by only changing definitions to refer to the dual space.

\begin{definition}
  The \emph{covariance bilinear form} (\docs{ProbabilityTheory.covarianceBilin}) of $\mu$ is the bilinear form $C_{\mu} : E \times E \to \R$ defined by
  $$C_\mu(s, t) = \mathrm{cov}_\mu(\langle s, \cdot \rangle, \langle t, \cdot \rangle).$$
\end{definition}
\todor{needs a moment condition}

The properties of the covariance allow to easily deduce that $C_\mu$ is actually a continuous and positive semidefinite bilinear form. We have the following result (\docs{ProbabilityTheory.isGaussian_iff_charFun_eq}).

\begin{theorem}\label{thm:gaussian_charFun}
  The measure $\mu$ is Gaussian if and only if its characteristic function is given by
  $$\psi_\mu(t) = \exp(i \langle t, \mathbf E_\mu[\mathrm{id}] \rangle - C_\mu(t, t) / 2).$$
\end{theorem}

This simple description of Gaussian measures allows to easily formalize several important facts about them. For instance, the product $N(0, 1)^{\otimes n}$ of $n \in \mathbb{N}$ times many Gaussian measures $N(0, 1)$ is a Gaussian measure.
Another result is that if two Gaussian random variables have covariance zero, then they are independent.

Theorem~\ref{thm:gaussian_charFun} used implicitly the fact that Gaussian measures have finite moments $(\mathbf E_\mu[\lVert X \rVert^n])_{n \in \mathbb{N}}$.
While this is relatively easy to prove in finite dimension and is all we would need to define a Brownian motion on $\mathbb{R}_+$, we proved a more general result, applicable to Banach spaces, called Fernique's theorem\todorm{ref}.
It states that if $\mu$ is a Gaussian measure over a Banach space $E$, then there exists $\alpha > 0$ such that
\begin{align*}
  \int_E \exp\left( \alpha \|x\|^2 \right) \mu(\mathrm dx) < \infty.
\end{align*}
This implies that all moments of a Gaussian measure in a separable Banach space are finite.

\paragraph{On Euclidean spaces}

As stated before, any Gaussian measure is uniquely determined by its mean and covariance. A follow-up question is to know whether, given $m \in E$ and $C$ a continuous and positive semidefinite bilinear form over $E$, there exists a Gaussian measure with mean $m$ and covariance $C$.
The answer is yes if $E$ is a Euclidean space, and we formalized this construction in the project\todorm{The question is also answerable in Hilbert spaces, but we did not do it}.
As a bilinear form over a finite-dimensional vector-space can be represented by a matrix, we will use the \emph{covariance matrix} to represent the covariance of a measure.

Assume now that $E$ is a Euclidean space with a basis $(e_1, \dots, e_n)$. We define the \emph{standard Gaussian measure} (\docs{ProbabilityTheory.stdGaussian}) as the pushforward measure
$$\left(x \mapsto \sum_{i=1}^n x_i e_i\right)_* N(0, 1)^{\otimes n}.$$

As stated above, $N(0, 1)^{\otimes n}$ is a Gaussian measure because it is a product of Gaussian measures, and the map $x \mapsto \sum_{i=1}^n x_i e_i$ is a linear form over $E$, so that this measure is indeed Gaussian. It is easy to see that its mean is $0$ and its covariance matrix is the identity matrix $I_n$.
Also, the measure we obtain does not depend on the choice of the basis.
We denote it $N(0, I_n)$.

Now from Definition~\ref{def:gaussian} it is clear that the pushforward of a Gaussian measure by a continuous affine map is again a Gaussian measure. In particular, given $A$ an $n \times n$ matrix, $b \in E$ and $\mu$ a Gaussian measure over $E$ with mean $m$ and covariance $C$, we have that $(x \mapsto Ax + b)_* \mu$ is a Gaussian measure with mean $Am + b$ and covariance matrix $A^\top C A$.
Consider thus $b \in E$ and $A$ a positive semidefinite $n \times n$ matrix.
There exist $S \in \mathcal{M}_n(\R)$ such that $A = S^\top S$, and thus $(x \mapsto Sx + b)_* N(0, I_n)$ is a Gaussian measure with mean $m$ and covariance matrix $A$. This proves that for any $m \in E$ and any covariance matrix $A$, there exist a Gaussian measure over $E$ with mean $m$ and covariance matrix $A$. This measure is denoted $N(m, A)$, and is formalized as \docs{ProbabilityTheory.multivariateGaussian}.


\todor{talk here about projectivity of the family of finite-dimensional Gaussian measures?}


\section{The Kolmogorov-Chentsov Theorem}
\label{S:continuity}

The Kolmogorov-Chentsov theorem gives the existence of a Hölder continuous modification of a stochastic process, provided that the process satisfies a certain moment condition called the Kolmogorov condition.


\subsection{Stochastic processes and the Kolmogorov condition}

Let $\Omega$ be a measurable space, on which we define a finite measure $\mathbb{P}$ (with expectation denoted by $\mathbb{E}$).
Let $T$ be a set, and for each $t \in T$, let $E_t$ be a measurable space.
A stochastic process indexed by $T$ with values in $(E_t)_{t \in T}$ is a collection $(X_t)_{t \in T}$ of random variables $X_t : \Omega \to E_t$, which satisfies a measurability condition (see further down).
We will often take $E_t = E$ for all $t \in T$ for some fixed measurable space $E$, but we first prove general results about processes with values in different spaces.

The goal of our formalization of Brownian motion is to define a process with a given law and continuous paths.
The law of a stochastic process is the pushforward of the measure $\mathbb{P}$ by the map $\Omega \to \prod_{t \in T} E_t$ defined by $\omega \mapsto (X_t(\omega))_{t \in T}$.
The paths of the process are the maps $T \to E_t$ defined by $t \mapsto X_t(\omega)$ for each $\omega \in \Omega$.

A process $Y$ is said to be a modification of $X$ if for all $t \in T$, $X_t = Y_t$ almost surely.
$X$ and $Y$ are said to be indistinguishable if almost surely, $X_t = Y_t$ for all $t \in T$ simultaneously.
Being indistinguishable implies being a modification, which in turn implies equality of laws (that last part is proved through finite dimensional distributions, see below).
We will use repeatedly that if $T$ is a separable topological space, $E$ is a Hausdorff and the paths of $X$ and $Y$ are continuous almost surely, then being a modification implies being indistinguishable\todorm{add link}.


\paragraph{Measurability}

Working with stochastic processes requires a form of measurability.
First, we say that $X$ is measurable if $X_t$ is measurable for all $t \in T$.
This is equivalent to requiring that the map $\Omega \to \prod_{t \in T} E_t$ defined by $\omega \mapsto (X_t(\omega))_{t \in T}$ is measurable for the product $\sigma$-algebra on $\prod_{t \in T} E_t$ \todorm{ref to Mathlib?}.

\todor{explain the need for a.e.-measurability. Well explained in the change of variable paper by Sébastien Gouëzel iirc.}

We say that $X$ is almost everywhere measurable if there exists a measurable process $Y$ such that almost surely, $X_t = Y_t$ for all $t \in T$.
That is, there exists a measurable process $Y$ which is indistinguishable from $X$.
Contrary to the case of measurability, this is strictly stronger than requiring that for all $t \in T$, $X_t$ is almost everywhere equal to a measurable function (unless $T$ is countable, in which case the two notions coincide).
In the following, a \emph{stochastic process} will always mean an almost everywhere measurable collection $(X_t)_{t \in T}$.

\paragraph{Finite dimensional distributions}

The finite dimensional distributions of a stochastic process $(X_t)_{t \in T}$ are the pushforwards of $\mathbb{P}$ by the maps $\Omega \to \prod_{i=1}^n E_{t_i}$ defined by $\omega \mapsto (X_{t_1}(\omega),...,X_{t_n}(\omega))$ for all $n \in \mathbb{N}$ and all choices of $t_1,...,t_n \in T$.
Their importance comes from the fact that two stochastic processes have same law if and only if they have the same finite-dimensional distributions, which we prove in TODO LINK by using the uniqueness of the projective limit of a family of measures. Indeed the law of a process is the projective limit of its finite-dimensional distributions.
Our goal in this project is thus to build a process with given finite-dimensional distributions and continuous paths.

\paragraph{The Kolmogorov condition}

Suppose now that $T, E$ are extended pseudo-metric spaces.
That is, $T$ is equipped with a ``distance'' $d_T$ with values in $[0,+\infty]$ (infinity included), which is reflexive, commutative and satisfies the triangle inequality, and a similar distance $d_E$ is defined on $E$. Note that $d_T$ can be zero for pairs of non-identical points.
``extended'' refers to the fact that the distance can take the value $+\infty$, and ``pseudo'' to the possibility of non-identical points having distance zero.
$T$ will denote the index set of a stochastic process, and $E$ the space of values of that process.
The space $E$ is equipped with the Borel $\sigma$-algebra $\mathcal{B}(E)$ generated by the open sets of $E$.

The key assumption of the Kolmogorov-Chentsov theorem that gives the existence of a continuous modification is that the process satisfies the Kolmogorov condition.

\begin{definition}[Kolmogorov condition]\label{def:kolmogorov_condition}
Let $p, q, M$ be non-negative real numbers with $p,q>0$.
A stochastic process $(X_t)_{t \in T}$ is said to satisfy the Kolmogorov condition for exponents $(p, q)$ with constant $M$ if for all $s, t \in T$, the pair $(X_t, X_t) : \Omega \to E \times E$ is measurable for the Borel $\sigma$-algebra on $E \times E$, and
\begin{align*}
  \mathbb{E}[d_E(X_s, X_t)^p] \le M d_T(s, t)^q
  \: .
\end{align*}
\end{definition}

The measurability condition ensures that the distance $d_E(X_s, X_t)$ is a measurable.
If $E$ was second-countable\todorm{write second-countable or separable?}, we could simply assume that $X_s$ is measurable for all $s \in T$, and the measurability of the pair would follow.
For a second-countable space $E$, $\mathcal{B}(E \times E)$ is the product of the Borel $\sigma$-algebras on $E$, which is not true in general.

We remark that the inequality in Definition~\ref{def:kolmogorov_condition} remains true for any modification of the process $X$, and that Theorem~\ref{thm:kolmogorov_chentsov} also holds for a modification of a process satisfying the Kolmogorov condition.
Therefore, we introduce a definition \lstinline|IsAEKolmogorovProcess| for the property of being a modification of a process satisfying the Kolmogorov condition, and prove the theorem under that hypothesis.


\subsection{The main theorem}

We formalize a recent and general version of the Kolmogorov-Chentsov theorem \cite[Theorem 1]{kratschmer2023kolmogorov}, which applies to extended pseudo-metric spaces under a covering assumption on the index set $T$.
We say that $T$ has bounded covering number with constant $c > 0$ and exponent $d > 0$ if for all $\varepsilon \in (0, \mathrm{diam}(T)]$, $T$ can be covered by at most $c \varepsilon^{-d}$ balls of radius $\varepsilon$.

\todor{say somewhere that simpler versions exist in simpler cases, e.g. $T = [0,1]^n$.
Talk briefly about the dyadics proof vs chaining.}


\begin{theorem}[Kolmogorov-Chentsov]\label{thm:kolmogorov_chentsov}
Suppose that the index set $T$ has bounded covering number with constant $c>0$ and exponent $d > 0$.
Let $(X_t)_{t \in T}$ be a stochastic process that satisfies the Kolmogorov condition for exponents $(p,q)$ with constant $M$, with $q > d$ and $p > 0$.
Then for all $\beta \in(0, (q - d)/p)$ there exists a finite constant $L(T, c, d, p, q, \beta)$ such that for every countable subset $T' \subseteq T$,
\begin{align*}
  \mathbb{E}\left[ \sup_{s, t \in T'} \frac{d_E(X_s, X_t)^p}{d_T(s, t)^{\beta p}} \right]
  \le M L(T, c, d, p, q, \beta)
  \: .
\end{align*}
If furthermore $E$ is an extended metric space (not pseudo-metric: $d_E(x,y) = 0$ implies $x = y$) and is complete, and $T$ is second-countable, then the process $X$ has a modification with Hölder continuous paths of exponent $\beta$ for all $\beta \in (0, (q - d)/p)$.
\end{theorem}

\todor{In the code we also have $E$ second-countable, but we are trying to remove it.}

We adopt here \Lean's convention that $0/0 = 0$, which means that the expression under the expectation is well-defined even if we allow $d_T(s, t) = 0$ in the supremum.
Indeed, if $d_T(s, t) = 0$, then the Kolmogorov condition implies that $d_E(X_s, X_t) = 0$ almost surely, so the ratio is then $0/0 = 0$ and the expectation is well-defined.


Note that $T$ is said to be an \emph{extended} pseudo-metric space, but the bounded covering number condition implies that $T$ has finite diameter, hence is a pseudo-metric space.
The \Lean code is nonetheless written for a distance with values in $[0,+\infty]$, for two reasons: first the distance on $E$ has that type and using the same types for both is easier, and second it allows us to work with \mathlib's Lebesgue integral (which expect that co-domain for the functions we integrate).
Working with the Lebesgue integral is easier than using \mathlib's Bochner integral since manipulating the latter often requires proving integrability of the functions we consider.
Therefore, from the beginning of the project we made the design choice to consider any non-negative function as taking values in $[0,+\infty]$, and to use the Lebesgue integral whenever possible.

Our reference \cite{kratschmer2023kolmogorov} proves the theorem for $T$ and $E$ metric spaces.
The most notable difference is that we allow pseudo-metrics, for which $d_T(s, t)$ can be zero for $s \ne t$.
\todorm{The Krätschmer paper also does not mention separability of $T$ for the second part of the theorem, but they take a countable dense subset of $T$ in the proof, which requires $T$ to be second-countable.}

\todor{For reference, if we want to mention it:
\begin{align*}
  L(T, c, d, p, q, \beta)
    = 2^{2p+5q+1} c (1 + \mathrm{diam}(T))^{q - d}
      \sum_{k=0}^\infty 2^{k (\beta p - (q - d))}\left(4^d \max\{0, \log_2 c + (k+2)d\}^q + \max\left\{\frac{1}{2^{q-d} - 1}, \frac{1}{(2^{(q-d)/p} - 1)^p}\right\}\right)
\end{align*}}

The proof we implement follows that paper, with two minor changes.
The first one is that the authors of \cite{kratschmer2023kolmogorov} used the Minkowski inequality in a part of the proof, which requires $p \ge 1$, and did not mention how to handle the (easier) case $p < 1$, although the theorem is also valid there.
Beyond fixing this minor omission by adding a separate argument for $p \in (0,1)$, the second change we make is to ensure that all exponents we use are natural numbers, in order to simplify the formalization.
Whenever the proof in \cite{kratschmer2023kolmogorov} uses a geometric grid $2^{-n_1}, 2^{-n_1-1},\ldots, 2^{-n_2}$ with exponents $n_1 < n_2 \in \mathbb{Z}$ possibly negative, we introduce a multiplicative factor $\varepsilon_0$ and use a grid of the form $\varepsilon_0 2^{-m_1}, \ldots, \varepsilon_0 2^{-m_2}$ with $m_1, m_2 \in \mathbb{N}$\todorm{and why is that desirable?}.

\todor{and there might be other changes: I'm not sure I followed exactly the computations from the paper. I might have refactored some of them.}

\paragraph{Localized version}

In order to obtain a Brownian motion, we will want to apply Theorem~\ref{thm:kolmogorov_chentsov} to a process defined on $T = \mathbb{R}_+$, which is not bounded and does not have bounded covering number.
We thus need an extension of the theorem.
The following localized version of the theorem fulfills this need, and also uses several sequences of exponents $(p_m, q_m)$ instead of a single pair $(p, q)$ to improve the exponents in the local Hölder continuity of the resulting process.
The conclusion of the theorem is not Hölder continuity of the paths, but local Hölder continuity.
A function $f : T \to E$ is said to be locally Hölder continuous of order $\gamma$ if for all $t \in T$, there exists a neighborhood $U$ of $t$ such that $f$ is $\gamma$-Hölder continuous on $U$.

\begin{theorem}[Localized Kolmogorov-Chentsov]\label{thm:localized_holder_modification_sup}
Suppose that $T$ can be covered by an increasing sequence of totally bounded open subsets $(T_n)_{n \in \mathbb{N}}$ such that each $T_n$ has bounded covering number with constant $c_n > 0$ and exponent $d > 0$ (the same exponent for all $n$).
Let $(p_m, q_m)_{m \in \mathbb{N}}$ be a sequence of pairs of positive numbers such that $q_m > d$ for all $m \in \mathbb{N}$.
Let $(X_t)_{t \in T}$ be a process that satisfies the Kolmogorov condition with exponents $(p_m, q_m)$ for all $m \in \mathbb{N}$.
Then $X$ has a modification $Y$ such that the paths of $Y$ are locally Hölder continuous of all orders $\gamma \in \left(0, \sup_m \frac{q_m - d}{p_m}\right)$.
\end{theorem}

The hypothesis on $T$ is satisfied for $T = \mathbb{R}_+$, with $T_n = [0,n)$ (which is open in $\mathbb{R}_+$), for $d = 1$.
Although we use it only for $T = \mathbb{R}_+$, Theorem~\ref{thm:localized_holder_modification_sup} can be used in higher dimension and in more involved cases: the authors of \cite{kratschmer2023kolmogorov} show how to apply it to subsets of $m$-dimensional Riemannian manifolds.

We don't describe the proof of Theorem~\ref{thm:localized_holder_modification_sup} in detail here (we refer the reader to \cite{kratschmer2023kolmogorov}, or to the code\todorm{and blueprint: link}): it consists in applying the same type of arguments as in the proof of the second part of Theorem~\ref{thm:kolmogorov_chentsov} to build separately modifications for each set $T_n$ and exponents ($p_m, q_m)$ and then combine them into one process (see section~\ref{sub:holder_process}).



\subsection{Covers and chaining}



\subsection{Proof of the main inequality}

Under the assumptions of Theorem~\ref{thm:kolmogorov_chentsov}, we prove that for all countable subset $T' \subseteq T$,
\begin{align*}
  \mathbb{E}\left[ \sup_{s, t \in T'} \frac{d_E(X_s, X_t)^p}{d_T(s, t)^{\beta p}} \right]
  \le M L(T, c, d, p, q, \beta)
  \: ,
\end{align*}
for a finite constant $L(T, c, d, p, q, \beta)$ to be determined.

First, it suffices to show the inequality for finite subsets $J \subseteq T$, since the countable case follows by monotone convergence.
Since $J \subseteq T$, it also has bounded covering number with the same exponent $d$ and a different constant $2^d c$\todorm{ref}.


\todor{how much detail do we want? Certainly we don't want to repeat the paper that we follow.}

\todor{Any formalization insights in that part? Proving that L is finite was painful (but there is a tactic coming that will help).}



\subsection{Building a Hölder continuous process}
\label{sub:holder_process}

We discuss the proof of the second part of Theorem~\ref{thm:kolmogorov_chentsov}, which states that under additional assumptions on $E$ and $T$, the process $X$ has a modification with Hölder continuous paths.
We first prove the result for a fixed $\beta \in (0, (q - d)/p)$, and will then deduce it for all $\beta$ in that interval simultaneously.
An advantage of the formal proof is that it is easy to inspect the code to see which assumptions are used where.
This part of the theorem requires $T$ to be second-countable because we first take a countable dense subset $T'$ of $T$, on which we apply the main inequality of Theorem~\ref{thm:kolmogorov_chentsov}.
We obtain that an expectation is finite, hence that the quantity inside the expectation is almost surely finite:
\begin{align*}
  \sup_{s, t \in T'} \frac{d_E(X_s, X_t)^p}{d_T(s, t)^{\beta p}} < +\infty
  \quad \text{almost surely.}
\end{align*}

On the event $A$ that the supremum is finite, $(X_t)_{t \in T'}$ has Hölder continuous paths of order $\beta$.
Let $x_0 \in E$ be arbitrary and let $Y$ be the process defined by
\begin{align*}
  Y_t(\omega)
  &= \begin{cases}
    \lim_{s \to t, s \in T'} X_s(\omega) & \text{if } \omega \in A \: , \\
    x_0 & \text{otherwise .}
  \end{cases}
\end{align*}
The limits in that definition are the reason for the completeness assumption on $E$.
Then $Y$ has Hölder continuous paths of order $\beta$ almost surely.
We finally need to show that $Y$ is a modification of $X$, which we do by showing that for all $t \in T$, $d_E(X_t, Y_t) = 0$ almost surely.
That is why that part of the theorem requires $E$ to be metric and not pseudo-metric.

\todor{explain the last part of the proof? refer to the paper? It caused us to refactor convergence in probability in Mathlib to allow extended metric spaces.}

Now that we have a modification $Y$ with Hölder continuous paths of order $\beta$ for a fixed $\beta \in (0, (q - d)/p)$, we want to obtain a modification with Hölder continuous paths of order $\gamma$ for all $\gamma \in (0, (q - d)/p)$ simultaneously.
Note that since $T$ has finite diameter, being Hölder continuous of order $\beta$ implies being Hölder continuous of order $\gamma$ for all $\gamma \in (0, \beta]$\todorm{add ref?}.

Let $\beta_n$ be an increasing sequence in $(0, (q - d)/p)$ converging to $(q - d)/p$.
For each $n$, we obtain a modification $Y^n$ with Hölder continuous paths of order $\beta_n$.
Those $Y^n$ are all modifications of each other and are continuous, hence indistinguishable.
Thus almost surely, for all $t \in T$, $Y^n_t = Y^1_t$ for all $n$.
We can thus define a process $Z$ by $Z_t(\omega) = Y^1_t(\omega)$ if $\omega$ is in the event where all $Y^n$ are equal, and $Z_t(\omega) = x_0$ otherwise.
Then $Z$ is a modification of $X$ and has Hölder continuous paths of order $\beta_n$ for all $n$.
It thus has Hölder continuous paths of order $\gamma$ for all $\gamma \in (0, (q - d)/p)$.

\todor{Here talk about the measurability issue and the need to modify the main theorem to include an additional property of the modification.}



\section{Construction of a Brownian motion and a Wiener measure}
\label{S:BM}

All the results presented so far can be used together to construct a Brownian motion indexed by $\mathbb{R}_+$ with values in $\mathbb{R}$.
We build a stochastic process $(B_t)_{t \in \mathbb{R}_+}$ such that
\begin{itemize}
  \item $B$ has continuous paths,
  \item $B_0 = 0$ almost surely,
  \item for all $n \in \mathbb{N}$ and $t_1, \ldots, t_n \in \mathbb{R}_+$, the vector $(B_{t_1}, \ldots, B_{t_n})$ has a multivariate normal distribution with mean $0$ and covariance matrix given by $\mathrm{cov}(B_{t_i}, B_{t_j}) = t_i \wedge t_j$.
\end{itemize}

We build the finite-dimensional distributions of $B$ using the multivariate normal distribution defined in section~\ref{S:gaussian}: the prescribed covariance matrix is positive semi-definite since it can be written as a Gram matrix, which means that the multivariate normal distribution with mean 0 and that covariance matrix is well-defined.
We then check that they form a projective family and apply the Kolmogorov extension theorem (Theorem~\ref{T1}) to obtain a measure $P_B$ on $\mathbb{R}^{\mathbb{R}_+}$ which is the projective limit of that family.

Let $(C_t)_{t \in \mathbb{R}_+}$ be the canonical process on $\mathbb{R}^{\mathbb{R}_+}$, defined by $C_t(\omega) = \omega(t)$ for all $\omega \in \mathbb{R}^{\mathbb{R}_+}$.
The process $C$ is measurable, and its finite-dimensional distributions in the measure space $(\mathbb{R}^{\mathbb{R}_+}, P_B)$ are the projective family we defined.

Finally, we check that $C$ satisfies the Kolmogorov condition for exponents $(2n, n)$ for all $n \in \mathbb{N}$ with constant $M = (2n - 1)!!$ (the double factorial, product of all odd integers up to $2n - 1$).
We can then apply Theorem~\ref{thm:localized_holder_modification_sup} to obtain a modification $B$ of $C$ with locally Hölder continuous paths of all orders $\gamma \in (0, 1/2)$.
In particular, $B$ has continuous paths.
The process $B$ is a Brownian motion indexed by $\mathbb{R}_+$ with values in $\mathbb{R}$.


\todor{We need to list properties about our Brownian motion to certify that it is indeed a Brownian motion. Independent increments, etc.

It would be nice to finish the code the Brownian characterization and talk about it here.}


\paragraph{Wiener measure}

The Brownian motion $B$ can be seen as a random variable on the space $\mathbb{R}^{\mathbb{R}_+}$, with the product $\sigma$-algebra and the measure $P_B$.
It also has continuous paths, and thus is a random variable on the subspace of continuous functions of $\mathbb{R}^{\mathbb{R}_+}$, with the subset $\sigma$-algebra coming from the product $\sigma$-algebra on $\mathbb{R}^{\mathbb{R}_+}$.

We may want to consider instead the space of continuous functions $C(\mathbb{R}_+, \mathbb{R})$, equipped with the Borel $\sigma$-algebra generated the topology of uniform convergence on compact sets.

We prove that those two $\sigma$-algebras on continuous functions from a topological space $X$ to another $Y$ are equal\todorm{ref} whenever $X$ is a second-countable, locally compact space and $Y$ is second-countable and regular, equipped with the Borel $\sigma$-algebra.
These assumptions are satisfied for $X = \mathbb{R}_+$ and $Y = \mathbb{R}$.
Thus $B$ is a random variable on $(C(\mathbb{R}_+, \mathbb{R}), \mathcal{B}(C(\mathbb{R}_+, \mathbb{R})))$.
The law of $B$ in that space is called the Wiener measure.

\section*{Acknowledgments}

Other contributors, who made one or two PRs: Jonas Bayer, Lorenzo Luccioli, Alessio Rondelli, Jérémy Scanvic

Lean/Mathlib community, Mathlib reviewers (Sébastien Gouëzel in particular?)

%%
%% Bibliography
%%

\printbibliography

\end{document}
