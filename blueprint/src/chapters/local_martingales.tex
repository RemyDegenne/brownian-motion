\chapter{Local martingales}

\section{Local properties}

This section contains material taken mostly from \cite[Chapters 10 and 18]{kallenberg2021} and \cite{almostsuremath}.


\begin{definition}\label{def:preLocalizingSequence}
  \uses{def:IsStoppingTime}
  \leanok
  \lean{ProbabilityTheory.IsPreLocalizingSequence}
A pre-localizing sequence is a sequence of stopping times $(\tau_n)_{n \in \mathbb{N}}$ such that $\tau_n \to \infty$ as $n \to \infty$ (a.s.).
\end{definition}


\begin{definition}[Localizing sequence]\label{def:localizingSequence}
  \uses{def:preLocalizingSequence}
  \leanok
  \lean{ProbabilityTheory.IsLocalizingSequence}
A localizing sequence is a sequence of stopping times $(\tau_n)_{n \in \mathbb{N}}$ such that $\tau_n$ is non-decreasing and $\tau_n \to \infty$ as $n \to \infty$ (a.s.).
That is, it is a pre-localizing sequence that is also almost surely non-decreasing.
\end{definition}


\begin{lemma}\label{lem:localizingSequence_const_top}
  \uses{def:localizingSequence}
  \leanok
  \lean{ProbabilityTheory.isLocalizingSequence_const_top}
The constant sequence $\tau_n = \infty$ is a localizing sequence.
\end{lemma}

\begin{proof}\leanok

\end{proof}


\begin{lemma}\label{lem:localizingSequence_min}
  \uses{def:localizingSequence}
  \leanok
  \lean{ProbabilityTheory.IsLocalizingSequence.min}
Let $(\sigma_n), (\tau_n)$ be localizing sequences.
Then $(\sigma_n \wedge \tau_n)$ is a localizing sequence.
\end{lemma}

\begin{proof}\leanok

\end{proof}


\begin{definition}[Local property]\label{def:locally}
  \uses{def:localizingSequence, def:stoppedProcess}
  \leanok
  \lean{ProbabilityTheory.Locally, ProbabilityTheory.Locally.localSeq}
Let $P$ be a class of stochastic processes (or equivalently a predicate on stochastic processes).
We say that a stochastic process $X : T \to \Omega \to E$ is locally in $P$ (or satisfies $P$ locally) if there exists a localizing sequence $(\tau_n)_{n \in \mathbb{N}}$ such that for all $n \in \mathbb{N}$, the process $X^{\tau_n}\mathbb{I}_{\tau_n > 0}$ is in $P$ (in which $X^{\tau_n}$ denotes the stopped process).
We denote the class of processes that are locally in $P$ by $P_{\mathrm{loc}}$.
\end{definition}


\begin{lemma}\label{lem:implies_locally}
  \uses{def:locally}
  \leanok
  \lean{ProbabilityTheory.locally_of_prop}
For any class of processes $P$, we have $P \subseteq P_{\mathrm{loc}}$.
\end{lemma}

\begin{proof}\leanok
  \uses{lem:localizingSequence_const_top}
Take $\tau_n = \infty$ for all $n$.
\end{proof}


\begin{lemma}\label{lem:locally_mono}
  \uses{def:locally}
  \leanok
  \lean{ProbabilityTheory.Locally.mono}
If $P \subseteq Q$ then $P_{\mathrm{loc}} \subseteq Q_{\mathrm{loc}}$.
\end{lemma}

\begin{proof}\leanok
Let $X \in P_{\mathrm{loc}}$.
Then there exists a localizing sequence $(\tau_n)_{n \in \mathbb{N}}$ such that for all $n \in \mathbb{N}$, $X^{\tau_n}\mathbb{I}_{\tau_n > 0} \in P$.
Since $P \subseteq Q$, for all $n \in \mathbb{N}$, $X^{\tau_n}\mathbb{I}_{\tau_n > 0} \in Q$.
Thus $X \in Q_{\mathrm{loc}}$.
\end{proof}


\begin{definition}\label{def:stable}
  \uses{def:locally}
  \leanok
  \lean{ProbabilityTheory.IsStable}
A class of stochastic processes $P$ is stable if whenever $X$ is in $P$, then for any stopping time $\tau$, the process $X^{\tau}\mathbb{I}_{\tau > 0}$ is also in $P$.
\end{definition}


\begin{lemma}\label{lem:isStable_locally}
  \uses{def:locally, def:stable}
  \leanok
  \lean{ProbabilityTheory.IsStable.isStable_locally}
If $P$ is a stable class of processes, then $P_{\mathrm{loc}}$ is also stable.
\end{lemma}

\begin{proof}\leanok

\end{proof}


\begin{lemma}\label{lem:locally_inter}
  \uses{def:locally, def:stable}
  \leanok
  \lean{ProbabilityTheory.locally_and}
If $P, Q$ are stable classes of processes then $(P\cap Q)_{\mathrm{loc}} = P_{\mathrm{loc}}\cap Q_{\mathrm{loc}}$.
\end{lemma}

\begin{proof}\leanok
  \uses{lem:localizingSequence_min}
The forward direction is trivial so we only provide proof for the reverse.

Suppose that $X \in P_{\mathrm{loc}}\cap Q_{\mathrm{loc}}$. Then, there exists localizing sequences $(\tau_n)_{n \in \mathbb{N}}$ and $(\sigma_n)_{n \in \mathbb{N}}$ such that $X^{\tau_n} \mathbb{I}_{\tau_n > 0}\in P$ and $X^{\sigma_n} \mathbb{I}_{\sigma_n > 0} \in Q$. Consequently, by the stability of $P$,
\[X^{\sigma_n \wedge \tau_n} \mathbb{I}_{\sigma_n \wedge \tau_n > 0} = (X^{\tau_n} \mathbb{I}_{\tau_n > 0})^{\sigma_n \wedge \tau_n} \mathbb{I}_{\sigma_n \wedge \tau_n > 0} \in P.\]
Similarly, by the stability of $Q$, $X^{\sigma_n \wedge \tau_n} \mathbb{I}_{\sigma_n \wedge \tau_n > 0} \in Q$. Thus, as $\sigma_n \wedge \tau_n$ is a localizing sequence by Lemma~\ref{lem:localizingSequence_min} and $X^{\sigma_n \wedge \tau_n} \mathbb{I}_{\sigma_n \wedge \tau_n > 0} \in P \cap Q$ for all $n$, it follows that $X \in (P \cap Q)_{\mathrm{loc}}$
\end{proof}

\begin{lemma}\label{lem:isLocalizingSequence_of_isPreLocalizingSequence}
  \uses{def:localizingSequence, def:rightContinuous}
  \leanok
  \lean{ProbabilityTheory.isLocalizingSequence_of_isPreLocalizingSequence}
If $(\tau_n)_{n \in \mathbb{N}}$ is a pre-localizing sequence, then the sequence defined by $\tau'_n = \inf_{m \ge n} \tau_m$ is a localizing sequence.
\end{lemma}

\begin{proof}\leanok

\end{proof}


\begin{lemma}\label{lem:locally_of_isPreLocalizingSequence}
  \uses{def:locally, def:localizingSequence, def:stable, def:rightContinuous, def:preLocalizingSequence}
  \leanok
  \lean{ProbabilityTheory.locally_of_isPreLocalizingSequence}
Let $P$ be a stable class of processes and let $(\tau_n)_{n \in \mathbb{N}}$ be a pre-localizing sequence such that for all $n \in \mathbb{N}$, $X^{\tau_n}\mathbb{I}_{\tau_n > 0}$ is in $P$.
If the filtration is right-continuous, then $X$ is locally in $P$.
\end{lemma}

\begin{proof}\leanok
  \uses{lem:isLocalizingSequence_of_isPreLocalizingSequence}
  Using the localizing sequence defined by Lemma~\ref{lem:isLocalizingSequence_of_isPreLocalizingSequence} suffices.
\end{proof}


\begin{lemma}\label{lem:isPreLocalizingSequence_of_isLocalizingSequence}
  \uses{def:preLocalizingSequence, def:localizingSequence}
  \leanok
  \lean{ProbabilityTheory.isPreLocalizingSequence_of_isLocalizingSequence}
Let $(\tau_n)_{n \in \mathbb{N}}$ be a localizing sequence and let $(\sigma_{n,k})_{k \in \mathbb{N}}$ be a localizing sequence for each $n$.
Then, there exists a strictly increasing sequence $(k_n)_{n \in \mathbb{N}}$ such that the sequence defined by $\tau'_n = \tau_n \wedge \sigma_{n,k_n}$ is a pre-localizing sequence.
\end{lemma}

\begin{proof}\leanok
  For each $n$, since $\sigma_{n,k} \to \infty$ a.s. as $k \to \infty$, we may choose $k_n \in \mathbb{N}$ such that $P(\sigma_{n,k_n} < \tau_n \wedge n) \le 2^{-n}$.
  Then, defining $\tau'_n = \tau_n \wedge \sigma_{n,k_n}$, we have $\tau_n' \to \infty$ by the Borel-Cantelli lemma.
\end{proof}


\begin{lemma}\label{lem:locally_locally}
  \uses{def:locally, def:stable}
  \leanok
  \lean{ProbabilityTheory.locally_locally}
Suppose that the filtration is right-continuous.
For any stable class of processes $P$, we have $(P_{\mathrm{loc}})_{\mathrm{loc}} = P_{\mathrm{loc}}$.
\end{lemma}

\begin{proof}\leanok
  \uses{lem:locally_of_isPreLocalizingSequence, lem:isStable_locally, lem:isPreLocalizingSequence_of_isLocalizingSequence}
$(P_{\mathrm{loc}})_{\mathrm{loc}} \supseteq P_{\mathrm{loc}}$ by Lemma~\ref{lem:isStable_locally} so we only prove the reverse inclusion.

Let $X$ be a process in $(P_{\mathrm{loc}})_{\mathrm{loc}}$.
By definition there exists a localizing sequence $(\tau_n)_{n \in \mathbb{N}}$ such that for all $n \in \mathbb{N}$, $X^{\tau_n}\mathbb{I}_{\tau_n > 0}$ is in $P_{\mathrm{loc}}$.
By definition of $P_{\mathrm{loc}}$, for each $n$ there exists a localizing sequence $(\sigma_{n,k})_{k \in \mathbb{N}}$ such that for all $k \in \mathbb{N}$, $(X^{\tau_n}\mathbb{I}_{\tau_n > 0})^{\sigma_{n,k}}\mathbb{I}_{\sigma_{n,k} > 0}$ is in $P$.

By Lemma~\ref{lem:locally_of_isPreLocalizingSequence}, it suffices to show that there exists a pre-localizing sequence $(\tau'_n)_{n \in \mathbb{N}}$ such that for all $n \in \mathbb{N}$, $X^{\tau'_n}\mathbb{I}_{\tau'_n > 0}$ is in $P$.
Thus, using the localizing sequences $\tau'_n = \tau_n \wedge \sigma_{n, k_n}$ defined by Lemma~\ref{lem:isPreLocalizingSequence_of_isLocalizingSequence},
it remains to argue that by stability of $P$, $X^{\tau'_n}\mathbb{I}_{\tau'_n > 0}$ is in $P$ for all $n$.
Indeed, this follows as $X^{\tau'_n}\mathbb{I}_{\tau'_n > 0} = ((X^{\tau_n}\mathbb{I}_{\tau_n > 0})^{\sigma_{n,k_n}}\mathbb{I}_{\sigma_{n,k_n} > 0})^{\tau'_n}\mathbb{I}_{\tau'_n > 0}$ where $(X^{\tau_n}\mathbb{I}_{\tau_n > 0})^{\sigma_{n,k_n}}\mathbb{I}_{\sigma_{n,k_n} > 0}$ is in $P$ by construction and $P$ is stable.
\end{proof}


\begin{lemma}[Local implication from global implication]\label{lem:local_induction}
  \uses{def:locally, def:stable}
  \leanok
  \lean{ProbabilityTheory.locally_induction}
Suppose that the filtration is right-continuous.
Let $P, Q$ be two classes of stochastic processes such that $P \subseteq Q_{\mathrm{loc}}$ and $Q$ is stable.
Let $X$ be a stochastic process that satisfies $P$ locally.
Then $X$ satisfies $Q$ locally.
In short, if $P$ implies $Q$ locally, then $P$ locally implies $Q$ locally.
\end{lemma}

\begin{proof}\leanok
  \uses{lem:locally_locally, lem:locally_mono}
Since $X \in P_{\mathrm{loc}}$, then $X \in (Q_{\mathrm{loc}})_{\mathrm{loc}}$ by assumption and Lemma~\ref{lem:locally_mono}.
By Lemma \ref{lem:locally_locally}, $(Q_{\mathrm{loc}})_{\mathrm{loc}} = Q_{\mathrm{loc}}$.
Thus $X \in Q_{\mathrm{loc}}$.
\end{proof}


\section{Local martingales}


\begin{definition}[Local martingale]\label{def:IsLocalMartingale}
  \uses{def:Martingale, def:locally}
  \leanok
  \lean{ProbabilityTheory.IsLocalMartingale}
A stochastic process is a local martingale if it is locally a martingale in the sense of Definition~\ref{def:locally}.
That is, there exists a localizing sequence $(\tau_n)_{n \in \mathbb{N}}$ such that for all $n \in \mathbb{N}$, the process $M^{\tau_n}\mathbb{I}_{\tau_n > 0}$ is a martingale.
\end{definition}


\begin{definition}\label{def:IsLocalSubmartingale}
  \uses{def:Submartingale, def:localizingSequence, def:stoppedProcess}
  \leanok
  \lean{ProbabilityTheory.IsLocalSubmartingale}
A stochastic process is a local submartingale if it is locally a submartingale in the sense of Definition~\ref{def:locally}.
That is, there exists a localizing sequence $(\tau_n)_{n \in \mathbb{N}}$ such that for all $n \in \mathbb{N}$, the process $M^{\tau_n}\mathbb{I}_{\tau_n > 0}$ is a submartingale.
\end{definition}


\begin{lemma}\label{lem:Martingale.IsLocalMartingale}
  \uses{def:IsLocalMartingale}
  \leanok
  \lean{ProbabilityTheory.Martingale.IsLocalMartingale}
Every martingale is a local martingale.
\end{lemma}

\begin{proof}\leanok
  \uses{lem:implies_locally}
This follows from Lemma \ref{lem:implies_locally}.
\end{proof}


\begin{lemma}\label{lem:stable_IsMartingale}
  \uses{def:Martingale, def:stable}
  \leanok
  \lean{ProbabilityTheory.isStable_martingale}
The class of martingales is stable. That is, if $M$ is a martingale and $\tau$ is a stopping time, then the stopped process $M^{\tau}\mathbb{I}_{\tau > 0}$ is also a martingale.
\end{lemma}

\begin{proof}
  \uses{lem:optionalSampling}
  Fixing $s \le t \in T$, as $\{\tau > 0\} \in \mathcal{F}_0 \subseteq \mathcal{F}_s$, we have
  $$P[M^{\tau}_t \mathbb{I}_{\tau > 0} \mid \mathcal{F}_s] = \mathbb{I}_{\tau > 0}P[M_{\tau \wedge t} \mid \mathcal{F}_{s}].$$
  Thus, as $\tau \wedge t$ is a bounded stopping time, we have by the optional stopping theorem
  (Lemma~\ref{lem:optionalSampling}) that $P[M_{\tau \wedge t} \mid \mathcal{F}_{s}] = M_{(\tau \wedge t) \wedge s} = M_{\tau \wedge s}$
  and so, $P[M^{\tau}_t \mathbb{I}_{\tau > 0} \mid \mathcal{F}_s] = M^{\tau}_s \mathbb{I}_{\tau > 0}$ as required.
\end{proof}


\begin{lemma}\label{lem:stable_IsSubmartingale}
  \uses{def:Submartingale, def:stable}
  \leanok
  \lean{ProbabilityTheory.isStable_submartingale}
The class of submartingales is stable. That is, if $M$ is a submartingale and $\tau$ is a stopping time, then the stopped process $M^{\tau}\mathbb{I}_{\tau > 0}$ is also a submartingale.
\end{lemma}

\begin{proof}
  \uses{lem:optionalSampling}

\end{proof}


\begin{theorem}\label{thm:IsLocalMartingale.eq_zero_of_finiteVariation}
  \uses{def:IsLocalMartingale}
Let $M$ be a continuous local martingale with $M_0 = 0$. If $M$ is also a finite variation process, then $M_t = 0$ for all $t$.
\end{theorem}

\begin{proof}

\end{proof}




\section{Doob-Meyer class}


\begin{definition}\label{def:locallyIntegrableSup}
  \uses{def:locally}
  \leanok
  \lean{ProbabilityTheory.HasLocallyIntegrableSup}
We say that a stochastic process is integrable if for all $t$, $X_t$ is integrable.
A process has locally integrable supremum if $(\sup_{s \le t} \Vert X_s \Vert)_t$ is locally integrable.
\end{definition}


\begin{definition}[Doob-Meyer class, class D]\label{def:classD}
  \uses{def:IsStoppingTime}
A stochastic process $(X_t)$ is of class D (or in the Doob-Meyer class) if it is adapted and the set $\{X_\tau \mid \tau \text{ is a finite stopping time}\}$ is uniformly integrable.
\end{definition}


\begin{definition}[Class DL]\label{def:classDL}
  \uses{def:IsStoppingTime}
A stochastic process $(X_t)$ is of class DL if it is adapted and for all $t \ge 0$, the set $\{X_\tau \mid \tau \text{ is a stopping time with } \tau \le t\}$ is uniformly integrable.
\end{definition}


\begin{lemma}\label{lem:Submartingale.classDL}
  \uses{def:Submartingale, def:classDL}
Every positive cadlag submartingale is of class DL.
\end{lemma}

\begin{proof}

\end{proof}


\begin{lemma}\label{lem:Submartingale.classD_iff_uniformIntegrable}
  \uses{def:Submartingale, def:classD}
A positive cadlag submartingale is of class D if and only if it is uniformly integrable
\end{lemma}

\begin{proof}
  \uses{lem:Submartingale.classDL}

\end{proof}


\begin{lemma}\label{lem:Martingale.classDL}
  \uses{def:Martingale, def:classDL}
Every cadlag martingale is of class DL.
\end{lemma}

\begin{proof}
  \uses{lem:Submartingale.classDL}

\end{proof}

\begin{lemma}\label{lem:Martingale.classD_iff_uniformIntegrable}
  \uses{def:Martingale, def:classD}
A cadlag martingale is of class D if and only if it is uniformly integrable.
\end{lemma}

\begin{proof}
  \uses{lem:Martingale.classDL, lem:Submartingale.classD_iff_uniformIntegrable}

\end{proof}


\begin{lemma}\label{lem:isStable_classD}
  \uses{def:stable, def:classD}
The class D is stable.
\end{lemma}

\begin{proof}

\end{proof}


% todo: this and the next lemma are proved together, with whatever chain of implications is easiest
\begin{lemma}\label{lem:locally_classD_iff_locallyIntegrableSup}
  \uses{def:classD, def:locallyIntegrableSup, def:locally}
A cadlag adapted process is locally of class D if and only if it has locally integrable supremum.
\end{lemma}

\begin{proof}

\end{proof}


% todo: this and the previous lemma are proved together, with whatever chain of implications is easiest
\begin{lemma}\label{lem:locally_classD_iff_locally_classDL}
  \uses{def:classD, def:classDL, def:locally}
A cadlag adapted process is locally of class D if and only if it is locally of class DL.
\end{lemma}

\begin{proof}

\end{proof}


\begin{lemma}\label{lem:sup_stoppedProcess_le}
  \uses{def:stoppedProcess}
For $Y$ a stochastic process, let $Y^*_t = \sup_{s \le t} \Vert Y_s \Vert$.
Let $X$ be a stochastic process and let $\tau = \inf \{t \mid \Vert X_t \Vert \ge n\}$ for some $n \in \mathbb{R}$.
Then
\begin{align*}
  (X^{\tau})^*_t
  \le n + \mathbb{1}_{\tau \le t} \Vert X_{\tau} \Vert
  \: .
\end{align*}
\end{lemma}

\begin{proof}

\end{proof}


\begin{lemma}\label{lem:Submartingale.locallyIntegrableSup}
  \uses{def:Submartingale, def:locallyIntegrableSup}
Every cadlag submartingale for a right-continuous filtration has locally integrable supremum.
\end{lemma}

\begin{proof}
  \uses{lem:sup_stoppedProcess_le, lem:Submartingale.integrable_stoppedValue, cor:isStoppingTime_leastGE_of_rightContinuous}
Define the stopping times $\sigma_n = \inf\{t \mid \Vert X_t \Vert \ge n\}$ and set $\tau_n = \sigma_n \wedge n$.
$\sigma_n$ is a stopping time by Lemma~\ref{cor:isStoppingTime_leastGE_of_rightContinuous} and so $\tau_n$ is also a stopping time.
The times $(\tau_n)_{n \in \mathbb{N}}$ form a localizing sequence.
By Lemma~\ref{lem:sup_stoppedProcess_le}, we have that
\begin{align*}
  (X^{\tau_n})^*_t
  & \le n + \mathbb{1}_{\tau_n \le t} \Vert X_{\tau_n} \Vert \\
  & \le n + \Vert X_{\tau_n} \Vert
  \: .
\end{align*}
It remains to show that $X_{\tau_n}$ is integrable for each $n$.
This follows by Lemma~\ref{lem:Submartingale.integrable_stoppedValue} as $\tau_n$ is a bounded stopping time.
\end{proof}



\begin{lemma}\label{lem:Submartingale.locally_classD}
  \uses{def:Submartingale, def:classD, def:locally}
Every cadlag submartingale is locally of class D.
\end{lemma}

\begin{proof}
  \uses{lem:Submartingale.locallyIntegrableSup, lem:locally_classD_iff_locallyIntegrableSup}
By Lemma~\ref{lem:locally_classD_iff_locallyIntegrableSup}, it suffices to show that every cadlag submartingale has locally integrable supremum.
This is done in Lemma~\ref{lem:Submartingale.locallyIntegrableSup}.
\end{proof}


\begin{lemma}\label{lem:IsLocalSubmartingale.locally_classD}
  \uses{def:IsLocalSubmartingale, def:classD, def:locally}
Every local submartingale is locally of class D.
\end{lemma}

\begin{proof}
  \uses{lem:local_induction, lem:stable_IsSubmartingale, lem:Submartingale.locally_classD, lem:isStable_classD}
By Lemma~\ref{lem:local_induction}, it suffices to show that if $X$ is a submartingale then it is locally of class D.
This is done in Lemma~\ref{lem:Submartingale.locally_classD}.
\end{proof}


\begin{lemma}\label{lem:IsLocalMartingale.locally_classD}
  \uses{def:IsLocalMartingale, def:classD, def:locally}
Every local martingale is locally of class D.
\end{lemma}

\begin{proof}

\end{proof}


\begin{lemma}\label{lem:IsLocalMartingale.martingale_iff_classDL}
  \uses{def:IsLocalMartingale, def:classDL, def:Martingale}
A local martingale is a martingale if and only if it is of class DL.
\end{lemma}

\begin{proof}

\end{proof}


\begin{lemma}\label{lem:IsLocalSubmartingale.submartingale_iff_classDL_of_nonnegative}
  \uses{def:IsLocalSubmartingale, def:classDL, def:Submartingale}
A nonnegative local submartingale is a submartingale if and only if it is of class DL.
\end{lemma}

\begin{proof}

\end{proof}
