\chapter{{Filtrations, processes and martingales}}\label{chap:filtration_martingale}


First, recall the definitions of a filtration, an adapted process, a (sub)martingale, a stopping time and a stopped process, which are already in Mathlib.


\begin{definition}[Filtration]\label{def:filtration}
  \mathlibok
  \lean{MeasureTheory.Filtration}
A filtration on a measurable space $(\Omega, \mathcal{A})$ with measure $P$ indexed by a preordered set $T$ is a family of sigma-algebras $\mathcal{F} = (\mathcal{F}_t)_{t \in T}$ such that for all $i \le j$, $\mathcal{F}_i \subseteq \mathcal{F}_j$ and for all $t \in T$, $\mathcal{F}_t \subseteq \mathcal{A}$.
\end{definition}


\begin{definition}\label{def:adapted}
  \uses{def:filtration}
  \mathlibok
  \lean{MeasureTheory.Adapted}
A process $X : T \to \Omega \to E$ is said to be adapted with respect to a filtration $\mathcal{F}$ if for all $t \in T$, $X_t$ is $\mathcal{F}_t$-measurable.
\end{definition}


\begin{definition}[Progressively measurable]\label{def:ProgMeasurable}
  \uses{def:filtration}
  \mathlibok
  \lean{MeasureTheory.ProgMeasurable}
A stochastic process $X$ is said to be progressively measurable with respect to a filtration $\mathcal{F}$ if at each point in time $i$, $X$ restricted to $(-\infty, i] \times \Omega$ is measurable with respect to the product $\sigma$-algebra where the $\sigma$-algebra used for $\Omega$ is $\mathcal{F}_i$.
\end{definition}


\begin{lemma}\label{lem:Adapted.progMeasurable_of_rightContinuous}
  \uses{def:adapted, def:ProgMeasurable}
  \leanok
  \lean{MeasureTheory.Adapted.progMeasurable_of_rightContinuous}
If a stochastic process $(X_t)_{t \in T}$ is right continuous and adapted, then it is progressively measurable.
\end{lemma}

\begin{proof}
  Fixing $t \in T$, we need to show that $X$ restricted to $[0, t] \times \Omega$ is measurable with respect to $\mathcal{B}([0, t]) \otimes \mathcal{F}_t$.
  To this end, we define a left continuous discrete approximation of $X$ on $[0, t]$ by
  $$X^n_s = X_{(k + 1)t 2^{-n}} \text{ for } s \in (k t 2^{-n}, (k + 1) t 2^{-n}]$$
  and $X^n_0 = X_0$. As $X$ is right continuous, it is easy to see that $X^n \to X$ pointwise as $n \to \infty$.
  Thus, as each $X^n$ is progressively measurable, it follows that $X$ is also progressively measurable
  (by e.g. using \verb|MeasureTheory.progMeasurable_of_tendsto|).
\end{proof}


\begin{definition}[Martingale]\label{def:Martingale}
  \uses{def:adapted}
  \mathlibok
  \lean{MeasureTheory.Martingale}
Let $\mathcal{F}$ be a filtration on a measurable space $\Omega$ with measure $P$ indexed by $T$.
A family of functions $M : T \to \Omega \to E$ is a martingale with respect to a filtration $\mathcal{F}$ if $M$ is adapted with respect to $\mathcal{F}$ and for all $i \le j$, $P[M_j \mid \mathcal{F}_i] = M_i$ almost surely.
\end{definition}


\begin{definition}[Submartingale]\label{def:Submartingale}
  \uses{def:adapted}
  \mathlibok
  \lean{MeasureTheory.Submartingale}
Let $\mathcal{F}$ be a filtration on a measurable space $\Omega$ with measure $P$ indexed by $T$.
A family of functions $M : T \to \Omega \to E$ is a submartingale with respect to a filtration $\mathcal{F}$ if $M$ is adapted with respect to $\mathcal{F}$ and for all $i \le j$, $P[M_j \mid \mathcal{F}_i] \ge M_i$ almost surely.
\end{definition}

\begin{lemma}\label{lem:condExp_sub_nonneg}
  \uses{def:Submartingale}
  \mathlibok
  \lean{MeasureTheory.Submartingale.condExp_sub_nonneg}
Let $X$ be a real-valued submartingale with respect to a filtration $\mathcal{F}$. Then for all $i \le j$, we have $0 \le P[M_j - M_i \mid \mathcal{F}_i]$ almost surely.
\end{lemma}

\begin{proof}\mathlibok

\end{proof}


\begin{lemma}\label{lem:Submartingale.integrable_stoppedValue}
  \uses{def:Submartingale, def:stoppedProcess}
  \mathlibok
  \lean{MeasureTheory.Submartingale.integrable_stoppedValue}
Let $X$ be a submartingale. Then for all bounded stopping times $\tau$, the stopped value $X_\tau$ is integrable.
\end{lemma}

\begin{proof}\leanok

\end{proof}


\begin{lemma}\label{lem:Martingale.congr}
  \uses{def:Martingale, def:adapted}
  \leanok
  \lean{MeasureTheory.Martingale.congr}
If $X$ is a martingale and $Y$ is an adapted modification of $X$, then $Y$ is a martingale.
\end{lemma}

\begin{proof}
Let $i \le j$ in $T$. We want to show that $P[Y_j \mid \mathcal{F}_i] = Y_i$ almost surely.
It suffices to show that $\int_A Y_j \: dP = \int_A Y_i \: dP$ for all $A \in \mathcal{F}_i$.
Let then $A \in \mathcal{F}_i$.
\begin{align*}
  \int_A Y_j \: dP
  &= \int_A X_j \: dP
  = \int_A X_i \: dP
  = \int_A Y_i \: dP
  \: .
\end{align*}
\end{proof}


\begin{lemma}\label{lem:Submartingale.congr}
  \uses{def:Submartingale, def:adapted}
  \leanok
  \lean{MeasureTheory.Submartingale.congr}
If $X$ is a submartingale and $Y$ is an adapted modification of $X$, then $Y$ is a submartingale.
\end{lemma}

\begin{proof}
Let $i \le j$ in $T$. We want to show that $P[Y_j \mid \mathcal{F}_i] \ge Y_i$ almost surely.
It suffices to show that $\int_A Y_j \: dP \ge \int_A Y_i \: dP$ for all $A \in \mathcal{F}_i$.
Let then $A \in \mathcal{F}_i$.
\begin{align*}
  \int_A Y_j \: dP
  &= \int_A X_j \: dP
  \ge \int_A X_i \: dP
  = \int_A Y_i \: dP
  \: .
\end{align*}
\end{proof}


\begin{definition}[Stopping time]\label{def:IsStoppingTime}
  \uses{def:filtration}
  \mathlibok
  \lean{MeasureTheory.IsStoppingTime}
A stopping time with respect to some filtration $\mathcal{F}$ indexed by $T$ is a function $\tau : \Omega \to T \cup \{\infty\}$ such that for all $i$, the preimage of $\{j \mid j \le i\}$ along $\tau$ is measurable with respect to $\mathcal{F}_i$.
\end{definition}


\begin{definition}[$\sigma$-algebra generated by a stopping time]\label{def:StoppingTimeGen}
  \uses{def:IsStoppingTime, def:filtration}
  \mathlibok
  \lean{MeasureTheory.IsStoppingTime.measurableSpace}
  Given a stopping time $\tau$ on a time index $T$, define
  $\mathcal{F}_\tau = \bigcup_{t \in T} \{A \in \mathcal{F} \mid A \cap \{\tau \le t\} \in \mathcal{F}_t\}.$
\end{definition}


\begin{lemma}\label{lem:StoppingTimeGenMono}
  \uses{def:StoppingTimeGen}
  \mathlibok
  \lean{MeasureTheory.IsStoppingTime.measurableSpace_mono}
  Let $\tau, \sigma$ be stopping times such that $\tau \le \sigma$.
  Then, $\mathcal{F}_\tau \subseteq \mathcal{F}_\sigma$.
\end{lemma}

\begin{proof}\leanok

\end{proof}


\begin{definition}[Stopped process]\label{def:stoppedProcess}
  \mathlibok
  \lean{MeasureTheory.stoppedProcess}
Let $X : T \to \Omega \to E$ be a stochastic process and let $\tau : \Omega \to T$.
The stopped process with respect to $\tau$ is defined by
\begin{align*}
  (X^{\tau})_t = \begin{cases}
    X_t & \text{if } t \le \tau \\
    X_{\tau} & \text{otherwise}
  \end{cases}
\end{align*}
\end{definition}


\begin{definition}[Hitting time]\label{def:hittingAfter}
  \mathlibok
  \lean{MeasureTheory.hittingAfter}
For $X : T \to \Omega \to E$ a stochastic process, $B$ a subset of $E$ and $t_0 \in T$, the hitting time of $X$ in $B$ after $t_0$ is the random variable $\Omega \to T\cup\{\infty\}$ defined by
\begin{align*}
  \tau_{B, t_0}(\omega) = \inf\{t \in T \mid t \ge t_0, \: X_t(\omega) \in B\} \: ,
\end{align*}
in which the infimum is infinite if the set is empty.
\end{definition}


We now give the definition of a filtered probability space satisfying the usual conditions.


\begin{definition}\label{def:leftRightLimitFiltration}
  \uses{def:filtration}
  \lean{MeasureTheory.Filtration.rightCont}
Assume that $T$ is a partial order. For $\mathcal{F}$ a filtration indexed by $T$ and $t \in T$, we define the \emph{left continuation} as
$$\mathcal{F}_{t-} =
  \begin{cases}
    \mathcal{F}_t, & \text{if $t$ is isolated on the left in the order topology}; \\
    \bigsqcup_{s < t} \mathcal{F}_s, & \text{otherwise}.
  \end{cases}
$$
Similarly, we define the \emph{right continuation} as
$$\mathcal{F}_{t+} =
  \begin{cases}
    \mathcal{F}_t, & \text{if $t$ is isolated on the right in the order topology}; \\
    \bigsqcap_{s > t} \mathcal{F}_s, & \text{otherwise}.
  \end{cases}
$$

Note that $\bigsqcap$ and $\bigsqcup$ denote the infimum and supremum in the lattice of sigma-algebras on $\Omega$.
\end{definition}

\begin{lemma}\label{lem:filtrationRightCont}
  \uses{def:filtration, def:leftRightLimitFiltration}
  \leanok
  \lean{MeasureTheory.Filtration.rightCont}
The right continuation of $\mathcal{F}$ is a filtration.
\end{lemma}

\begin{proof}\leanok
We endow $T$ with the order topology. Let us first prove that the right continuation is nondecreasing. Let $s, t \in T$ such that $s \le t$.
\begin{itemize}
  \item Suppose $s$ is isolated on the right. Then $\mathcal{F}_{s+} = \mathcal{F}_s$.
  \begin{itemize}
    \item If $t$ is isolated on the right, then $\mathcal{F}_{t+} = \mathcal{F}_t$. Because $\mathcal{F}$ is a filtration, we have $\mathcal{F}_s \subseteq \mathcal{F}_t$, and thus $\mathcal{F}_{s+} \subseteq \mathcal{F}_{t+}$.
    \item If $t$ is not isolated on the right, then $\mathcal{F}_{t+} = \bigsqcap_{u > t} \mathcal{F}_{u}$. Let $u > t$. Then $s \le u$, thus $\mathcal{F}_s \subseteq \mathcal{F}_u$. This proves that $\mathcal{F}_s \subseteq \bigsqcap_{u > t}, \mathcal{F}_u$, and thus $\mathcal{F}_{s+} \subseteq \mathcal{F}_{t+}$.
  \end{itemize}
  \item Suppose now that $s$ is not isolated on the right, so that $\mathcal{F}_{s+} = \bigsqcup_{u > s} \mathcal{F}_u$.
  \begin{itemize}
    \item If $t$ is isolated on the right, then $\mathcal{F}_{t+} = \mathcal{F}_t$. As $s$ is not isolated on the right, we deduce that $s \ne t$, and thus $s < t$ because $T$ is a partial order. Therefore $\bigsqcap_{u > s} \mathcal{F}_u \subseteq \mathcal{F}_t$, and thus $\mathcal{F}_{s+} \subseteq \mathcal{F}_{t+}$.
    \item If $t$ is not isolated on the right, then $\mathcal{F}_{t+} = \bigsqcap_{u > t} \mathcal{F}_u$. For any $u > t$, we have $u > s$ and thus $\bigsqcap_{v > s} \mathcal{F}_v \subseteq \mathcal{F}_u$, proving that $\bigsqcap_{u > s} \mathcal{F}_u \subseteq \bigsqcap_{u > t} \mathcal{F}_u$, and thus $\mathcal{F}_{s+} \subseteq \mathcal{F}_{t+}$.
  \end{itemize}
\end{itemize}

Turn now to the proof that for all $t \in T$, $\mathcal{F}_{t+} \subseteq \mathcal{A}$. Let $t \in T$. If $t$ is isolated on the right, then $\mathcal{F}_{t+} = \mathcal{F}_t \subseteq \mathcal{A}$ by definition of a filtration. Otherwise there exists $u > t$, and $\mathcal{F}_{t+} \subseteq \mathcal{F}_u \subseteq \mathcal{A}$, so we are done.
\end{proof}

\begin{lemma}\label{lem:rightContDef}
  \uses{def:leftRightLimitFiltration}
  \leanok
  \lean{MeasureTheory.Filtration.rightCont_def}
Suppose $T$ is a topological space with the topology being the order topology. For $t \in T$, the right continuation of $\mathcal{F}$ at $t$ is given by
$$\mathcal{F}_{t+} =
  \begin{cases}
    \mathcal{F}_t, & \text{if $t$ is isolated on the right}; \\
    \bigsqcap_{s > t} \mathcal{F}_s, & \text{otherwise}.
  \end{cases}
$$
\end{lemma}

\begin{proof}
  \uses{def:leftRightLimitFiltration}
This follows from Definition~\ref{def:leftRightLimitFiltration} because the topology on $T$ agrees with the one used to define the right continuation.
\end{proof}

\begin{lemma}\label{lem:rightContIsolated}
  \uses{def:leftRightLimitFiltration}
  \leanok
  \lean{MeasureTheory.Filtration.rightCont_eq_of_nhdsGT_eq_bot}
Suppose $T$ is a topological space with the topology being the order topology. Assume that $t \in T$ is isolated on the right, meaning that there exists a neighbourhood $\mathcal{V}$ of $t$ such that $\mathcal{V} \cap \{u \mid u > t\} = \emptyset$. Then $\mathcal{F}_{t+} = \mathcal{F}_t$.
\end{lemma}

\begin{proof}
  \uses{lem:rightContDef}
This is a direct consequence of Lemma~\ref{lem:rightContDef}.
\end{proof}

\begin{lemma}\label{lem:rightContSuccOrder}
  \uses{def:leftRightLimitFiltration}
  \leanok
  \lean{MeasureTheory.Filtration.rightCont_eq_self}
Assume that $T$ is a linear order with successor. This means that for any $t$, there is an element $succ(t) \ge t$ such that for any $u > t$, $u \ge succ(t)$, and if $succ(t) \le t$, then $t$ is maximal. Then the right continuation of $\mathcal{F}$ is equal to $\mathcal{F}$.
\end{lemma}

\begin{proof}
  \uses{lem:rightContIsolated}
Endow $T$ with the order topology. In a linear order with successor equipped with the order topology, every point is isolated on the right, so we can conclude by Lemma~\ref{lem:rightContIsolated}.
\end{proof}

\begin{lemma}\label{lem:rightContIsMax}
  \uses{def:leftRightLimitFiltration}
  \leanok
  \lean{MeasureTheory.Filtration.rightCont_eq_of_isMax}
If $t \in T$ is maximal, $\mathcal{F}_{t+} = \mathcal{F}_t$.
\end{lemma}

\begin{proof}
  \uses{lem:rightContIsolated}
Endow $T$ with the order topology. As $t$ is maximal, it is isolated on the right for this topology, so we can conclude by Lemma~\ref{lem:rightContIsolated}.
\end{proof}

\begin{lemma}\label{lem:rightContExistsGt}
  \uses{def:leftRightLimitFiltration}
  \leanok
  \lean{MeasureTheory.Filtration.rightCont_eq_of_exists_gt}
If $T$ is a linear order and there exists $u > t$ such that $(t, u) = \emptyset$, then $\mathcal{F}_{t+} = \mathcal{F}_t$.
\end{lemma}

\begin{proof}
  \uses{lem:rightContIsolated}
Endow $T$ with the order topology. The hypothesis implies that $t$ is isolated on the right in this topology, so we can conclude by Lemma~\ref{lem:rightContIsolated}.
\end{proof}

\begin{lemma}\label{lem:rightContNeBot}
  \uses{def:leftRightLimitFiltration}
  \leanok
  \lean{MeasureTheory.Filtration.rightCont_eq_of_neBot_nhdsGT}
Suppose $T$ is a topological space with the topology being the order topology. Assume that $t \in T$ is not isolated on the right, meaning that for all neighbourhood $\mathcal{V}$ of $t$, $\mathcal{V} \cap \{u | u > t\} \ne \emptyset$. Then $\mathcal{F}_{t+} = \bigsqcap_{u > t} \mathcal{F}_u$.
\end{lemma}

\begin{proof}
  \uses{lem:rightContDef}
This is a direct consequence of Lemma~\ref{lem:rightContDef}.
\end{proof}

\begin{lemma}\label{lem:rightContNotIsMax}
  \uses{def:leftRightLimitFiltration}
  \leanok
  \lean{MeasureTheory.Filtration.rightCont_eq_of_not_isMax}
Assume that $T$ is a densely ordered linear order, meaning that for all $s < t$, there exists $u$ such that $s < u < t$. If $t$ is not maximal, then $\mathcal{F}_{t+} = \bigsqcap_{u > t} \mathcal{F}_u$.
\end{lemma}

\begin{proof}
  \uses{lem:rightContNeBot}
Endow $T$ with the order topology. In a densely ordered linear order, a point which is not maximal is not isolated on the right, so we can conclude by Lemma~\ref{lem:rightContNeBot}.
\end{proof}

\begin{lemma}\label{lem:rightContEq}
  \uses{def:leftRightLimitFiltration}
  \leanok
  \lean{MeasureTheory.Filtration.rightCont_eq}
If $T$ is a densely ordered linear order with no maximal element, then forall $t \in T$ we have $\mathcal{F}_{t+} = \bigsqcap_{u > t} \mathcal{F}_u$.
\end{lemma}

\begin{proof}
  \uses{lem:rightContNotIsMax}
For all $t$, $t$ is not maximal, so we can conlude by Lemma~\ref{lem:rightContNotIsMax}.
\end{proof}

\begin{lemma}\label{lem:leRightCont}
  \uses{def:leftRightLimitFiltration}
  \leanok
  \lean{MeasureTheory.Filtration.le_rightCont}
The filtration $\mathcal{F}$ is contained in its right continuation.
\end{lemma}

\begin{proof}
  \uses{lem:rightContDef}
Endow $T$ with the order topology, and consider $t \in T$. Using Lemma~\ref{lem:rightContDef}, we split into two cases. If $t$ is isolated on the right, then $\mathcal{F}_{t+} = \mathcal{F}_t \supseteq \mathcal{F}_t$ and we are done. Otherwise, for all $u > t$, $\mathcal{F}_t \subseteq \mathcal{F}_u$, therefore $\mathcal{F}_t \subseteq \bigsqcap_{u > t} \mathcal{F}_u$, and we are done.
\end{proof}

\begin{lemma}\label{lem:rightContSelf}
  \uses{def:leftRightLimitFiltration}
  \leanok
  \lean{MeasureTheory.Filtration.rightCont_self}
The right continuation of the right continuation of $\mathcal{F}$ is equal to the right continuation of $\mathcal{F}$.
\end{lemma}

\begin{proof}
  \uses{lem:leRightCont, lem:rightContDef}
Let $t \in T$. From Lemma~\ref{lem:leRightCont}, we already now that $\mathcal{F}_{t+} \subseteq \mathcal{F}_{t++}$. Endow $T$ with the order topology and split according to Lemma~\ref{lem:rightContDef}. If $t$ is isolated on the right, then $\mathcal{F}_{t++} = \mathcal{F}_{t+}$ and we are done. Otherwise consider $u > t$. Then there exists $v \in T$ such that $t < v < u$. If $v$ is not isolated on the right then
$$\mathcal{F}_{t++} = \bigsqcap_{s > t} \mathcal{F}_{s+} \subseteq \mathcal{F}_{v+} = \bigsqcap_{s > v} \mathcal{F}_s \subseteq \mathcal{F}_u,$$
and otherwise
$$\mathcal{F}_{t++} = \bigsqcap_{s > t} \mathcal{F}_{s+} \subseteq \mathcal{F}_{v+} = \mathcal{F}_v \subseteq \mathcal{F}_u,$$
thus $\mathcal{F}_{t++} \subseteq \bigsqcap_{s > t} \mathcal{F}_s = \mathcal{F}_{t+}$, which concludes the proof.
\end{proof}

\begin{definition}[Right-continuous filtration]\label{def:rightContinuous}
  \uses{def:leftRightLimitFiltration}
  \leanok
  \lean{MeasureTheory.Filtration.IsRightContinuous}
We say that the filtration is \emph{right-continuous} if for all $t \in T$, $\mathcal{F}_{t+} \subseteq \mathcal{F}_t$.
\end{definition}

\begin{lemma}\label{lem:eqRightCont}
  \uses{def:leftRightLimitFiltration, def:rightContinuous}
  \leanok
  \lean{MeasureTheory.Filtration.IsRightContinuous.eq}
If $\mathcal{F}$ is right-continuous, then for all $t \in T$, $\mathcal{F}_t = \mathcal{F}_{t+}$.
\end{lemma}

\begin{proof}
  \uses{def:rightContinuous, lem:leRightCont}
This is a direct consequence of Definition~\ref{def:rightContinuous} and Lemma~\ref{lem:leRightCont}.
\end{proof}

\begin{lemma}\label{lem:rightContinuousRightCont}
  \uses{def:leftRightLimitFiltration, def:rightContinuous}
  \leanok
  \lean{MeasureTheory.Filtration.isRightContinuous_rightCont}
The right continuation of $\mathcal{F}$ is right-continuous.
\end{lemma}

\begin{proof}
  \uses{lem:rightContSelf}
This follows immediately from Lemma~\ref{lem:rightContSelf}.
\end{proof}

\begin{lemma}\label{lem:rightContinuousMeasurableSet}
  \uses{def:leftRightLimitFiltration, def:rightContinuous}
  \leanok
  \lean{MeasureTheory.Filtration.IsRightContinuous.measurableSet}
If $\mathcal{F}$ is right-continuous, then for all $t \in T$, any set $A \subseteq \Omega$ which is $\mathcal{F}_t$-measurable is also $\mathcal{F}_{t+}$-measurable.
\end{lemma}

\begin{proof}
  \uses{def:rightContinuous}
This is a direct consequence of Definition~\ref{def:rightContinuous}.
\end{proof}

\begin{definition}[Usual conditions]\label{def:hasUsualConditions}
  \uses{def:rightContinuous}
  \leanok
  \lean{MeasureTheory.Filtration.HasUsualConditions}
We say that a filtered probability space $(\Omega, \mathcal{F}, P)$ satisfies the usual conditions if the filtration is right-continuous and if $\mathcal{F}_0$ contains all the $P$-null sets.
\end{definition}


\begin{definition}[Predictable $\sigma$-algebra]\label{def:predictableMeasurableSpace}
  \uses{def:filtration}
  \mathlibok
  \lean{MeasureTheory.Filtration.predictable}
Let $\mathcal{F}$ be a filtration on a measurable space indexed $\Omega$ by a linearly ordered set $T$.
Let $S = \{\{\bot\} \times A \mid A \in \mathcal{F}_\bot\}$ if $T$ has a bottom element and $S = \emptyset$ otherwise.
The predictable sigma-algebra on $T \times \Omega$ is the sigma-algebra generated by the set of sets $\{(t, \infty] \times A \mid t \in T, \: A \in \mathcal{F}_t\} \cup S$.
\end{definition}


\begin{definition}[Predictable process]\label{def:predictable}
  \uses{def:predictableMeasurableSpace}
  \mathlibok
  \lean{MeasureTheory.IsPredictable}
A process $X : T \to \Omega \to E$ is said to be predictable with respect to a filtration $\mathcal{F}$ if it is measurable with respect to the predictable sigma-algebra on $T \times \Omega$.
\end{definition}


\begin{lemma}\label{lem:Predictable.progressive}
  \uses{def:predictable, def:ProgMeasurable}
  \mathlibok
  \lean{MeasureTheory.IsPredictable.progMeasurable}
A predictable process is progressively measurable.
\end{lemma}

\begin{proof}\leanok
Let $X : T \times \Omega \to E$ be a predictable process, we will show that it is progressively measurable. Namely, fixing $t \in T$, denoting
$$\iota_t : [0, t] \to T : s \mapsto s$$
we need to show that $\iota_t \circ X : [0, t] \times \Omega \to E$ is measurable with respect to $\mathcal{B}([0, t]) \otimes \mathcal{F}_t$.

Denoting $\Sigma_{\mathcal{F}}$ for the predictable $\sigma$-algebra generated by $\mathcal{F}$, as $u$ is predictable, we have that $X^{-1}(\mathcal{B}(E)) \le \Sigma_{\mathcal{F}}$. Thus, to show that $\iota_t \circ X$ is $\mathcal{B}([0, t]) \otimes \mathcal{F}_t$-measurable, it suffices to show that $\iota_t^{-1}(\Sigma_{\mathcal{F}}) \le \mathcal{B}([0, t]) \otimes \mathcal{F}_t$. In particular, as
$$\Sigma_{\mathcal{F}} = \sigma(\{(s, \infty) \times A \mid A \in \mathcal{F}_s\} \cup \{\{\perp\} \times A \mid A \in \mathcal{F}_\perp\})$$
is suffices to show that sets of the form $\iota_t^{-1}((s, \infty) \times A)$ for some $s \in T, A \in \mathcal{F}_s$ and $\iota_t^{-1}(\{\bot\} \times A)$ for some $A \in \mathcal{F}_\bot$ are $\mathcal{B}([0, t]) \otimes \mathcal{F}_t$-measurable.

Indeed, if $A \in \mathcal{F}_\bot$
$$\iota_t^{-1}(\{\bot\} \times A) = \{\bot\} \times A$$
while for any $s \in T$ and $A \in \mathcal{F}_s$,
$$\iota_t^{-1}((s, \infty) \times A) = \begin{cases}
    \varnothing, & t < s\\
    (s, t] \times A, & s \le t.
\end{cases}$$
By the monotonicity of the filtration $\mathcal{F}$, all of these cases are $\mathcal{B}([0, t]) \otimes \mathcal{F}_t$-measurable allowing us to conclude.
\end{proof}


\begin{lemma}\label{lem:predictable_Ioc_prod}
  \uses{def:predictableMeasurableSpace}
  \mathlibok
  \lean{MeasureTheory.measurableSet_predictable_Ioc_prod}
Sets of the form $(s, t] \times A$ for any $A \in \mathcal{F}_s$ is measurable with respect to the predictable $\sigma$-algebra.
\end{lemma}

\begin{proof}\leanok
For $t \le s$, the set in question is empty and thusly, trivially measurable. On the other hand, for $s < t$, measurability follows as
$(s, t] \times A = (s, \infty) \times A \setminus (t, \infty) \times A$.
\end{proof}


\begin{lemma}\label{lem:predictable_nat_iff}
  \uses{def:predictable}
  \mathlibok
  \lean{MeasureTheory.isPredictable_iff_measurable_add_one}
Let $X : \mathbb{N} \to \Omega \to E$ be a stochastic process and let $\mathcal{F}$ be a filtration indexed by $\mathbb{N}$.
Then $X$ is predictable if and only if $X_0$ is $\mathcal{F}_0$-measurable and for all $n \in \mathbb{N}$, $X_{n+1}$ is $\mathcal{F}_n$-measurable.
\end{lemma}


\begin{proof}\leanok
  \uses{lem:predictable_Ioc_prod}
Suppose first that $X$ is predictable. Straightaway, $X_0$ is $\mathcal{F}_0$-measurable as predictable implies progressively measurable which in turn implies adapted.

Fixing $n$, we observe that for any $S \in \mathcal{B}(E)$,
$$X_{n + 1}^{-1}(S) = \{\omega \mid (n + 1, \omega) \in X^{-1}(S)\} = \pi^{-1}(\iota^{-1}(X^{-1}(S)))$$
where
$$\pi : \Omega \to \{n + 1\} \times \Omega : \omega \mapsto (n + 1, \omega)$$
and
$$\iota : \{n + 1\} \times \Omega \to T \times \Omega : (n + 1, \omega) \mapsto (n + 1, \omega).$$
As $X^{-1}(S) \in \Sigma_{\mathcal{F}}$ -- the predictable $\sigma$-algebra, it suffices to show that $\pi^{-1}(\iota^{-1}(\Sigma_{\mathcal{F}})) \in \mathcal{F}_n$. To this end, we again only need to show these for the generating sets of $\Sigma_{\mathcal{F}}$:
\begin{itemize}
    \item For $A \in \mathcal{F}_0$, measurability is clear as $\iota^{-1}(\{0\} \times A) = \varnothing$.
    \item Similarly, for $m > n$ and $A \in \mathcal{F}_m$, $\iota^{-1}((m, \infty) \times A) = \varnothing$.
    \item For $m \le n$ and $A \in \mathcal{F}_m \le \mathcal{F}_n$ we have that $\pi^{-1}(\iota^{-1}((m, \infty) \times A)) = A$ which is $\mathcal{F}_n$ measurable by the monotonicity of the filtration.
\end{itemize}

Now, supposing $X_0$ is $\mathcal{F}_0$-measurable and $X_{n + 1}$ is $\mathcal{F}_n$-measurable, we will show that $X$ is predictable. Indeed, fixing $S \in \mathcal{B}(E)$, we have
$$X^{-1}(S) = \bigcup_{n \in \mathbb{N}} \{n\} \times X_n^{-1}(S) = {0} \times X_0^{-1}(S) \cup \bigcup_{n \in \mathbb{N}} \{n + 1\} \times X_{n + 1}^{-1}(S).$$
Thus, as $\{0\} \times X_0^{-1}(S) \in \Sigma_{\mathcal{F}}$ by construction and $\{n + 1\} \times X_{n + 1}^{-1}(S) = (n, n + 1] \times X_{n + 1}^{-1}(S) \in \Sigma_{\mathcal{F}}$ by Lemma~\ref{lem:predictable_Ioc_prod} and the fact that $X_{n + 1}^{-1}(S) \in \mathcal{F}_n$, we have that $X^{-1}(S) \in \Sigma_{\mathcal{F}}$ as required.
\end{proof}


\begin{lemma}[Optional sampling (discrete time)]\label{lem:optionalSampling_discrete}
  \uses{def:Martingale, def:IsStoppingTime}
  \mathlibok
  \lean{MeasureTheory.Martingale.stoppedValue_min_ae_eq_condExp}
  Let $X$ be a discrete time martingale with respect to the filtration $\mathcal{F}$ and let
  $\tau, \sigma$ be stopping times. Then, if $\tau$ is bounded, we have that
  $X_{\tau \wedge \sigma} = P[X_{\tau} \mid \mathcal{F}_{\sigma}]$.
\end{lemma}

\begin{proof}\leanok

\end{proof}


\begin{lemma}\label{lem:conExpUI}
  \leanok
  \lean{MeasureTheory.UniformIntegrable.condExp'}
  If $(X_i)_{i \in \iota}$ is a family of (probabilistically) uniformly integrable (pUI) functions and $(\mathcal{F}_j)_{j \in \kappa}$ is a family of $\sigma$-algebras,
  then the family $(P[X_i \mid \mathcal{F}_j])_{i \in \iota, j \in \kappa}$ is pUI.
\end{lemma}

\begin{proof}\leanok
  Since $(X_i)_{i \in \iota}$ is pUI, it is uniformly bounded in $L^1$, thus so is $(P[X_i \mid \mathcal{F}_j])_{i \in \iota, j \in \kappa}$. Moreover, for any \(\epsilon > 0\),
  there exists some \(\delta > 0\) such that
  for any measurable set \(A\) with \(P(A) < \delta\), we have that \(\sup_{i \in \iota} P[|X_i| \mathbb{I}_A] < \epsilon\).

  On the other hand, by Markov's inequality, for any $\lambda > 0$, $i \in \iota$ and $j \in \kappa$ we have that
  \[P(|P[X_i \mid \mathcal{F}_j]| \ge \lambda) \le \lambda^{-1}P[|P[X_i \mid \mathcal{F}_j]|] \le \lambda^{-1}P[|X_i|].\]
  Now set \(\lambda := \delta^{-1} \sup_{i \in \iota} P[|X_i|] + 1\). Then for any $i \in \iota$ and $j \in \kappa$ we have that
  \[P(|P[X_i \mid \mathcal{F}_j]| \ge \lambda) \le \frac{P[|X_i|]}{\delta^{-1} \sup_{k \in \iota} P[|X_k|] + 1} < \delta,\]
  and so,
  \begin{align*}
    P[|P[X_i \mid \mathcal{F}_j]| \mathbb{I}_{|P[X_i \mid \mathcal{F}_i]| \ge \lambda}]
    & = P[|P[X_i \mid \mathcal{F}_j] \mathbb{I}_{|P[X_i \mid \mathcal{F}_i]| \ge \lambda}|] \\
    & = P[|P[X_i \mathbb{I}_{|P[X_i \mid \mathcal{F}_i]| \ge \lambda} \mid \mathcal{F}_j]|] \\
    & \le P[P[|X_i|\mathbb{I}_{|P[X_i \mid \mathcal{F}_j]| \ge \lambda} \mid \mathcal{F}_j]] \\
    & = P[|X_i|\mathbb{I}_{|P[X_i \mid \mathcal{F}_j]| \ge \lambda}] < \epsilon,
  \end{align*}
  showing that \((P[X_i \mid \mathcal{F}_j])_{i \in \iota, j \in \kappa}\) is pUI.
\end{proof}


\begin{lemma}\label{lem:uniformIntegrable_stoppedValue_martingale}
  \uses{def:Martingale, def:IsStoppingTime}
  \leanok
  \lean{MeasureTheory.Martingale.uniformIntegrable_stoppedValue}
Let $X$ be a martingale on a discrete index set and let $(\tau_k)_{k \in \mathbb{N}}$ be a sequence of stopping times that are uniformly bounded by $n$.
Then, the family of stopped values $\{X_{\tau_k}\}_{k \in \mathbb{N}}$ is pUI.
\end{lemma}

\begin{proof}
  \uses{lem:optionalSampling_discrete, lem:conExpUI}
  \leanok
By optional sampling (Lemma~\ref{lem:optionalSampling_discrete}), we have that for each $k$, $X_{\tau_k} = P[X_n \mid \mathcal{F}_{\tau_k}]$.
Thus, the result follows by Lemma~\ref{lem:conExpUI} as $\{X_n\}$ is pUI.
\end{proof}

\begin{lemma}\label{lem:uniformIntegrable_stoppedValue_martingale_of_countable_range}
  \uses{def:Martingale, def:IsStoppingTime}
  \leanok
  \lean{MeasureTheory.Martingale.uniformIntegrable_stoppedValue_of_countable_range}
Let $X$ be a martingale and let $(\tau_k)_{k \in \mathbb{N}}$ be a sequence of stopping times that are uniformly bounded by $n$.
Then, the family of stopped values $\{X_{\tau_k}\}_{k \in \mathbb{N}}$ is pUI if for each $k$, $\tau_k$ takes value in a countable set.
\end{lemma}

\begin{proof}
  \uses{lem:optionalSampling_discrete, lem:conExpUI}
  Same proof as in Lemma~\ref{lem:uniformIntegrable_stoppedValue_martingale}.
\end{proof}

\begin{lemma}[Vitali convergence theorem]\label{lem:vitali}
  \mathlibok
  \lean{MeasureTheory.tendstoInMeasure_iff_tendsto_Lp_finite}
  A sequence of functions converges in $L^1$ if and only if it converges in probability and is pUI.
\end{lemma}

\begin{proof}\leanok

\end{proof}

\begin{definition}[Discrete approximation sequence]\label{def:approxSeq}
  \uses{def:IsStoppingTime}
  \leanok
  \lean{MeasureTheory.DiscreteApproxSequence}
  Given a stopping time \(\tau : \Omega \to T \cup \{\infty\}\), a sequence of stopping times \((\tau_n)_{n \in \mathbb{N}}\) is called an
  discrete approximation of \(\tau\) if \(\tau_n(\Omega)\) is countable for each \(n\) and \(\tau_n \downarrow \tau\) a.s. as \(n \to \infty\).
\end{definition}

\begin{lemma}\label{lem:tendsto_stoppedValue_discreteApproxSequence}
  \uses{def:IsStoppingTime, def:approxSeq}
  \leanok
  \lean{MeasureTheory.tendsto_stoppedValue_discreteApproxSequence}
  Given a right continuous process \(X\) and a discrete approximation sequence \((\tau_n)\) of the stopping time \(\tau\), we have that
  \[\lim_{n \to \infty} X_{\tau_n} = X_\tau \text{ a.s.}\]
\end{lemma}

\begin{proof}
  This follows directly as \(X\) is right continuous and \(\tau_n \downarrow \tau\) a.s.
\end{proof}

\begin{lemma}\label{lem:discreteApproxSequence_of}
  \uses{def:IsStoppingTime, def:approxSeq}
  \leanok
  \lean{MeasureTheory.discreteApproxSequence_of}
  Let \(\tau\) be a stopping time bounded by \(t \in T\) and \((\tau_n)\) be a discrete approximation sequence of \(\tau\).
  Then, the sequence of stopping times \(\tau_n \wedge t\) is also a discrete approximation sequence of \(\tau\).
\end{lemma}

\begin{proof}
  \leanok
\end{proof}

\begin{lemma}\label{lem:uniformIntegrable_stoppedValue_discreteApproxSequence}
  \leanok
  \lean{uniformIntegrable_stoppedValue_discreteApproxSequence}
  Let \(\tau\) be a stopping time bounded by \(t \in T\) and \((\tau_n)\) be a discrete approximation sequence of \(\tau\).
  Then, for any martingale \(X\), the sequence of stopped values \((X_{\tau_n \wedge t})\) is pUI.
\end{lemma}

\begin{proof}
  \uses{lem:discreteApproxSequence_of, lem:uniformIntegrable_stoppedValue_martingale_of_countable_range}
  \leanok
  Follows directly by Lemma~\ref{lem:uniformIntegrable_stoppedValue_martingale_of_countable_range} and Lemma~\ref{lem:discreteApproxSequence_of}.
\end{proof}

\begin{lemma}\label{lem:stoppingTime_approximation}
  \uses{def:IsStoppingTime}
  Let $\tau$ be a stopping time on $\overline{\mathbb{R}_+}$ with respect to the filtration $\mathcal{F}$.
  Then defining $\tau_n =  2^{-n} \lceil 2^n \tau \rceil$,
  we have that for each $n$, $\tau_n$ is a stopping time and $\tau_n \downarrow \tau$ as $n \to \infty$.
\end{lemma}

\begin{proof}
  Clearly $\tau_n \downarrow \tau$ as $n \to \infty$ and so it remains to show that each $\tau_n$ is a stopping time.
  Indeed,
  \[\{\tau_n \le t\} = \{\tau \le 2^{-n} \lfloor 2^n t\rfloor\}
    \in \mathcal{F}_{2^{-n} \lfloor 2^n t\rfloor} \subseteq \mathcal{F}_t\]
  where the last inclusion follows as \(2^{-n} \lfloor 2^n t\rfloor \le t\).
\end{proof}

\begin{lemma}[Optional sampling (continuous time)]\label{lem:optionalSampling}
  \uses{def:Martingale, def:IsStoppingTime, def:rightContinuous}
  Let $X$ be a right-continuous martingale with respect to the filtration $\mathcal{F}$. Then for
  any bounded stopping times $\sigma, \tau$, we have that
  $X_{\sigma \wedge \tau} = P[X_{\tau} \mid \mathcal{F}_{\sigma}]$ almost surely.
\end{lemma}

\begin{proof}
  \uses{lem:stoppingTime_approximation, lem:vitali, lem:StoppingTimeGenMono, lem:uniformIntegrable_stoppedValue_martingale}
  Fixing $A \in \mathcal{F}_{\sigma}$, we need to show that $P[X_{\tau} \mathbb{I}_A] = P[X_{\sigma \wedge \tau} \mathbb{I}_A]$.

  Denoting $\tau_n = 2^{-n} \lceil 2^n \tau \rceil$ and $\sigma_n = 2^{-n} \lceil 2^n \sigma \rceil$ as in
  Lemma~\ref{lem:stoppingTime_approximation}, $(\tau_n), (\sigma_n)$ are sequences of stopping times
  decreasing to $\tau$ and $\sigma$ respectively. Now, as $\tau_n, \sigma_n$ take values in a countable set,
  we have by the discrete time optional sampling theorem (Lemma~\ref{lem:optionalSampling_discrete}) that
  $$X_{\sigma_n \wedge \tau_n} = P[X_{\tau_n} \mid \mathcal{F}_{\sigma_n}]$$
  and so, as \(\mathcal{F}_{\sigma} \subseteq \mathcal{F}_{\sigma_n}\) by Lemma~\ref{lem:StoppingTimeGenMono},
  we have that $P[X_{\sigma_n \wedge \tau_n} \mathbb{I}_A] = P[X_{\tau_n} \mathbb{I}_A]$.
  On the other hand, by Lemma~\ref{lem:uniformIntegrable_stoppedValue_martingale}, the families $\{X_{\tau_n}\}$ and $\{X_{\sigma_n \wedge \tau_n}\}$ are pUI.
  Thus, as $X$ is right-continuous, $(X_{\sigma_n \wedge \tau_n}, X_{\tau_n}) \to (X_{\sigma \wedge \tau}, X_{\tau})$ a.s.
  We have $P[X_{\tau} \mathbb{I}_A] = P[X_{\sigma \wedge \tau} \mathbb{I}_A]$
  by the Vitali convergence theorem (Lemma~\ref{lem:vitali}) as desired.
\end{proof}
