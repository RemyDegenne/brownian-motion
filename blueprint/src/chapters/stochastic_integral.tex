\chapter{Stochastic integral}

The lecture notes at \href{https://dec41.user.srcf.net/h/III_L/stochastic_calculus_and_applications/}{this link} as well as chapter 18 of \cite{kallenberg2021} are good references for this chapter.
Some of the proofs are taken from \cite{pascucci2024}.

\section{Total variation and Lebesgue-Stieltjes integral}

TODO: in Mathlib, we can integrate with respect to the measure given by a right-continuous monotone function (\texttt{StieltjesFunction.measure}). This will be useful to integrate against the quadratic variation of a local martingale.
However, we will also want to integrate with respect to a signed measure given by a càdlàg function with finite variation.
We need to investigate what's already in Mathlib. See \texttt{Mathlib.Topology.EMetricSpace.BoundedVariation}.


\section{Doob's Lp inequality}

In this section, we prove Doob's Lp inequality.

\begin{lemma}\label{lem:convex_of_mg_is_submg}
  Let $X:T\times\Omega\rightarrow E$ a martingale with values in a normed space $E$.
  Let $\phi : E \rightarrow \mathbb{R}$ convex such that
  $\phi(X_t)\in L^1(\Omega)$ for every $t\in T$. Then $\phi(X)$ is a sub-martingale.
\end{lemma}
\begin{proof}
  % See 1.4.12 Pascucci
  By the conditional Jensen's inequality (see \href{https://github.com/leanprover-community/mathlib4/pull/27953}{\#27953})
  $\phi(X_t) = \phi\left( \mathbb{E}[X_T\ |\ \mathcal{F}_t] \right)\leq \mathbb{E}[\phi(X_T)\ |\ \mathcal{F}_t]$.
\end{proof}

\begin{lemma}\label{lem:convex_of_submg_is_submg}
  Let $X:T\times\Omega\rightarrow \mathbb{R}^d$ a sub-martingale.
  Let $\phi:\mathbb{R^d}\rightarrow \mathbb{R}$ convex increasing such that
  $\phi(X_t)\in L^1(\Omega)$ for every $t\in T$. Then $\phi(X)$ is a sub-martingale.
\end{lemma}
\begin{proof}
  By Jensen and the fact that $\phi$ is increasing
  $\phi(X_t) \leq \phi\left( \mathbb{E}[X_T\ |\ \mathcal{F}_t] \right)\leq \mathbb{E}[\phi(X_T)\ |\ \mathcal{F}_t]$.
\end{proof}

\begin{lemma}[Doob Inequality for countable]\label{lem:doob_countable}
  Let $X:I\times\Omega\rightarrow \mathbb{R}$ be a non-negative sub-martingale.  Let $I$ be countable.
  For every $M\in I,\lambda > 0$ and $p>1$ we have
  $$
  P\left( \sup_{i\in I, i\leq M}X_i\geq\lambda \right)\leq \frac{\mathbb{E}[X_M]}{\lambda}.
  $$
\end{lemma}
\begin{proof}
  \uses{lem:convex_of_mg_is_submg}
  % TODO: put a sketch of the proof here
  See 8.1.1 Pascucci.
\end{proof}

\begin{lemma}[Doob Inequality Corollary for countable]\label{lem:doob_countable_cor}
  Let $X:I\times\Omega\rightarrow \mathbb{R}$ be a sub-martingale. Let $I$ be countable.
  For every $M\in I,\lambda > 0$ and $p>1$ we have
  $$
  \mathbb{E}\left[ \sup_{i\in I, i\leq M}X_i^p \right]\leq \left(\frac{p}{p-1}\right)^p\mathbb{E}[X_M^p].
  $$
\end{lemma}
\begin{proof}
  \uses{lem:convex_of_mg_is_submg,lem:doob_countable}
  % TODO: put a sketch of the proof here
  8.1.1 Pascucci.
\end{proof}

\begin{theorem}[Doob Inequality]\label{thm:doob_ineq}
  Let $X:\mathbb{R}\times\Omega\rightarrow \mathbb{R}$ be a right-continuous non-negative sub-martingale.
  For every $T, \lambda>0$ and $p>1$ we have
  $$
  P\left( \sup_{t\in[0,T]}X_t\geq\lambda \right)\leq \frac{\mathbb{E}[X_T]}{\lambda}.
  $$
\end{theorem}
\begin{proof}
  \uses{lem:doob_countable}
  % TODO: put a sketch of the proof here
  8.1.2 Pascucci.
\end{proof}

\begin{corollary}[Doob Inequality for normed spaces]\label{cor:doob_ineq}
  Let $X:\mathbb{R}\times\Omega\rightarrow E$ be a right-continuous martingale with values in a normed space $E$.
  For every $T, \lambda>0$ and $p>1$ we have
  $$
  P\left( \sup_{t\in[0,T]} \lVert X_t \rVert \geq \lambda \right) \leq \frac{\mathbb{E}[\lVert X_T \rVert]}{\lambda}.
  $$
\end{corollary}
\begin{proof}
  \uses{lem:convex_of_mg_is_submg, thm:doob_ineq}
  By Lemma~\ref{lem:convex_of_mg_is_submg}, $\lVert X \rVert$ is a sub-martingale.
  Then apply Theorem~\ref{thm:doob_ineq}.
\end{proof}

\begin{theorem}[Doob's Lp inequality]\label{thm:doob_lp}
  Let $X:\mathbb{R}\times\Omega\rightarrow \mathbb{R}$ be a right-continuous non-negative sub-martingale.
  For every $T, \lambda>0$ and $p>1$ we have
  $$
  \mathbb{E}\left[ \sup_{t\in[0,T]}X_t^p \right]\leq \left(\frac{p}{p-1}\right)^p\mathbb{E}[X_T^p].
  $$
\end{theorem}
\begin{proof}
  \uses{lem:doob_countable_cor}
  % TODO: put a sketch of the proof here
  8.1.2 Pascucci.
\end{proof}

\begin{corollary}[Doob's Lp inequality for normed spaces]\label{cor:doob_lp}
  Let $X : \mathbb{R}\times\Omega\rightarrow E$ be a right-continuous martingale with values in a normed space $E$.
  For every $T, \lambda>0$ and $p>1$ we have
  $$
  \mathbb{E}\left[ \sup_{t\in[0,T]} \lVert X_t \rVert ^p \right]\leq \left(\frac{p}{p-1}\right)^p\mathbb{E}[\lVert X_T \rVert ^p].
  $$
\end{corollary}
\begin{proof}
  \uses{lem:convex_of_mg_is_submg, thm:doob_lp}
  By Lemma~\ref{lem:convex_of_mg_is_submg}, $\lVert X \rVert$ is a sub-martingale.
  Then apply Theorem~\ref{thm:doob_lp}.
\end{proof}

\begin{lemma}[Stopped Martingale]\label{lem:stop_of_mg_is_mg}
  Let $X:\mathbb{R}\times\Omega\rightarrow \mathbb{R}$ be a cadlag martingale and $\tau_0$ a stopping time. Then $(X_{t\wedge\tau_0})_{t\geq 0}$ is a martingale.
\end{lemma}

\begin{lemma}[Doob Inequality for stopping times]\label{lem:doob_ineq_stop}
  Let $X:\mathbb{R}\times\Omega\rightarrow \mathbb{R}$ be a right-continuous non-negative sub-martingale.
  For every $\lambda>0$ and $p>1$ and $\tau$ stopping time a.s. bounded by $T>0$, we have
  $$
  P\left( \sup_{t\in[0,\tau]}X_t\geq\lambda \right)\leq \frac{\mathbb{E}[X_\tau]}{\lambda}.
  $$
\end{lemma}
\begin{proof}
  \uses{thm:doob_ineq, lem:stop_of_mg_is_mg}
  Almost already in mathlib MeasureTheory.Submartingale.stoppedProcess.
\end{proof}

\begin{corollary}[Doob Inequality for stopping times in normed spaces]\label{cor:doob_ineq_stop}
  Let $X:\mathbb{R}\times\Omega\rightarrow E$ be a right-continuous martingale with values in a normed space $E$.
  For every $\lambda>0$ and $p>1$ and $\tau$ stopping time a.s. bounded by $T>0$, we have
  $$
  P\left( \sup_{t\in[0,\tau]}\lVert X_t \rVert \geq\lambda \right)\leq \frac{\mathbb{E}[\lVert X_\tau \rVert]}{\lambda}.
  $$
\end{corollary}
\begin{proof}
  \uses{lem:convex_of_mg_is_submg, lem:doob_ineq_stop}
  By Lemma~\ref{lem:convex_of_mg_is_submg}, $\lVert X \rVert$ is a sub-martingale.
  Then apply Theorem~\ref{lem:doob_ineq_stop}.
\end{proof}

\begin{lemma}[Doob's Lp Inequality for stopping times]\label{lem:doob_ineq_stop_exp_val}
  Let $X:\mathbb{R}\times\Omega\rightarrow \mathbb{R}$ be a right-continuous non-negative sub-martingale.
  For every $\lambda>0$ and $p>1$ and $\tau$ stopping time a.s. bounded by $T>0$, we have
  $$
  \mathbb{E}\left[ \sup_{t\in[0,\tau]}X_t^p \right]\leq \left(\frac{p}{p-1}\right)^p\mathbb{E}[X_\tau^p].
  $$
\end{lemma}
\begin{proof}
  \uses{lem:doob_ineq_stop, lem:stop_of_mg_is_mg}
  8.1.3 Pascucci.
\end{proof}

\begin{corollary}[Doob's Lp Inequality for stopping times in normed spaces]\label{cor:doob_ineq_stop_exp_val}
  Let $X:\mathbb{R}\times\Omega\rightarrow E$ be a right-continuous martingale with values in a normed space $E$.
  For every $\lambda>0$ and $p>1$ and $\tau$ stopping time a.s. bounded by $T>0$, we have
  $$
  \mathbb{E}\left[ \sup_{t\in[0,\tau]}\lVert X_t \rVert^p \right]\leq \left(\frac{p}{p-1}\right)^p\mathbb{E}[\lVert X_\tau \rVert^p].
  $$
\end{corollary}
\begin{proof}
  \uses{lem:convex_of_mg_is_submg, lem:doob_ineq_stop_exp_val}
  By Lemma~\ref{lem:convex_of_mg_is_submg}, $\lVert X \rVert$ is a sub-martingale.
  Then apply Theorem~\ref{lem:doob_ineq_stop_exp_val}.
\end{proof}

\section{Square integrable martingales}

In this section, $E$ denotes a complete normed space.

First, recall the definitions of a martingale, a stopping time and a stopped process, which are already in Mathlib.


\begin{definition}[Martingale]\label{def:Martingale}
  \mathlibok
  \lean{MeasureTheory.Martingale}
Let $\mathcal{F}$ be a filtration on a measurable space $\Omega$ with measure $P$ indexed by $T$.
A family of functions $M : T \to \Omega \to E$ is a martingale with respect to a filtration $\mathcal{F}$ if $M$ is adapted with respect to $\mathcal{F}$ and for all $i \le j$, $P[M_j \mid \mathcal{F}_i] = M_i$ almost surely.
\end{definition}


\begin{definition}[Stopping time]\label{def:IsStoppingTime}
  \mathlibok
  \lean{MeasureTheory.IsStoppingTime}
A stopping time with respect to some filtration $\mathcal{F}$ indexed by $T$ is a function $\tau : \Omega \to T$ such that for all $i$, the preimage of $\{j \mid j \le i\}$ along $\tau$ is measurable with respect to $\mathcal{F}_i$.
\end{definition}


\begin{definition}[Stopped process]\label{def:stoppedProcess}
  \mathlibok
  \lean{MeasureTheory.stoppedProcess}
Let $X : T \to \Omega \to E$ be a stochastic process and let $\tau : \Omega \to T$.
The stopped process with respect to $\tau$ is defined by
\begin{align*}
  (X^{\tau})_t = \begin{cases}
    X_t & \text{if } t \le \tau \\
    X_{\tau} & \text{otherwise}
  \end{cases}
\end{align*}
\end{definition}


\begin{definition}[Square integrable martingales]\label{def:squareIntegrableMartingales}
  \uses{def:Martingale}
Let $\mathcal{M}^2$ be the set of square integrable continuous martingales with respect to a filtration $\mathcal{F}$ indexed by $\mathbb{R}_+$,
\begin{align*}
  \mathcal{M}^2
  = \{ M : \mathbb{R}_+ \to \Omega \to \mathbb{R} \mid M \text{ continuous martingale with } \sup_{t}\mathbb{E}[M_t^2] < \infty \}
  \: .
\end{align*}
\end{definition}


\begin{theorem}\label{thm:hilbertSpace_squareIntegrableMartingales}
  \uses{def:squareIntegrableMartingales}
The space $\mathcal{M}^2$ is a Hilbert space with the inner product defined by
\begin{align*}
  \langle M, N \rangle = \mathbb{E}[M_\infty N_\infty]
  \: .
\end{align*}
\end{theorem}

\begin{proof}
  \uses{thm:doob_lp}

\end{proof}


\section{Local martingales}

TODO: filtrations should be assumed right-continuous and complete whenever needed.

\begin{definition}[Local martingale]\label{def:IsLocalMartingale}
  \uses{def:Martingale, def:IsStoppingTime, def:stoppedProcess}
Let $\mathcal{F} = (\mathcal{F}_t)_{t \in \mathbb{R}_+}$ be a filtration on a measurable space $\Omega$.
A local martingale with respect to $\mathcal{F}$ is a stochastic process $M : \mathbb{R}_+ \to \Omega \to E$ adapted to $\mathcal{F}$ such that there exists a localizing sequence $(\tau_n)_{n \in \mathbb{N}}$ such that the following conditions hold:
\begin{itemize}
  \item $\tau_n$ is a stopping time for every $n \in \mathbb{N}$,
  \item $\tau_n$ is non-decreasing and $\tau_n \to \infty$ as $n \to \infty$ (a.s.),
  \item for all $n \in \mathbb{N}$, the stopped and centered process $M^{\tau_n} - M_0$ is a martingale with respect to $\mathcal{F}$.
\end{itemize}
\end{definition}


\begin{lemma}\label{lem:Martingale.IsLocalMartingale}
  \uses{def:IsLocalMartingale}
Every martingale is a local martingale.
\end{lemma}

\begin{proof}

\end{proof}


\begin{theorem}\label{thm:IsLocalMartingale.eq_zero_of_finiteVariation}
  \uses{def:IsLocalMartingale}
Let $M$ be a continuous local martingale with $M_0 = 0$. If $M$ is also a finite variation process, then $M_t = 0$ for all $t$.
\end{theorem}

\begin{proof}

\end{proof}


\begin{definition}[Quadratic variation]\label{def:quadraticVariation}
  \uses{def:IsLocalMartingale}
For any continuous local martingale $M$, there exists a continuous process $[M]$ with $[M]_0 = 0$ such that $M^2 - [M]$ is a local martingale. That process is a.s. unique and is called the \emph{quadratic variation} of $M$.
\end{definition}


\begin{definition}[Covariation]\label{def:covariation}
  \uses{def:IsLocalMartingale}
For any continuous local martingales $M$ and $N$, there exists a continuous process $[M,N]$ with $[M,N]_0 = 0$ such that $MN - [M,N]$ is a local martingale. That process is a.s. unique and is called the \emph{covariation} of $M$ and $N$.

It can be defined by $[M, N]_t = \frac{1}{4}\left([M+N]_t - [M-N]_t \right)$~.
\end{definition}


\begin{lemma}\label{lem:covariation_eq_inner}
  \uses{def:covariation, def:squareIntegrableMartingales}
Let $M$ and $N$ be continuous square integrable martingales. Then
\begin{align*}
  \mathbb{E}\left[[M,N]_\infty\right] = \langle M - M_0, N - N_0 \rangle_{\mathcal{M}^2}
  \: .
\end{align*}
\end{lemma}

\begin{proof}

\end{proof}


\begin{lemma}\label{lem:quadraticVariation_brownian}
  \uses{def:brownian, def:quadraticVariation}
Let $B$ be a standard Brownian motion. Then the quadratic variation of $B$ is given by $[B]_t = t$~.
\end{lemma}

\begin{proof}

\end{proof}


\begin{definition}[Continuous semi-martingale]\label{def:continuousSemiMartingale}
  \uses{def:IsLocalMartingale}
A continuous semi-martingale is a process that can be decomposed into a local martingale and a finite variation process.
More formally, a process $X : \mathbb{R}_+ \to \Omega \to E$ is a continuous semi-martingale if there exists a continuous local martingale $M$ and a continuous adapted process $A$ with locally finite variation and $A_0 = 0$ such that
\begin{align*}
  X_t = M_t + A_t
\end{align*}
for all $t \ge 0$.
The decomposition is a.s. unique.
\end{definition}


\section{Stochastic integral}


\begin{definition}[Simple process]\label{def:simpleProcess}
Let $0 \le t_0 \le t_1 \le \ldots \le t_n$ in $\mathbb{R}_+$.
Let $(\eta_k)_{0 \le k \le n-1}$ be $\mathcal{F}_{t_k}$-measurable random variables.
Then the simple process for that sequence is the process $V : \mathbb{R}_+ \to \Omega \to E$ defined by
\begin{align*}
  V_t = \sum_{k=0}^{n-1} \eta_k \mathbb{1}_{(t_k, t_{k+1}]}(t)
  \: .
\end{align*}
Let $\mathcal{E}$ be the set of simple processes.
\end{definition}


\begin{definition}[Elementary stochastic integral]\label{def:elemStochIntegral}
  \uses{def:simpleProcess}
Let $V \in \mathcal{E}$ be a simple process and let $X$ be a stochastic process.
The \emph{elementary stochastic integral} process $V \cdot X : \mathbb{R}_+ \to \Omega \to E$ is defined by
\begin{align*}
  (V \cdot X)_t
  &= \sum_{k=0}^{n-1} \eta_k (X^t_{t_{k+1}} - X^t_{t_k})
  \: .
\end{align*}
\end{definition}


\begin{lemma}\label{lem:sq_norm_elemStochIntegral}
  \uses{def:elemStochIntegral}
For $V \in \mathcal{E}$ and $M \in \mathcal{M}^2$, then $V \cdot M \in \mathcal{M}^2$ and
\begin{align*}
  \Vert V \cdot M \Vert_{\mathcal{M}^2}^2
  &= \mathbb{E}\left[ \int_0^{\infty} V_t^2 \: d[M]_t \right]
  \: .
\end{align*}
\end{lemma}

\begin{proof}

\end{proof}


\subsection{Itô isometry}

\begin{definition}\label{def:L2M}
  \uses{def:squareIntegrableMartingales}
Let $M \in \mathcal{M}^2$ be a continuous square integrable martingale. We define
\begin{align*}
  L^2(M) = L^2(\Omega \times \mathbb{R}_+, \mathcal{P}, \mathbb{P} \times d[M])
\end{align*}
in which $\mathcal{P}$ is the predictable $\sigma$-algebra and $d[M]$ is the measure induced by the quadratic variation of $M$.
The norm on that Hilbert space is $\Vert X \Vert^2 = \mathbb{E}\left[ \int_0^{\infty} X_t^2 \: d[M]_t \right]$~.
\end{definition}

TODO the sources don't use the same assumptions: predictable vs progressive (\texttt{MeasureTheory.ProgMeasurable}). Progressive would be more general.


\begin{lemma}\label{lem:dense_simpleProcess}
  \uses{def:L2M, def:simpleProcess}
Let $M \in \mathcal{M}^2$. Then the set of simple processes is dense in $L^2(M)$.
\end{lemma}

\begin{proof}

\end{proof}


\begin{definition}[Itô isometry]\label{def:itoIsometry}
  \uses{lem:dense_simpleProcess, lem:sq_norm_elemStochIntegral, thm:hilbertSpace_squareIntegrableMartingales}
Let $M \in \mathcal{M}^2$. Then the elementary stochastic integral map $\mathcal{E} \to \mathcal{M}^2$ defined by $V \mapsto V \cdot M$ extends to an isometry $L^2(M) \to \mathcal{M}^2$.
\end{definition}


\begin{lemma}\label{lem:inner_itoIsometry}
  \uses{def:itoIsometry}
$\langle X \cdot M, Y \cdot M \rangle_{\mathcal{M}^2} = (XY) \cdot \langle M, N \rangle_{\mathcal{M}^2}$.
\end{lemma}

\begin{proof}

\end{proof}


\subsection{Local martingales}

\begin{definition}[$L^2_{loc}(M)$]\label{def:L2locM}
  \uses{def:L2M}
Let $M$ be a continuous local martingale.
We define $L^2_{loc}(M)$ as the space of predictable processes $X$ such that for all $t \ge 0$, $\mathbb{E}\left[ \int_0^t X_s^2 \: d[M]_s \right] < \infty$.
\end{definition}


\begin{definition}[Stochastic integral for continuous local martingales]\label{def:locStochIntegral}
  \uses{def:L2locM, def:itoIsometry}
Let $M$ be a continuous local martingale and let $X \in L^2_{loc}(M)$.
We define the local stochastic integral $X \cdot M$ as the unique continuous local martingale with $(X \cdot M)_0 = 0$ such that for any continuous local martingale $N$, almost surely,
\begin{align*}
  [X \cdot M, N] = X \cdot [M, N]
  \: .
\end{align*}
\end{definition}


\subsection{Semi-martingales}

\begin{definition}\label{def:stochIntegral}
  \uses{def:continuousSemiMartingale, def:locStochIntegral}
For a continuous semi-martingale $X = M + A$ and $V \in L^2_{semi}(X)$ (to be defined) we define the stochastic integral as
\begin{align*}
  V \cdot X = V \cdot M + V \cdot A
  \: ,
\end{align*}
in which $V \cdot M$ is the local stochastic integral defined in \ref{def:locStochIntegral} and $V \cdot A$ is the Lebesgue-Stieltjes integral with respect to the locally finite variation process $A$.
\end{definition}


For $X = M + A$ and $Y = N + B$, we define the covariation as
\begin{align*}
  [X, Y] = [M, N]
  \: .
\end{align*}

\section{Itô formula}


\begin{theorem}[Integration by parts]\label{thm:integration_by_parts}
  \uses{def:continuousSemiMartingale, def:stochIntegral}
Let $X$ and $Y$ be two continuous semi-martingales. Then we have almost surely
\begin{align*}
  X_t Y_t - X_0 Y_0
  = (X \cdot Y)_t + (Y \cdot X)_t + [X,Y]_t
  \: .
\end{align*}
\end{theorem}

\begin{proof}

\end{proof}


\begin{theorem}[Itô's formula]\label{thm:Ito_formula}
  \uses{def:continuousSemiMartingale}
Let $X^1, \ldots, X^d$ be continuous semi-martingales and let $f : \mathbb{R}^d \to \mathbb{R}$ be a twice continuously differentiable function.
Then, writing $X = (X^1, \ldots, X^d)$, the process $f(X)$ is a semi-martingale and we have
\begin{align*}
  f(X_t)
  &= f(X_0)
  + \sum_{i=1}^d \int_0^t \frac{\partial f}{\partial x_i}(X_s) \: dX^i_s
  + \frac{1}{2} \sum_{i,j=1}^d \int_0^t \frac{\partial^2 f}{\partial x_i \partial x_j}(X_s) \: d[X^i, X^j]_s
  \: .
\end{align*}
\end{theorem}

\begin{proof}
  \uses{thm:integration_by_parts}

\end{proof}
