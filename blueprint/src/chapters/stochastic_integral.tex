\chapter{Stochastic integral}

The lecture notes at \href{https://dec41.user.srcf.net/h/III_L/stochastic_calculus_and_applications/}{this link} as well as chapter 18 of \cite{kallenberg2021} are good references for this chapter.
Some of the proofs are taken from \cite{pascucci2024}.

\section{Total variation and Lebesgue-Stieltjes integral}

TODO: in Mathlib, we can integrate with respect to the measure given by a right-continuous monotone function (\texttt{StieltjesFunction.measure}). This will be useful to integrate against the quadratic variation of a local martingale.
However, we will also want to integrate with respect to a signed measure given by a càdlàg function with finite variation.
We need to investigate what's already in Mathlib. See \texttt{Mathlib.Topology.EMetricSpace.BoundedVariation}.

\section{Square integrable martingales}

In this section, $E$ denotes a complete normed space.

\begin{definition}[Square integrable martingales]\label{def:IsSquareIntegrable}
  \uses{def:Martingale}
Let $T$ be a linear order with bottom element 0, on which we have a filtration $\mathcal{F}$ satisfying the usual conditions.
We say that a martingale $M : T \to \Omega \to E$ is square integrable if it is càdlàg and $\sup_{t \in T} \mathbb{E}[\Vert M_t \Vert^2] < \infty$ (Lean remark: use \texttt{eLpNorm (M t) 2}).
We denote by $\mathcal{M}^2(E)$ or simply $\mathcal{M}^2$ the space of square integrable martingales with values in $E$.
\end{definition}

TODO: add results about $M_\infty$ for $M \in \mathcal{M}^2$~.

\begin{theorem}\label{thm:hilbertSpace_isSquareIntegrable}
  \uses{def:IsSquareIntegrable, def:limitProcess}
The space $\mathcal{M}^2$ is a Hilbert space with the inner product defined by
\begin{align*}
  \langle M, N \rangle = \mathbb{E}[M_\infty N_\infty]
  \: .
\end{align*}
\end{theorem}

\begin{proof}
  \uses{cor:doob_lp_norm}

\end{proof}


\begin{lemma}\label{lem:eLpNorm_elemStochIntegralBilin_le}
  \uses{def:IsSquareIntegrable, def:elemStochIntegralBilin}
For $V \in \mathcal{E}_{T, F}$ bounded by a constant $D$, $M \in \mathcal{M}^2(E)$ and a continuous bilinear map $B: E \times F \to G$,
\begin{align*}
  \Vert (V \bullet_B M)_t \Vert_{L^2}
  \le 2 D \: \Vert B \Vert \: \sup_t \Vert M_t \Vert_{L^2}
\end{align*}
\end{lemma}

\begin{proof}
Let $C$ be a bound on $\Vert M_t \Vert_{L^2}$ for all $t \in T$~.
Let $(s_k < t_k)_{k \in \{1, ..., n\}}$ and $\eta_k$ be the intervals and random variables defining $V$~.
Let $D$ be a bound on $\Vert\eta_k\Vert$.
Then, for all $t$~,
\begin{align*}
  \Vert (V \bullet_B M)_t \Vert_{L^2}
  &\le \sum_{k=1}^n \Vert B(M^t_{t_k} - M^t_{s_k}, \eta_k) \Vert_{L^2}
  \: .
\end{align*}
Since only at most one term of that sum is non-zero for each fixed $t$~, we can bound the sum by the maximum of its terms.
It suffices then to bound each term of that sum.

TODO: here we supposed that the intervals of the simple process are disjoint. Check with our Lean def.

For each $k$~,
\begin{align*}
  \Vert B(M^t_{t_k} - M^t_{s_k}, \eta_k) \Vert_{L^2}
  &\le \left\Vert \Vert B \Vert \: \Vert M^t_{t_k} - M^t_{s_k} \Vert  \: \Vert \eta_k \Vert \right\Vert_{L^2}
  \\
  &\le \Vert B \Vert \: \Vert M^t_{t_k} - M^t_{s_k} \Vert_{L^2} \: D
  \\
  &\le 2 \Vert B \Vert \: C \: D
  \: .
\end{align*}
\end{proof}


\begin{lemma}\label{lem:isSquareIntegrable_elemStochIntegralBilin}
  \uses{def:IsSquareIntegrable, def:elemStochIntegralBilin}
For $V \in \mathcal{E}_{T, F}$, $M \in \mathcal{M}^2(E)$ and a continuous bilinear map $B: E \times F \to G$, the elementary stochastic integral $V \bullet_B M$ is in $\mathcal{M}^2(G)$.
\end{lemma}

\begin{proof}
  \uses{lem:cadlag_elemStochIntegralBilin, lem:Martingale.elemStochIntegral, lem:eLpNorm_elemStochIntegralBilin_le}
By Lemma~\ref{lem:cadlag_elemStochIntegralBilin}, $V \bullet_B M$ is càdlàg, and we know that it is a martingale by Lemma~\ref{lem:Martingale.elemStochIntegral}~.
It remains to show that $\sup_{t \in T} \Vert (V \bullet_B M)_t \Vert_{L^2} < \infty$~.
By Lemma~\ref{lem:eLpNorm_elemStochIntegralBilin_le}, this supremum is bounded by $2 D \Vert B \Vert \sup_{t \in T} \Vert M_t \Vert_{L^2}$, which is finite since $M \in \mathcal{M}^2(E)$ and $V$ is bounded.
\end{proof}


\begin{lemma}\label{lem:inner_elemStochIntegral}
  \uses{def:IsSquareIntegrable, def:elemStochIntegralBilin, lem:isSquareIntegrable_elemStochIntegralBilin}
For $V \in \mathcal{E}_{T, \mathbb{R}}$ and $M, N \in \mathcal{M}^2$, we have
\begin{align*}
  \langle V \bullet_{\mathbb{R}} M, N \rangle_{\mathcal{M}^2}
  &= V \bullet_{\mathbb{R}} \langle M, N \rangle_{\mathcal{M}^2}
  \: .
\end{align*}
\end{lemma}

\begin{proof}

\end{proof}


\section{Local martingales}

TODO: filtrations should be assumed right-continuous and complete whenever needed.


\begin{lemma}\label{lem:IsLocalMartingale.isLocalSubmartingale_sq_norm}
  \uses{def:IsLocalMartingale, def:IsLocalSubmartingale}
  \leanok
  \lean{ProbabilityTheory.IsLocalMartingale.isLocalSubmartingale_sq_norm}
If $M$ is a cadlag local martingale, then $\Vert M \Vert^2$ is a cadlag local sub-martingale.
\end{lemma}

\begin{proof}

\end{proof}


\begin{definition}[Quadratic variation]\label{def:quadraticVariation}
  \uses{def:IsLocalMartingale, thm:local_doobMeyer, lem:IsLocalMartingale.isLocalSubmartingale_sq_norm}
  \leanok
  \lean{ProbabilityTheory.quadraticVariation}
For any continuous local martingale $M$, there exists a continuous process $[M]$ with $[M]_0 = 0$ such that $\Vert M \Vert^2 - [M]$ is a local martingale. That process is a.s. unique and is called the \emph{quadratic variation} of $M$.
$[M]$ is defined as the predictable part of the Doob-Meyer decomposition of the local sub-martingale $\Vert M \Vert^2$~.
\end{definition}


\begin{definition}[Covariation]\label{def:covariation}
  \uses{def:IsLocalMartingale, def:quadraticVariation}
For any continuous local martingales $M$ and $N$, there exists a continuous process $[M,N]$ with $[M,N]_0 = 0$ such that $MN - [M,N]$ is a local martingale. That process is a.s. unique and is called the \emph{covariation} of $M$ and $N$.

It can be defined by $[M, N]_t = \frac{1}{4}\left([M+N]_t - [M-N]_t \right)$~.
\end{definition}


\begin{lemma}\label{lem:covariation_eq_inner}
  \uses{def:covariation, def:IsSquareIntegrable}
Let $M$ and $N$ be continuous square integrable martingales. Then
\begin{align*}
  \mathbb{E}\left[[M,N]_\infty\right] = \langle M - M_0, N - N_0 \rangle_{\mathcal{M}^2}
  \: .
\end{align*}
\end{lemma}

\begin{proof}

\end{proof}


\begin{lemma}\label{lem:quadraticVariation_brownian}
  \uses{def:brownian, def:quadraticVariation}
Let $B$ be a standard Brownian motion. Then the quadratic variation of $B$ is given by $[B]_t = t$~.
\end{lemma}

\begin{proof}

\end{proof}


\begin{definition}[Continuous semi-martingale]\label{def:continuousSemiMartingale}
  \uses{def:IsLocalMartingale}
A continuous semi-martingale is a process that can be decomposed into a local martingale and a finite variation process.
More formally, a process $X : \mathbb{R}_+ \to \Omega \to E$ is a continuous semi-martingale if there exists a continuous local martingale $M$ and a continuous adapted process $A$ with locally finite variation and $A_0 = 0$ such that
\begin{align*}
  X_t = M_t + A_t
\end{align*}
for all $t \ge 0$.
The decomposition is a.s. unique.
\end{definition}


\section{Stochastic integral}

TODO: relax continuity of the martingales, be clear about continuous quadratic variation vs general càdlàg quadratic variation.

For $M$ a continuous local martingale and $X$ a stochastic process, we will use integrals of the form $\int_0^t X_s \: d[M]_s$~.
Let's explain what those integrals mean. For all $\omega \in \Omega$, $[M](\omega)$ is a right-continuous non-decreasing function (called a Stieltjes function in Mathlib) so it defines a measure on $\mathbb{R}_+$, denoted by $d[M]$~.
Then, for each fixed $\omega \in \Omega$, if the function $s \mapsto X_s(\omega)$ is integrable with respect to the measure $d[M](\omega)$, the Bochner integral $\int_0^t X_s(\omega) \: d[M](\omega)$ is well-defined.
By $\int_0^t X_s \: d[M]_s$, we mean the random variable $\omega \mapsto \int_0^t X_s(\omega) \: d[M](\omega)$~.
If we also vary $t$, we get a stochastic process.


\subsection{Itô isometry}

\begin{definition}\label{def:L2M}
  \uses{def:IsSquareIntegrable, def:quadraticVariation, def:predictableMeasurableSpace}
Let $M \in \mathcal{M}^2$ be a continuous square integrable martingale. We define
\begin{align*}
  L^2(M) = L^2(\Omega \times \mathbb{R}_+, \mathcal{P}, \mathbb{P} \times d[M])
\end{align*}
in which $\mathcal{P}$ is the predictable $\sigma$-algebra and $d[M]$ is the measure induced by the quadratic variation of $M$.
The norm on that Hilbert space is $\Vert X \Vert^2 = \mathbb{E}\left[ \int_0^{\infty} X_t^2 \: d[M]_t \right]$~.
\end{definition}

TODO the sources don't use the same assumptions: predictable vs progressive (\texttt{MeasureTheory.ProgMeasurable}). Progressive would be more general.

\begin{lemma}\label{lem:sq_norm_elemStochIntegral}
  \uses{def:elemStochIntegral, def:IsSquareIntegrable, def:quadraticVariation, def:L2M}
For $V \in \mathcal{E}$ and $M \in \mathcal{M}^2$, then $V \bullet M \in \mathcal{M}^2$ (by Lemma~\ref{lem:isSquareIntegrable_elemStochIntegralBilin}) and
\begin{align*}
  \Vert V \bullet M \Vert_{\mathcal{M}^2}^2
  &= \Vert V \Vert_{L^2(M)}^2
  \: .
\end{align*}
\end{lemma}

\begin{proof}
  \uses{lem:Martingale.elemStochIntegral, lem:isSquareIntegrable_elemStochIntegralBilin}
  There are two steps to the proof.

  First, in order to make sense of $\Vert V \Vert_{L^2(M)}$,
  we define the natural linear map from $\mathcal{E}$ to $L^2(M)$ via \texttt{SimpleProcess.toFun}.
  (Informally, this is identifying $\mathcal{E}$ as a subset $\mathcal{E} \subseteq L^2(M)$.)
  This induces the $L^2(M)$-norm on $\mathcal{E}$, using something like \texttt{NormedSpace.induced}.

  Next, we show that integration $V \mapsto V \bullet M$ is an isometry from $\mathcal{E}$
  with the $L^2(M)$-norm, to $\mathcal{M}^2$. The proof is TODO.
\end{proof}


\begin{lemma}\label{lem:integral_process_eq_zero}
  \uses{def:L2M}
Let $X\in L^2(M)$ such that $\int_0^t X_s \: d[M]_s = 0$ for all $t \ge 0$ a.s.. Then $X = 0$ $(\mathbb{P} \times d[M])$-almost everywhere.
\end{lemma}

\begin{proof}
For $B$ a measurable set of $\mathbb{R}_+$ and $\omega \in \Omega$, let $\nu_\omega(B) = \int_B X_s(\omega) \: d[M]_s(\omega)$~. This is a signed measure on $\mathbb{R}_+$~.
Then if for all $t$, $\int_0^t X_s(\omega) \: d[M]_s(\omega) = 0$ then $\nu_\omega([0,t]) = 0$ for all $t$.
Those intervals generate the Borel $\sigma$-algebra on $\mathbb{R}_+$, so $\nu_\omega$ is the zero measure.
Thus, for almost all $\omega$, $\nu_\omega$ is the zero measure.

The measure $\nu_\omega$ is absolutely continuous with respect to the measure $d[M](\omega)$~, and its Radon-Nikodym derivative is $X(\omega)$~.
Since $\nu_\omega$ is the zero measure for almost all $\omega$, we have that $X(\omega) = 0$ $d[M](\omega)$-almost everywhere for almost all $\omega$~.
Equivalently, $X = 0$ $(\mathbb{P} \times d[M])$-almost everywhere.
\end{proof}


\begin{lemma}[Injectivity of the integral]\label{lem:integral_process_injective}
  \uses{def:L2M}
Let $X, Y \in L^2(M)$ such that $\int_0^t X_s \: d[M]_s = \int_0^t Y_s \: d[M]_s$ for all $t \ge 0$ a.s.. Then $X = Y$ almost everywhere.
\end{lemma}

\begin{proof}
  \uses{lem:integral_process_eq_zero}
By linearity of the integral, we have $\int_0^t (X_s - Y_s) \: d[M]_s = 0$ for all $t \ge 0$ a.s..
Then apply Lemma~\ref{lem:integral_process_eq_zero} to $X - Y$~.
\end{proof}


\begin{lemma}\label{lem:martingale_integral_of_forall_eq_zero}
  \uses{def:L2M}
Let $X \in L^2(M)$ such that $\langle X, V\rangle = 0$ for any simple process $V$. Let $A_t = \int_0^t X_s \: d[M]_s$. Then $A_t$ is a martingale.
\end{lemma}

\begin{proof}
  \uses{lem:martingale_iff_integral_elemStochIntegral_eq_zero}
By Lemma~\ref{lem:martingale_iff_integral_elemStochIntegral_eq_zero}, it is enough to show that for any bounded real simple process $V$, $\mathbb{E}[(V \bullet A)_\infty] = 0$~.
\begin{align*}
  \mathbb{E}\left[(V \bullet A)_\infty\right]
  &= \mathbb{E}\left[ \int_0^{\infty} V_t X_t \: d[M]_t \right]
  \\
  &= \langle X, V \rangle
  \\
  &= 0
  \: .
\end{align*}
\end{proof}


\begin{lemma}\label{lem:integral_eq_zero_of_forall_eq_zero}
  \uses{def:L2M}
Let $X \in L^2(M)$ such that $\langle X, V\rangle = 0$ for any simple process $V$. Then for all $t$, $\int_0^t X_s \: d[M]_s = 0$.
\end{lemma}

\begin{proof}
  \uses{thm:IsLocalMartingale.eq_zero_of_finiteVariation, lem:martingale_integral_of_forall_eq_zero, lem:Martingale.IsLocalMartingale}
$A_t := \int_0^t X_s \: d[M]_s$ is a finite variation process such that $A_t$ is integrable for all $t \ge 0$~.
By Theorem~\ref{thm:IsLocalMartingale.eq_zero_of_finiteVariation}, it is enough to show that $A$ is a local martingale. We have by Lemma~\ref{lem:martingale_integral_of_forall_eq_zero} that $A$ is a martingale, and hence a local martingale (Lemma~\ref{lem:Martingale.IsLocalMartingale}).
\end{proof}


\begin{lemma}\label{lem:dense_simpleProcess}
  \uses{def:L2M, def:simpleProcess}
The set of simple processes is dense in $L^2(M)$.
\end{lemma}

\begin{proof}
  \uses{lem:integral_eq_zero_of_forall_eq_zero, lem:integral_process_eq_zero}
Since $L^2(M)$ is a Hilbert space, it is enough to show that if $X \in L^2(M)$ is orthogonal to all simple processes, then $X = 0$~.
Let $X \in L^2(M)$ such that for any simple process $V$, $\mathbb{E}\left[ \int_0^{\infty} X_t V_t \: d[M]_t \right] = 0$~.
Let $A_t = \int_0^t X_s \: d[M]_s$.
It suffices to show that $A = 0$ by Lemma~\ref{lem:integral_process_eq_zero}~.
This is proved in Lemma~\ref{lem:integral_eq_zero_of_forall_eq_zero}~.

In Lean, this is stated as: the natural linear map \texttt{toFun} from $\mathcal{E}$ to $L^2(M)$
is \texttt{IsDenseInducing}. The Itô isometry is then defined using \texttt{IsDenseInducing.extend}.
\end{proof}


\begin{definition}[Itô isometry]\label{def:itoIsometry}
  \uses{lem:dense_simpleProcess, lem:sq_norm_elemStochIntegral, thm:hilbertSpace_isSquareIntegrable}
Let $M \in \mathcal{M}^2$. Then the elementary stochastic integral map $\mathcal{E} \to \mathcal{M}^2$ defined by $V \mapsto V \bullet M$ extends to an isometry $L^2(M) \to \mathcal{M}^2$.
\end{definition}


\begin{lemma}\label{lem:inner_itoIsometry}
  \uses{def:itoIsometry}
$\langle X \cdot M, Y \cdot M \rangle_{\mathcal{M}^2} = (XY) \cdot \langle M, N \rangle_{\mathcal{M}^2}$.
\end{lemma}

\begin{proof}

\end{proof}


\subsection{Local martingales}

\begin{definition}[$L^2_{loc}(M)$]\label{def:L2locM}
  \uses{def:L2M}
Let $M$ be a continuous local martingale.
We define $L^2_{loc}(M)$ as the space of predictable processes $X$ such that for all $t \ge 0$, $\mathbb{E}\left[ \int_0^t X_s^2 \: d[M]_s \right] < \infty$.
\end{definition}


\begin{definition}[Stochastic integral for continuous local martingales]\label{def:locStochIntegral}
  \uses{def:L2locM, def:itoIsometry, def:covariation}
Let $M$ be a continuous local martingale and let $X \in L^2_{loc}(M)$.
We define the local stochastic integral $X \cdot M$ as the unique continuous local martingale with $(X \cdot M)_0 = 0$ such that for any continuous local martingale $N$, almost surely,
\begin{align*}
  [X \cdot M, N] = X \cdot [M, N]
  \: .
\end{align*}
\end{definition}


\subsection{Semi-martingales}

\begin{definition}\label{def:stochIntegral}
  \uses{def:continuousSemiMartingale, def:locStochIntegral}
For a continuous semi-martingale $X = M + A$ and $V \in L^2_{semi}(X)$ (to be defined) we define the stochastic integral as
\begin{align*}
  V \cdot X = V \cdot M + V \cdot A
  \: ,
\end{align*}
in which $V \cdot M$ is the local stochastic integral defined in \ref{def:locStochIntegral} and $V \cdot A$ is the Lebesgue-Stieltjes integral with respect to the locally finite variation process $A$.
\end{definition}


For $X = M + A$ and $Y = N + B$, we define the covariation as
\begin{align*}
  [X, Y] = [M, N]
  \: .
\end{align*}

\section{Itô formula}


\begin{theorem}[Integration by parts]\label{thm:integration_by_parts}
  \uses{def:continuousSemiMartingale, def:stochIntegral}
Let $X$ and $Y$ be two continuous semi-martingales. Then we have almost surely
\begin{align*}
  X_t Y_t - X_0 Y_0
  = (X \cdot Y)_t + (Y \cdot X)_t + [X,Y]_t
  \: .
\end{align*}
\end{theorem}

\begin{proof}

\end{proof}


\begin{theorem}[Itô's formula]\label{thm:Ito_formula}
  \uses{def:continuousSemiMartingale}
Let $X^1, \ldots, X^d$ be continuous semi-martingales and let $f : \mathbb{R}^d \to \mathbb{R}$ be a twice continuously differentiable function.
Then, writing $X = (X^1, \ldots, X^d)$, the process $f(X)$ is a semi-martingale and we have
\begin{align*}
  f(X_t)
  &= f(X_0)
  + \sum_{i=1}^d \int_0^t \frac{\partial f}{\partial x_i}(X_s) \: dX^i_s
  + \frac{1}{2} \sum_{i,j=1}^d \int_0^t \frac{\partial^2 f}{\partial x_i \partial x_j}(X_s) \: d[X^i, X^j]_s
  \: .
\end{align*}
\end{theorem}

\begin{proof}
  \uses{thm:integration_by_parts}

\end{proof}
