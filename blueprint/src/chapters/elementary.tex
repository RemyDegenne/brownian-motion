\chapter{Simple processes and elementary integrals}


\begin{definition}[Simple process]\label{def:simpleProcess}
  \uses{def:filtration}
  \leanok
  \lean{ProbabilityTheory.SimpleProcess}
Let $(s_k < t_k)_{k \in \{1, ..., n\}}$ be points in a linear order $T$ with a bottom element 0.
Let $(\eta_k)_{0 \le k \le n}$ be bounded random variables with values in a normed real vector space $F$ such that $\eta_0$ is $\mathcal{F}_0$-measurable and $\eta_k$ is $\mathcal{F}_{s_k}$-measurable for $k \ge 1$.
Then the simple process for that sequence is the process $V : T \to \Omega \to F$ defined by
\begin{align*}
  V_t = \eta_0 \mathbb{1}_{\{0\}}(t) + \sum_{k=1}^{n} \eta_k \mathbb{1}_{(s_k, t_k]}(t)
  \: .
\end{align*}
Let $\mathcal{E}_{T, F}$ be the set of simple processes indexed by $T$ with value in $F$.
\end{definition}


\begin{definition}[Elementary predictable set]\label{def:elementaryPredictableSet}
  \uses{def:filtration}
  \leanok
  \lean{ProbabilityTheory.ElementaryPredictableSet}
A set $A \subseteq T \times \Omega$ is an \emph{elementary predictable set} if it is a finite union of sets of the form ${0} \times B$ for $B \in \mathcal{F}_0$ or of the form $(s, t] \times B$ with $0 \le s < t$ in $T$ and $B \in \mathcal{F}_s$.
\end{definition}


\begin{lemma}\label{lem:predictableSet_elementaryPredictableSet}
  \uses{def:elementaryPredictableSet, def:predictableMeasurableSpace}
  \leanok
  \lean{ProbabilityTheory.ElementaryPredictableSet.measurableSet_predictable}
An elementary predictable set is measurable with respect to the predictable $\sigma$-algebra.
\end{lemma}

\begin{proof}
  \uses{lem:predictable_Ioc_prod}
  \leanok
It is a union of sets of the form ${0} \times B$ with $B \in \mathcal{F}_0$ or of the form $(s, t] \times B$ with $B \in \mathcal{F}_s$, which are measurable by Lemma~\ref{lem:predictable_Ioc_prod}.
\end{proof}


\begin{lemma}\label{lem:predictable_simpleProcess}
  \uses{def:predictable, def:simpleProcess}
  \leanok
  \lean{ProbabilityTheory.SimpleProcess.isPredictable}
A simple process is predictable.
\end{lemma}

\begin{proof}
  \leanok
\end{proof}


\begin{lemma}\label{lem:iSup_comap_simpleProcess}
  \uses{def:simpleProcess, def:predictableMeasurableSpace}
  \leanok
  \lean{ProbabilityTheory.SimpleProcess.iSup_comap_eq_predictable}
Real simple processes generate the predictable $\sigma$-algebra, i.e., the predictable $\sigma$-algebra is the supremum of the comaps by simple processes (seen as maps from $T \times \Omega$ to $\mathbb{R}$) of the Borel $\sigma$-algebra.
\end{lemma}

\begin{proof}
  \uses{lem:predictable_simpleProcess}
  \leanok
\end{proof}


\begin{lemma}\label{lem:elementaryPredictableSet_iff_indicator}
  \uses{def:elementaryPredictableSet, def:simpleProcess}
  \leanok
  \lean{ProbabilityTheory.ElementaryPredictableSet.indicator}
A set $A \subseteq T \times \Omega$ is an elementary predictable set if and only if the indicator function $\mathbb{1}_A$ is a simple process.
\end{lemma}

\begin{proof}
  Currently only proved the forward direction, because the backward direction is not easy to state.
\end{proof}


\begin{lemma}\label{lem:addCommGroup_simpleProcess}
  \uses{def:simpleProcess}
  \leanok
  \lean{ProbabilityTheory.SimpleProcess.instAddCommGroup}
The simple processes $\mathcal{E}_{T, F}$ form an additive commutative group.
\end{lemma}

\begin{proof}
  \leanok
\end{proof}


\begin{lemma}\label{lem:module_simpleProcess}
  \uses{def:simpleProcess}
  \leanok
  \lean{ProbabilityTheory.SimpleProcess.instModule}
The simple processes $\mathcal{E}_{T, F}$ form a module over the scalars $\mathbb{R}$.
\end{lemma}

\begin{proof}
  \leanok
\end{proof}


\begin{definition}[Elementary stochastic integral]\label{def:elemStochIntegralBilin}
  \uses{def:simpleProcess}
  \leanok
  \lean{ProbabilityTheory.SimpleProcess.integral}
Let $V \in \mathcal{E}_{T, F}$ be a simple process and let $X$ be a stochastic process with values in a normed space $E$.
Let $B$ be a continuous bilinear map from $E \times F$ to another normed space $G$.
The \emph{general elementary stochastic integral} process $V \bullet_B X : T \to \Omega \to G$ is defined by
\begin{align*}
  (V \bullet_B X)_t
  &= \sum_{k=1}^{n} B (X^t_{t_k} - X^t_{s_k}, \eta_k)
  \: .
\end{align*}
\end{definition}


\begin{lemma}\label{lem:simpleProcess_elemStochIntegralBilin}
  \uses{def:elemStochIntegralBilin, def:simpleProcess}
  \leanok
  \lean{ProbabilityTheory.SimpleProcess.integralSimpleProcess}
The elementary stochastic integral $V \bullet_B W$ of two simple processes is a simple process.

TODO: with the current Lean definition of simple process, this cannot be a lemma but has to be a definition of an elementary stochastic integral of two simple processes, with value in the simple processes.
\end{lemma}

\begin{proof}
  \leanok
It has values $B(\xi_j, \eta_i)$ on the intervals $(s_i, t_i] \cap (u_j, v_j]$ if $V$ and $W$ are defined by the sequences $(s_i, t_i, \eta_i)$ and $(u_j, v_j, \xi_j)$ respectively.
\end{proof}


\begin{definition}\label{def:elemStochIntegralLinear}
  \uses{def:elemStochIntegralBilin}
  \lean{ProbabilityTheory.SimpleProcess.integralLinear}
  \leanok
Let $E$ and $G$ be normed real vector spaces.
We call \emph{linear elementary stochastic integral} the general elementary stochastic integral of Definition~\ref{def:elemStochIntegralBilin} with $G = G$, $F = L(E, G)$ (the continuous linear maps from $E$ to $G$) and $B(x, L) = L(x)$ the evaluation map.
We denote it by $V \bullet_L X$.
\end{definition}


\begin{definition}\label{def:elemStochIntegral}
  \uses{def:elemStochIntegralBilin}
We call \emph{scalar elementary stochastic integral} the general elementary stochastic integral of Definition~\ref{def:elemStochIntegralBilin} with $G = E$, $F = \mathbb{R}$ and $B(x, r) = r \cdot x$ the scalar multiplication.
We denote it by $V \bullet_{\mathbb{R}} X$.
\end{definition}


\begin{lemma}\label{lem:elemStochIntegral_linear}
  \uses{def:elemStochIntegralBilin, def:elemStochIntegralLinear, def:elemStochIntegral, lem:addCommGroup_simpleProcess, lem:module_simpleProcess}
  \leanok
  \lean{ProbabilityTheory.SimpleProcess.integral_add_left, ProbabilityTheory.SimpleProcess.integral_smul_left, ProbabilityTheory.SimpleProcess.integral_add_right, ProbabilityTheory.SimpleProcess.integral_smul_right, ProbabilityTheory.SimpleProcess.integral_sub_left, ProbabilityTheory.SimpleProcess.integral_sub_right, ProbabilityTheory.SimpleProcess.integral_zero_left, ProbabilityTheory.SimpleProcess.integral_zero_right}
The elementary stochastic integrals $\bullet_B$, $\bullet_L$ and $\bullet_{\mathbb{R}}$ are linear in both arguments.
\end{lemma}

\begin{proof}
  \leanok
(In Lean, this is split into several lemmas about each argument and add/sub/smul.)

Immediate from the definition.
\end{proof}


\begin{lemma}\label{lem:elemStochIntegralBilin_zero}
  \uses{def:elemStochIntegralBilin}
  \leanok
  \lean{ProbabilityTheory.SimpleProcess.integral_bot}
$(V \bullet_B X)_0 = 0$ for every simple process $V$ and stochastic process $X$.
\end{lemma}

\begin{proof}
  \leanok
\begin{align*}
  (V \bullet_B X)_0
  &= \sum_{k=1}^{n} B (X^0_{t_k} - X^0_{s_k}, \eta_k)
  \\
  &= \sum_{k=1}^{n} B (X_{0} - X_{0}, \eta_k)
  \\
  &= \sum_{k=1}^{n} B (0, \eta_k)
  \\
  &= 0
  \: .
\end{align*}
\end{proof}


\begin{lemma}\label{lem:elemStochIntegralBilin_const}
  \uses{def:elemStochIntegralBilin}
  \leanok
  \lean{ProbabilityTheory.SimpleProcess.integral_const}
Let $X_c$ be a constant process (equal to the same random variable for all times).
Then for every simple process $V$,
\begin{align*}
  V \bullet_B X_c = 0
  \: .
\end{align*}
\end{lemma}

\begin{proof}
  \leanok
\end{proof}


\begin{lemma}\label{lem:elemStochIntegralBilin_assoc}
  \uses{def:elemStochIntegralBilin, lem:simpleProcess_elemStochIntegralBilin}
  \leanok
  \lean{ProbabilityTheory.SimpleProcess.integral_assoc}
Let $E, F, G, H, I, J$ be normed real vector spaces.
Let $B_1 : E \times F \to G$, $B_2 : G \times H \to I$, $B_3 : F \times H \to J$ and $B_4 : E \times J \to I$ be continuous bilinear maps such that for all $x \in E$, $y \in F$ and $z \in H$, $B_2(B_1(x, y), z) = B_4(x, B_3(y, z))$.
Then for every simple process $W$ with values in $F$, simple process $V$ with values in $H$, and stochastic process $X$ with values in $E$, we have
\begin{align*}
  V \bullet_{B_2} (W \bullet_{B_1} X)
  &= (V \bullet_{B_3} W) \bullet_{B_4} X
  \: .
\end{align*}
\end{lemma}

\begin{proof}
  \uses{lem:elemStochIntegral_linear}
Unfold definitions, use linearity.
\end{proof}


\begin{corollary}\label{cor:elemStochIntegral_assoc_real_bilin}
Let $E, F, G$ be normed real vector spaces.
Let $B : E \times F \to G$ be a continuous bilinear map.
Then for every simple process $W$ with values in $F$, simple process $V$ with values in $\mathbb{R}$, and stochastic process $X$ with values in $E$, we have
\begin{align*}
  V \bullet_{\mathbb{R}} (W \bullet_{B} X)
  &= (V \bullet_{\mathbb{R}} W) \bullet_{B} X
  \: .
\end{align*}
\end{corollary}

\begin{proof}
  \uses{lem:elemStochIntegralBilin_assoc}
Use Lemma~\ref{lem:elemStochIntegralBilin_assoc} with $B_1, B_4$ equal to $B$, $B_2$ equal to the scalar multiplication on $G$ and $B_3$ equal to the scalar multiplication on $F$.
\end{proof}


\begin{lemma}\label{lem:elemStochIntegral_assoc}
  \uses{def:elemStochIntegral}
$V \bullet_{\mathbb{R}} (W \bullet_{\mathbb{R}} X) = (V \times W) \bullet_{\mathbb{R}} X$ for real-valued simple processes $V, W$ and stochastic process $X$.
\end{lemma}

\begin{proof}
  \uses{cor:elemStochIntegral_assoc_real_bilin}
Apply Corollary~\ref{cor:elemStochIntegral_assoc_real_bilin} with $B$ the scalar multiplication.
\end{proof}


\begin{lemma}\label{lem:elemStochIntegralLinear_assoc}
  \uses{def:elemStochIntegralLinear}
  \lean{ProbabilityTheory.SimpleProcess.integralLinear_assoc}
  \leanok
For simple process $W$ with values in $L(E, G)$, simple process $V$ with values in $L(G, H)$, and stochastic process $X$ with values in $E$, we have
\begin{align*}
  V \bullet_L (W \bullet_L X)
  &= (V \circ W) \bullet_L X
  \: ,
\end{align*}
in which $V \circ W$ is the simple process defined by $(V \circ W)_t = V_t \circ W_t$.
\end{lemma}

\begin{proof}
  \uses{lem:elemStochIntegralBilin_assoc}
  \leanok
Use Lemma~\ref{lem:elemStochIntegralBilin_assoc} with $B_1, B_2, B_4$ equal to the evaluation maps and $B_3$ the composition of continuous linear maps.
\end{proof}


\begin{lemma}\label{lem:cadlag_elemStochIntegralBilin}
  \uses{def:elemStochIntegralBilin}
For $V \in \mathcal{E}_{T, F}$, $X : T \to \Omega \to E$ a càdlàg process and a continuous bilinear map $B: E \times F \to G$, the elementary stochastic integral $V \bullet_B X$ is càdlàg.
\end{lemma}

\begin{proof}

\end{proof}


\begin{lemma}\label{lem:elemStochIntegral_stoppedProcess}
  \uses{def:elemStochIntegral, def:stoppedProcess}
  \lean{ProbabilityTheory.SimpleProcess.stoppedProcess_integral}
  \leanok
Let $X$ be a stochastic process, $V$ a simple process and $τ$ a stopping time.
Then
\begin{align*}
  (V \bullet_B X)^τ = V \bullet_B (X^τ)
  \: .
\end{align*}
\end{lemma}

\begin{proof}
  \leanok
\begin{align*}
  (V \bullet_B X)^\tau_t
  &= (V \bullet_B X)_{t \wedge \tau}
  \\
  &= \sum_{k=1}^{n} B (X^{t \wedge \tau}_{t_k} - X^{t \wedge \tau}_{s_k}, \eta_k)
  \\
  &= \sum_{k=1}^{n} B ((X^{\tau})^t_{t_k} - (X^{\tau})^t_{s_k}, \eta_k)
  \\
  &= (V \bullet_B (X^{\tau}))_t
  \: .
\end{align*}
\end{proof}


\begin{lemma}\label{lem:tendsto_elemStochIntegral_limitProcess}
  \uses{def:elemStochIntegralBilin, def:limitProcess}
For $V$ a simple process and $X$ a stochastic process, $(V \bullet_B X)_t$ tends to its limit process $(V \bullet_B X)_\infty$ as $t$ goes to infinity.
\end{lemma}

\begin{proof}
That process is eventually constant.
\end{proof}


\begin{lemma}\label{lem:limitProcess_elemStochIntegral}
  \uses{def:elemStochIntegralBilin, def:limitProcess}
  \lean{ProbabilityTheory.SimpleProcess.integral_top}
  \leanok
\begin{align*}
  (V \bullet_B X)_\infty
  &= \sum_{k=1}^{n} B (X_{t_k} - X_{s_k}, \eta_k)
  \: .
\end{align*}
\end{lemma}

\begin{proof}
  \leanok
Immediate from the definition.
\end{proof}


\begin{lemma}\label{lem:submartingale_iff_integral_elemStochIntegral_nonneg}
  \uses{def:elemStochIntegral, def:Submartingale, def:limitProcess}
An adapted integrable process $X$ is a submartingale if and only if for every bounded simple process $V$ with values in $\mathbb{R}_+$, $\mathbb{E}[(V \bullet_{\mathbb{R}} X)_\infty] \ge 0$.

Note that in Lean $V$ comes with the data of a sum representation: by ``values in $\mathbb{R}_+$'' we mean that the random variables in that particular sum representing $V$ are nonnegative.
\end{lemma}

\begin{proof}
  \uses{def:Submartingale, lem:tendsto_elemStochIntegral_limitProcess, lem:limitProcess_elemStochIntegral}
First suppose that $X$ is a submartingale.
The simple process $V$ can be written as $V_t = \eta_0 \mathbb{1}_{\{0\}}(t) + \sum_{k=1}^{n} \eta_k \mathbb{1}_{(s_k, t_k]}(t)$ for nonnegative $\eta_k$. Then
\begin{align*}
  \mathbb{E}[(V \bullet_{\mathbb{R}} X)_\infty]
  &= \mathbb{E}\left[\sum_{k=1}^{n} \eta_k (X_{t_k} - X_{s_k})\right]
  \\
  &= \sum_{k=1}^{n} \mathbb{E}[\eta_k (X_{t_k} - X_{s_k})]
  \: .
\end{align*}
It suffices to show that each term of the sum is nonnegative.
Since $\eta_k$ is $\mathcal{F}_{s_k}$-measurable and nonnegative, by the submartingale property we have
\begin{align*}
  \mathbb{E}[\eta_k (X_{t_k} - X_{s_k})]
  &= \mathbb{E}[\eta_k \mathbb{E}[X_{t_k} - X_{s_k} \mid \mathcal{F}_{s_k}]]
  \ge 0
  \: .
\end{align*}
This concludes the proof in one direction.
Suppose now that for every bounded simple process $V$ with values in $\mathbb{R}_+$, $\mathbb{E}[(V \bullet_{\mathbb{R}} X)_\infty] \ge 0$.
To show that $X$ is a submartingale, let $s < t$ in $T$ and let $A = \{\mathbb{E}[X_t \mid \mathcal{F}_s] < X_s\} \in \mathcal{F}_s$.
Define the simple process $V$ by $V_r = \mathbb{1}_A \mathbb{1}_{(s, t]}(r)$.
Then
\begin{align*}
  \mathbb{E}[\mathbb{1}_A (X_t - X_s)]
  &= \mathbb{E}[(V \bullet_{\mathbb{R}} X)_\infty]
  \ge 0
  \: .
\end{align*}
But we also have
\begin{align*}
  \mathbb{E}[\mathbb{1}_A (X_t - X_s)]
  &= \mathbb{E}[\min\{\mathbb{E}[X_t \mid \mathcal{F}_s] - X_s, 0\}]
  \le 0
  \: .
\end{align*}
Combining both inequalities, we get $\mathbb{E}[\min\{\mathbb{E}[X_t \mid \mathcal{F}_s] - X_s, 0\}] = 0$.
Since $\min\{\mathbb{E}[X_t \mid \mathcal{F}_s] - X_s, 0\}$ is nonpositive and has expectation zero, it is almost surely zero.
Thus $\mathbb{E}[X_t \mid \mathcal{F}_s] \ge X_s$ almost surely, which concludes the proof.
\end{proof}


\begin{lemma}\label{lem:martingale_iff_integral_elemStochIntegral_eq_zero}
  \uses{def:elemStochIntegral, def:Martingale}
An adapted integrable process $X$ is a martingale if and only if for every bounded simple process $V$ with values in $\mathbb{R}$, $\mathbb{E}[(V \bullet_{\mathbb{R}} X)_\infty] = 0$.
\end{lemma}

\begin{proof}
  \uses{lem:tendsto_elemStochIntegral_limitProcess, def:Martingale}
There might not be an order on the type $E$ of values of $X$, so we cannot just say that $X$ is a martingale if and only if both $X$ and $-X$ are submartingales and conclude with Lemma~\ref{lem:submartingale_iff_integral_elemStochIntegral_nonneg}.
But we can use almost the same proof as in Lemma~\ref{lem:submartingale_iff_integral_elemStochIntegral_nonneg}, replacing inequalities by equalities?
\end{proof}


\begin{corollary}\label{cor:Submartingale.integral_elemStochIntegral_le}
  \uses{def:elemStochIntegral, def:Submartingale}
Let $X$ be a submartingale and $A$ be an elementary predictable set.
Then for all $t \in T$,
\begin{align*}
  \mathbb{E}[(\mathbb{1}_A \bullet_{\mathbb{R}} X)_t] \le \mathbb{E}[X_t - X_0]
  \: .
\end{align*}
\end{corollary}

\begin{proof}
  \uses{lem:submartingale_iff_integral_elemStochIntegral_nonneg}
Let $A$ be an elementary predictable set and let $t \in T$.
\begin{align*}
  \mathbb{E}[(\mathbb{1}_A \bullet_{\mathbb{R}} X)_t]
  &= \mathbb{E}[X_t - X_0] - \mathbb{E}[(\mathbb{1}_{A^c} \bullet_{\mathbb{R}} X)_t]
  \le \mathbb{E}[X_t - X_0]
  \: .
\end{align*}
The inequality follows from Lemma~\ref{lem:submartingale_iff_integral_elemStochIntegral_nonneg} applied to the nonnegative bounded simple process $\mathbb{1}_{A^c}$.
\end{proof}


\begin{lemma}\label{lem:Submartingale.elemStochIntegral}
  \uses{def:elemStochIntegral, def:Submartingale}
Let $X$ be a submartingale and $V$ be a nonnegative bounded simple process.
Then the elementary stochastic integral $V \bullet_{\mathbb{R}} X$ is a submartingale.
\end{lemma}

\begin{proof}
  \uses{lem:submartingale_iff_integral_elemStochIntegral_nonneg, lem:elemStochIntegral_assoc}
We use Lemma~\ref{lem:submartingale_iff_integral_elemStochIntegral_nonneg} and have to show that for every nonnegative bounded simple process $W$, $\mathbb{E}[(W \bullet_{\mathbb{R}} (V \bullet_{\mathbb{R}} X))_\infty] \ge 0$.
By Lemma~\ref{lem:elemStochIntegral_assoc}, this is $\mathbb{E}[((W \times V) \bullet_{\mathbb{R}} X)_\infty]$.
Since $W \times V$ is a nonnegative bounded simple process, this is nonnegative by Lemma~\ref{lem:submartingale_iff_integral_elemStochIntegral_nonneg} applied to $X$.
\end{proof}


\begin{lemma}\label{lem:Martingale.elemStochIntegral}
  \uses{def:elemStochIntegral, def:Martingale}
Let $X$ be a martingale and $V$ be a bounded simple process.
Then the elementary stochastic integral $V \bullet_B X$ is a martingale.
\end{lemma}

\begin{proof}
  \uses{lem:martingale_iff_integral_elemStochIntegral_eq_zero, cor:elemStochIntegral_assoc_real_bilin}
We use Lemma~\ref{lem:martingale_iff_integral_elemStochIntegral_eq_zero} and have to show that for every bounded simple process $W$, $\mathbb{E}[(W \bullet_{\mathbb{R}} (V \bullet_B X))_\infty] = 0$.
By Lemma~\ref{cor:elemStochIntegral_assoc_real_bilin}, this is $\mathbb{E}[((W \bullet_{\mathbb{R}} V) \bullet_B X)_\infty]$.
Since $W \bullet_{\mathbb{R}} V$ is a bounded simple process, this is zero by Lemma~\ref{lem:martingale_iff_integral_elemStochIntegral_eq_zero} applied to $X$.
\end{proof}


\begin{lemma}\label{lem:stoppedProcess_eq_elemStochIntegral}
  \uses{def:elemStochIntegral, def:stoppedProcess}
  \notready
Let $X$ be a stochastic process and $τ$ be a stopping time taking finitely many values.
Then $\mathbb{1}_{[0, τ]}$ is a simple process and
\begin{align*}
  X^τ = X_0 + \mathbb{1}_{[0, τ]} \bullet X
  \: .
\end{align*}
\end{lemma}

\begin{proof}

\end{proof}
