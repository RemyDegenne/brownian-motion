% In this file you should put the actual content of the blueprint.
% It will be used both by the web and the print version.
% It should *not* include the \begin{document}
%
% If you want to split the blueprint content into several files then
% the current file can be a simple sequence of \input. Otherwise It
% can start with a \section or \chapter for instance.

\part{Brownian motion}

\paragraph{Overview}

This part of the blueprint is a guide to the formalization of a standard Brownian motion in Lean using Mathlib. There are two main parts to this formalization:
\begin{itemize}
  \item a development of the theory of Gaussian distributions, the construction of a projective family of Gaussian distributions and its projective limit by the Kolmogorov extension theorem,
  \item a proof of a Kolmogorov-Chentsov continuity theorem, following \cite{kratschmer2023kolmogorov}.
\end{itemize}

Putting the two sides together, we then build a stochastic process that fits the definition of a Brownian motion on the real line.

\paragraph{Status} The formalization is complete.

\paragraph{Formalization authors} Rémy Degenne, Markus Himmel, David Ledvinka, Etienne Marion, Peter Pfaffelhuber.

With additional contributions from Jonas Bayer, Lorenzo Loccioli, Pietro Monticone, Alessio Rondelli and Jérémy Scanvic.

\chapter{Characteristic function and covariance}

\section{Characteristic functions}
\label{sec:characteristic_function}


\begin{definition}[Characteristic function]\label{def:charFunDual}
  \mathlibok
  \lean{MeasureTheory.charFunDual}
The characteristic function of a measure $\mu$ on a normed space $E$ is the function $E^* \to \mathbb{C}$ defined by
\begin{align*}
  \hat{\mu}(L) = \int_E e^{i L(x)} \: d\mu(x) \: .
\end{align*}
\end{definition}


\begin{theorem}\label{thm:ext_of_charFunDual}
  \uses{def:charFunDual}
  \mathlibok
  \lean{MeasureTheory.Measure.ext_of_charFunDual}
In a separable Banach space, if two finite measures have same characteristic function, they are equal.
\end{theorem}

\begin{proof}\leanok

\end{proof}


\begin{definition}[Characteristic function]\label{def:charFun}
  \mathlibok
  \lean{MeasureTheory.charFun}
The characteristic function of a measure $\mu$ on an inner product space $E$ is the function $E \to \mathbb{C}$ defined by
\begin{align*}
  \hat{\mu}(t) = \int_E e^{i \langle t, x \rangle} \: d\mu(x) \: .
\end{align*}
This is equal to the normed space version of the characteristic function applied to the linear map $x \mapsto \langle t, x \rangle$.
\end{definition}


\begin{theorem}\label{thm:ext_of_charFun}
  \uses{def:charFun}
  \mathlibok
  \lean{MeasureTheory.Measure.ext_of_charFun}
In a separable Hilbert space, if two finite measures have same characteristic function, they are equal.
\end{theorem}

\begin{proof}\leanok

\end{proof}


\begin{lemma}\label{lem:charFun_map_eq_charFunDual_smul}
  \uses{def:charFun, def:charFunDual}
  \mathlibok
  \lean{MeasureTheory.charFun_map_eq_charFunDual_smul}
Let $\mu$ be a measure on $F$ and let $L \in F^*$. Then
\begin{align*}
  \widehat{L_*\mu}(x) &= \hat{\mu}(x \cdot L) \: .
\end{align*}
\end{lemma}

\begin{proof}\leanok

\end{proof}


\begin{lemma}\label{lem:charFunDual_map}
  \uses{def:charFunDual}
  \mathlibok
  \lean{MeasureTheory.charFunDual_map}
Let $\mu$ be a measure on a normed space $E$ and let $L$ be a continuous linear map from $E$ to $F$.
Then for all $L' \in F^*$,
\begin{align*}
  \widehat{L_*\mu}(L') = \hat{\mu}(L' \circ L) \: .
\end{align*}
\end{lemma}

\begin{proof}\leanok

\end{proof}



\section{Covariance}
\label{sec:covariance}

Let $F$ be a Banach space and $E$ be a Hilbert space.

\begin{definition}[Covariance]\label{def:covarianceBilin}
  \mathlibok
  \lean{ProbabilityTheory.covarianceBilin, ProbabilityTheory.covarianceBilin_apply, ProbabilityTheory.covarianceBilin_apply'}
The covariance bilinear form of a measure $\mu$ on $F$ with finite second moment is the continuous bilinear form $C_\mu : F^* \times F^* \to \mathbb{R}$ with
\begin{align*}
  C_\mu(L_1, L_2)
  &= \int_x (L_1(x) - L_1(m_\mu)) (L_2(x) - L_2(m_\mu)) \: d\mu(x)
  \\
  &= \int_x L_1(x - m_\mu) L_2(x- m_\mu) \: d\mu(x)
  \: .
\end{align*}
\end{definition}

\begin{lemma}\label{lem:covarianceBilin_same_eq_variance}
  \uses{def:covarianceBilin}
  \mathlibok
  \lean{ProbabilityTheory.covarianceBilin_same_eq_variance}
For $\mu$ a measure on $F$ with finite second moment and $L \in F^*$, $C_\mu(L, L) = \mathbb{V}_\mu[L]$.
\end{lemma}

\begin{proof}\leanok

\end{proof}


\begin{definition}[Covariance in a Hilbert space]\label{def:covInnerBilin}
  \leanok
  \lean{ProbabilityTheory.covInnerBilin}
The covariance bilinear form of a finite measure $\mu$ with finite second moment on a Hilbert space $E$ is the continuous bilinear form $C_\mu : E \times E \to \mathbb{R}$ with
\begin{align*}
  C'_\mu(x, y) = \int_z \langle x, z - m_\mu \rangle \langle y, z - m_\mu \rangle \: d\mu(z) \: .
\end{align*}
This is $C_\mu$ applied to the linear maps $L_x, L_y \in E^*$ defined by $L_x(z) = \langle x, z \rangle$ and $L_y(z) = \langle y, z \rangle$.
\end{definition}


\begin{lemma}\label{lem:covInnerBilin_map}
  \uses{def:covInnerBilin}
  \leanok
  \lean{ProbabilityTheory.covInnerBilin_map}
Let $E$ and $F$ be two Hilbert spaces with $F$ finite dimensional, $\mu$ a finite measure on $E$ with finite second moment, and $L : E \to F$ a continuous linear map.
Then the covariance bilinear form of the measure $L_*\mu$ is given by
\begin{align*}
  C'_{L_*\mu}(u, v)
  &= C'_\mu(L^\dagger(u), L^\dagger(v))
  \: ,
\end{align*}
in which $L^\dagger : F \to E$ is the adjoint of $L$.
\end{lemma}

\begin{proof}\leanok
\begin{align*}
  C'_{L_*\mu}(u, v)
  &= (L_*\mu)\left[\langle u, x - m_{L_*\mu}\rangle \langle x - m_{L_*\mu}, v \rangle\right]
  \\
  &= \mu\left[\langle u, L(x) - L(m_\mu)\rangle \langle L(x) - L(m_\mu), v \rangle \right]
  \\
  &= \mu\left[\langle L^\dagger(u), x - m_\mu\rangle \langle x - m_\mu, L^\dagger(v) \rangle \right]
  \\
  &= C'_\mu(L^\dagger(u), L^\dagger(v))
  \: .
\end{align*}
\end{proof}


\begin{definition}[Covariance matrix]\label{def:covMatrix}
  \uses{def:IsGaussian, lem:covarianceBilin_same_eq_variance}
  \leanok
  \lean{ProbabilityTheory.covMatrix, ProbabilityTheory.posSemidef_covMatrix}
The covariance matrix of a finite measure $\mu$ with finite second moment on a finite dimensional inner product space $E$ is the positive semidefinite matrix $\Sigma_\mu$ such that for $u, v \in E$,
\begin{align*}
  \langle u, \Sigma_\mu v\rangle = \mu[\langle u, x - m_\mu \rangle \langle x - m_\mu, v \rangle] \: .
\end{align*}
This is the covariance bilinear form $C'_\mu(u, v)$, as a matrix.
\end{definition}


\begin{lemma}\label{lem:covMatrix_map}
  \uses{def:covMatrix}
Let $E$ and $F$ be two finite dimensional inner product spaces, $\mu$ a measure on $E$ with finite second moment, and $L : E \to F$ a continuous linear map.
Then the covariance matrix of the measure $L_*\mu$ has entries
\begin{align*}
  \langle e_i, \Sigma_{L_*\mu} e_j\rangle
  &= \langle L^\dagger(e_i), \Sigma_\mu L^\dagger(e_j)\rangle
  \: ,
\end{align*}
in which $L^\dagger : F \to E$ is the adjoint of $L$.
\end{lemma}

\begin{proof}
  \uses{lem:covInnerBilin_map}

\end{proof}

\chapter{Stochastic processes}
\label{chap:process}

Let $T$ be an index set and $\Omega$ a measurable space, with measure $\mathbb{P}$.
A stochastic process is a function $X : T \to \Omega \to E$, where $E$ is another measurable space, such that for all $t \in T$, $X_t : \Omega \to E$ is $\mathbb{P}$-a.e. measurable.


\begin{definition}[Law of a stochastic process]\label{def:processLaw}
  \leanok
The law of a stochastic process $X$ is the measure on the measurable space $E^T$ obtained by pushing forward the measure $\mathbb{P}$ by the map $\omega \mapsto X(\cdot, \omega)$.
\end{definition}

\textbf{Lean remark}: we don't use a Lean definition for the law, but write the map in full.

\begin{definition}[Modification]\label{def:modification}
  \leanok
We say that a stochastic process $Y$ is a \emph{modification} of another stochastic process $X$ if for all $t \in T$, $Y_t =_{\mathbb{P}\text{-a.e.}} X_t$.
\end{definition}

\textbf{Lean remark}: we don't use a Lean definition for being a modification, but write explicitly the condition $\forall t \in T,\ Y_t =_{\mathbb{P}\text{-a.e.}} X_t$~.

\begin{definition}[Indistinguishable]\label{def:indistinguishable}
  \leanok
We say that a stochastic processes $Y$ is a \emph{indistinguishable} from $X$ if $\mathbb{P}$-a.e., for all $t \in T$, $X_t = Y_t$.
\end{definition}

A summary of the next few lemmas is this:
\begin{itemize}
  \item indistinguishable $\implies$ modification $\implies$ same law,
  \item modification and continuous with $T$ separable $\implies$ indistinguishable.
\end{itemize}


\begin{lemma}\label{lem:Indistinguishable.Modification}
  \uses{def:indistinguishable, def:modification}
  \leanok
  \lean{modification_of_indistinduishable}
If $Y$ is indistinguishable from $X$, then $Y$ is a modification of $X$.
\end{lemma}

\begin{proof}\leanok
Obvious.
\end{proof}


\begin{lemma}\label{lem:map_eq_of_modification}
  \uses{def:modification}
  \leanok
  \lean{finite_distributions_eq}
Let $X, Y : T \to \Omega \to E$ be two stochastic processes that are modifications of each other.
Then for all $t_1, \ldots, t_n \in T$, the random vector $(X_{t_1}, \ldots, X_{t_n})$ has the same distribution as the random vector $(Y_{t_1}, \ldots, Y_{t_n})$.
That is, $X$ and $Y$ have same finite-dimensional distributions.
\end{lemma}

\begin{proof}\leanok
By the modification property, almost surely $X_{t_i} = Y_{t_i}$ for all $i \in [n]$.
Thus the function $\omega \mapsto (X_{t_1}(\omega), \ldots, X_{t_n}(\omega))$ is equal to $\omega \mapsto (Y_{t_1}(\omega), \ldots, Y_{t_n}(\omega))$ almost surely, hence the maps of $\mathbb{P}$ by these two functions are equal.
\end{proof}


\begin{lemma}\label{lem:map_eq_iff}
  \uses{def:processLaw}
  \leanok
  \lean{finite_distributions_eq_iff_same_law}
Let $X, Y : T \to \Omega \to E$ be two stochastic processes.
Then $X$ and $Y$ have same finite-dimensional distributions if and only if they have the same law.
\end{lemma}

\begin{proof}\leanok
TODO: consider the $\pi$-system of cylinder sets.
\end{proof}


\begin{lemma}\label{lem:indistinguishable_of_modification_of_continuous}
  \uses{def:modification, def:indistinguishable}
  \leanok
  \lean{indistinduishable_of_modification}
Let $T$ and $E$ be topological spaces and suppose that $T$ is separable Hausdorff.
Let $X, Y : T \to \Omega \to E$ be two stochastic processes that are modifications of each other and are almost surely continuous.
Then $X$ and $Y$ are indistinguishable.
\end{lemma}

\begin{proof}\leanok
Since $T$ is separable, it has a countable dense subset $D$.
Since $D$ is countable,
\begin{align*}
  (\forall t \in D, \mathbb{P}\text{-a.e.}, X_t = Y_t)
  \iff (\mathbb{P}\text{-a.e.}, \forall t \in D, X_t = Y_t)
\end{align*}
Hence by the modification property we have that almost surely, for all $t \in D$, $X_t = Y_t$.
Then almost surely $X$ and $Y$ are continuous functions which are equal on a dense subset of $T$: those two functions are equal everywhere.
\end{proof}

\chapter{Gaussian distributions}
\label{chap:gaussian}

\section{Gaussian measures}
\label{sec:gaussian_measures}

\subsection{Real Gaussian measures}

\begin{definition}[Real Gaussian measure]\label{def:gaussianReal}
  \mathlibok
  \lean{ProbabilityTheory.gaussianReal}
  The real Gaussian measure with mean $\mu \in \mathbb{R}$ and variance $\sigma^2 > 0$ is the measure on $\mathbb{R}$ with density $\frac{1}{\sqrt{2 \pi \sigma^2}} \exp\left(-\frac{(x - \mu)^2}{2 \sigma^2}\right)$ with respect to the Lebesgue measure.
  The real Gaussian measure with mean $\mu \in \mathbb{R}$ and variance $0$ is the Dirac measure $\delta_\mu$.
  We denote this measure by $\mathcal{N}(\mu, \sigma^2)$.
\end{definition}


\begin{lemma}\label{lem:charFun_gaussianReal}
  \uses{def:gaussianReal, def:charFun}
  \mathlibok
  \lean{ProbabilityTheory.charFun_gaussianReal}
The characteristic function of a real Gaussian measure with mean $\mu$ and variance $\sigma^2$ is given by
$x \mapsto \exp\left(i \mu x - \frac{\sigma^2 x^2}{2}\right)$.
\end{lemma}

\begin{proof}\leanok

\end{proof}


\begin{lemma}\label{lem:centralMoment_two_mul_gaussianReal}
  \uses{def:gaussianReal}
  \leanok
  \lean{ProbabilityTheory.centralMoment_two_mul_gaussianReal}
The central moment of order $2n$ of a real Gaussian measure $\mathcal{N}(\mu, \sigma^2)$ is given by
\begin{align*}
  \mathbb{E}[(X - \mu)^{2n}] = \sigma^{2n} (2n - 1)!! \: ,
\end{align*}
in which $(2n - 1)!! = (2n - 1)(2n - 3) \cdots 3 \cdot 1$ is the double factorial of $2n - 1$.
\end{lemma}

\begin{proof}\leanok
\begin{align*}
	\mathbb{E}[(X - \mu)^{2n}] &= \int_{-\infty}^\infty (x - \mu)^{2n} \frac{1}{\sqrt{2 \pi \sigma^2}} e^{-\frac{(x - \mu)^2}{2 \sigma^2}} \mathrm dx \\
	&= \int_{-\infty}^\infty x^{2n} \frac{1}{\sqrt{2 \pi \sigma^2}} e^{-\frac{x^2}{2 \sigma^2}} \mathrm dx \\
	&= 2 \int_{0}^\infty x^{2n} \frac{1}{\sqrt{2 \pi \sigma^2}} e^{-\frac{x^2}{2 \sigma^2}} \mathrm dx \\
	&= 2 \int_{0}^\infty {\sqrt{2 \sigma^2 x}}^{2n} \frac{1}{\sqrt{2 \pi \sigma^2}} e^{-x)} \frac{\sigma^2}{\sqrt{2 \sigma^2 x'}} \mathrm dx \\
	&= \frac{\sigma^{2n} 2^n}{\sqrt{\pi}} \int_{0}^\infty x^{n - 1/2} e{-x} \mathrm dx \\
	&= \frac{\sigma^{2n} 2^n}{\sqrt{\pi}} \Gamma(n + 1/2) \\
	&= \frac{\sigma^{2n} 2^n}{\Gamma(1/2)} \left( \prod_{k=0}^{n-1} (k + 1/2) \right) \Gamma(1/2) \\
	&= \sigma^{2n} \prod_{k=0}^{n-1} (2k + 1) \\
	&= \sigma^{2n} (2n - 1)!!
\end{align*}
\end{proof}


\subsection{Gaussian measures on a Banach space}

That kind of generality is not needed for this project, but we happen to have results about Gaussian measures on a Banach space in Mathlib, so we will use them.
The main reference for this section is \cite{hairer2009introduction}.

Let $F$ be a separable Banach space.

\begin{definition}[Gaussian measure]\label{def:IsGaussian}
  \uses{def:gaussianReal}
  \mathlibok
  \lean{ProbabilityTheory.IsGaussian}
A measure $\mu$ on $F$ is Gaussian if for every continuous linear form $L \in F^*$, the pushforward measure $L_* \mu$ is a Gaussian measure on $\mathbb{R}$.
\end{definition}


\begin{lemma}\label{lem:IsGaussian.IsProbabilityMeasure}
  \uses{def:IsGaussian}
  \mathlibok
A Gaussian measure is a probability measure.
\end{lemma}

\begin{proof}\leanok

\end{proof}


\begin{theorem}\label{thm:isGaussian_iff_charFunDual_eq}
  \uses{def:IsGaussian, def:charFunDual}
  \mathlibok
  \lean{ProbabilityTheory.isGaussian_iff_charFunDual_eq}
A finite measure $\mu$ on $F$ is Gaussian if and only if for every continuous linear form $L \in F^*$, the characteristic function of $\mu$ at $L$ is
\begin{align*}
  \hat{\mu}(L) = \exp\left(i \mu[L] - \mathbb{V}_\mu[L] / 2\right) \: ,
\end{align*}
in which $\mathbb{V}_\mu[L]$ is the variance of $L$ with respect to $\mu$.
\end{theorem}

\begin{proof}\uses{thm:ext_of_charFunDual, lem:charFun_gaussianReal}\leanok

\end{proof}



\paragraph{Transformations of Gaussian measures}

\begin{lemma}\label{lem:isGaussian_map}
  \uses{def:IsGaussian}
  \mathlibok
  \lean{ProbabilityTheory.isGaussian_map}
Let $F, G$ be two Banach spaces, let $\mu$ be a Gaussian measure on $F$ and let $T : F \to G$ be a continuous linear map.
Then $T_*\mu$ is a Gaussian measure on $G$.
\end{lemma}

\begin{proof}\leanok

\end{proof}


\begin{lemma}\label{lem:isGaussian_add_const}
  \uses{def:IsGaussian}
  \leanok
  % This is an instance without name in the code, hence we don't give a \lean{...}.
Let $\mu$ be a Gaussian measure on $F$ and let $c \in F$.
Then the measure $\mu$ translated by $c$ (the map of $\mu$ by $x \mapsto x + c$) is a Gaussian measure on $F$.
\end{lemma}

\begin{proof}\leanok

\end{proof}


\begin{lemma}\label{lem:isGaussian_conv}
  \uses{def:IsGaussian}
  \mathlibok
  %\lean{ProbabilityTheory.isGaussian_conv} -- need a Mathlib update
The convolution of two Gaussian measures is a Gaussian measure.
\end{lemma}

\begin{proof}\leanok

\end{proof}



\paragraph{Fernique's theorem}


\begin{theorem}\label{thm:exists_integrable_exp_sq_of_map_rotation_eq_self}
  \leanok
  % In a Mathlib PR
Let $\mu$ be a finite measure on $F$ such that $\mu \times \mu$ is invariant under the rotation of angle $-\frac{\pi}{4}$.
Then there exists $C > 0$ such that the function $x \mapsto \exp (C \Vert x \Vert ^ 2)$ is integrable with respect to $\mu$.
\end{theorem}

\begin{proof}\leanok

\end{proof}


\begin{lemma}\label{lem:IsGaussian.map_rotation_eq_self}
  \uses{def:IsGaussian}
  \leanok
  % In a Mathlib PR
For a Gaussian measure $\mu$, $\mu \times \mu$ is invariant by rotation.
\end{lemma}

\begin{proof}\leanok
  \uses{lem:isGaussian_conv}

\end{proof}


\begin{theorem}[Fernique's theorem]\label{thm:IsGaussian.exists_integrable_exp_sq}
  \uses{def:IsGaussian}
  \leanok
  \lean{ProbabilityTheory.IsGaussian.exists_integrable_exp_sq}
For a Gaussian measure, there exists $C > 0$ such that the function $x \mapsto \exp (C \Vert x \Vert ^ 2)$ is integrable.
\end{theorem}

\begin{proof}\leanok
  \uses{thm:isGaussian_iff_charFunDual_eq, lem:IsGaussian.IsProbabilityMeasure, thm:exists_integrable_exp_sq_of_map_rotation_eq_self, lem:IsGaussian.map_rotation_eq_self}

\end{proof}


\begin{lemma}\label{lem:IsGaussian.memLp_id}
  \uses{def:IsGaussian}
  \leanok
  \lean{ProbabilityTheory.IsGaussian.memLp_id}
A Gaussian measure $\mu$ has finite moments of all orders.
In particular, there is a well defined mean $m_\mu := \mu[\mathrm{id}]$, and for all $L \in F^*$, $\mu[L] = L(m_\mu)$.
\end{lemma}

\begin{proof}\leanok
  \uses{thm:IsGaussian.exists_integrable_exp_sq}

\end{proof}

A Gaussian measure has finite second moment by Lemma~\ref{lem:IsGaussian.memLp_id}, hence its covariance bilinear form is well defined.


\subsection{Gaussian measures on a finite dimensional Hilbert space}

We specialize directly from Banach space to finite dimensional Hilbert space since that's what we need in this project, although there are results for Gaussian measures on infinite dimensional Hilbert spaces that would worth stating.

\begin{lemma}\label{lem:isGaussian_iff_charFun_eq}
  \uses{def:IsGaussian, def:charFunDual, def:charFun}
  \leanok
  \lean{ProbabilityTheory.isGaussian_iff_charFun_eq}
A finite measure $\mu$ on a Hilbert space $E$ is Gaussian if and only if for every $t \in E$, the characteristic function of $\mu$ at $t$ is
\begin{align*}
  \hat{\mu}(t) =  \exp\left(i \mu[\langle t, \cdot \rangle] - \mathbb{V}_\mu[\langle t, \cdot \rangle] / 2\right) \: .
\end{align*}
\end{lemma}

\begin{proof}\leanok
  \uses{thm:isGaussian_iff_charFunDual_eq}
By Theorem~\ref{thm:isGaussian_iff_charFunDual_eq}, $\mu$ is Gaussian iff for every continuous linear form $L \in E^*$, the characteristic function of $\mu$ at $L$ is
\begin{align*}
  \hat{\mu}(L) = \exp\left(i \mu[L] - \mathbb{V}_\mu[L] / 2\right) \: .
\end{align*}
Every continuous linear form $L \in E^*$ can be written as $L(x) = \langle t, x \rangle$ for some $t \in E$, hence we have that $\mu$ is Gaussian iff for every $t \in E$,
\begin{align*}
  \hat{\mu}(t) = \exp\left(i \mu[\langle t, \cdot \rangle] - \mathbb{V}_\mu[\langle t, \cdot \rangle] / 2\right) \: .
\end{align*}
\end{proof}

Let $E$ be a separable Hilbert space. We denote by $\langle \cdot, \cdot \rangle$ the inner product on $E$ and by $\Vert \cdot \Vert$ the associated norm.

\begin{lemma}\label{lem:IsGaussian.charFun_eq}
  \uses{def:IsGaussian, def:charFun, def:covInnerBilin}
  \leanok
  \lean{ProbabilityTheory.IsGaussian.charFun_eq}
The characteristic function of a Gaussian measure $\mu$ on $E$ is given by
\begin{align*}
  \hat{\mu}(t) = \exp\left(i \langle t, m_\mu \rangle - \frac{1}{2} C'_\mu(t, t)\right) \: .
\end{align*}
\end{lemma}

\begin{proof}\leanok
  \uses{lem:isGaussian_iff_charFun_eq, lem:IsGaussian.memLp_id, lem:covarianceBilin_same_eq_variance}
By Lemma~\ref{lem:isGaussian_iff_charFun_eq}, for every $t \in E$,
\begin{align*}
  \hat{\mu}(t) = \exp\left(i \mu[\langle t, \cdot \rangle] - \mathbb{V}_\mu[\langle t, \cdot \rangle] / 2\right) \: .
\end{align*}
By Lemma~\ref{lem:IsGaussian.memLp_id}, $\mu$ has finite first moment and $\mu[\langle t, \cdot \rangle] = \langle t, m_\mu \rangle$. By the same lemma, $\mu$ has finite second moment and for any $t$ we have $\mathbb{V}_\mu[\langle t, \cdot\rangle] = C'_\mu(t, t)$.
\end{proof}

\begin{lemma}\label{lem:isGaussian_iff_gaussian_charFun}
  \uses{def:IsGaussian, def:charFun, def:covMatrix}
  \leanok
  \lean{ProbabilityTheory.isGaussian_iff_gaussian_charFun, ProbabilityTheory.gaussian_charFun_congr}
A finite measure $\mu$ on $E$ is Gaussian if and only if there exists $m \in E$ and $C$ positive semidefinite such that for all $t \in E$, the characteristic function of $\mu$ at $t$ is
\begin{align*}
  \hat{\mu}(t) = \exp\left(i \langle t, m \rangle - \frac{1}{2} C(t, t)\right) \: ,
\end{align*}
If that's the case, then $m = m_\mu$ and $C = C'_\mu$.
\end{lemma}

Note that this lemma does not say that there exists a Gaussian measure for any such $m$ and $C$.
We will prove that later.

\begin{proof}\leanok
  \uses{lem:IsGaussian.charFun_eq, lem:charFun_map_eq_charFunDual_smul, thm:ext_of_charFun}
Lemma~\ref{lem:IsGaussian.charFun_eq} states that the characteristic function of a Gaussian measure has the wanted form.

Suppose now that there exists $m \in E$ and $C$ positive semidefinite such that for all $t \in E$, $\hat{\mu}(t) = \exp\left(i \langle t, m \rangle - \frac{1}{2} C(t, t)\right)$.

We need to show that for all $L \in E^*$, $L_*\mu$ is a Gaussian measure on $\mathbb{R}$.
Such an $L$ can be written as $\langle u, \cdot \rangle$ for some $u \in E$.
Let then $u \in E$. We compute the characteristic function of $\langle u, \cdot\rangle_*\mu$ at $x \in \mathbb{R}$ with Lemma~\ref{lem:charFun_map_eq_charFunDual_smul}:
\begin{align*}
  \widehat{\langle u, \cdot\rangle_*\mu}(x)
  &= \hat{\mu}(x \cdot u)
  \\
  &= \exp\left(i x \langle u, m \rangle - \frac{1}{2} x^2 C(u, u)\right)
  \: .
\end{align*}
This is the characteristic function of a Gaussian measure on $\mathbb{R}$ with mean $\langle u, m \rangle$ and variance $C(u, u)$.
By Theorem~\ref{thm:ext_of_charFun}, $\langle u, \cdot\rangle_*\mu$ is Gaussian, hence $\mu$ is Gaussian.

By Lemma~\ref{lem:IsGaussian.charFun_eq}, we deduce that for any $t \in E$ we have
$$\exp\left(i\langle t, m \rangle - \frac{1}{2} C(t, t)\right) = \exp\left(i\langle t, m_\mu \rangle - \frac{1}{2} C'_\mu(t, t)\right).$$
In particular, for any $t$ there exists $n_t \in \mathbb{Z}$ such that
$$i\langle t, m \rangle - \frac{1}{2} C(t, t) = i\langle t, m_\mu \rangle - \frac{1}{2} C'_\mu(t, t) + 2i\pi n_t.$$
We deduce that $n$ is a continuous map from $E$ to $\mathbb{Z}$, and thus must be constant because $E$ is connected. By looking at the value at $t = 0$, we deduce that for any $t$, $n_t = 0$. Looking at real and imaginary parts we obtain that for any $t$,
$$\langle t, m \rangle = \langle t, m_\mu \rangle \quad \text{and} \quad C(t, t) = C'_\mu(t, t).$$
We immediately deduce that $m = m_\mu$. Moreover, because $C$ and $C'_\mu$ are symmetric, they are characterized by their values on the diagonal. Indeed, for any $x, y$,
$$C(x, y) = \frac{1}{2} (C(x + y, x + y) - C(x, x) - C(y, y)).$$
We deduce that $C = C'_\mu$.
\end{proof}

\begin{lemma}\label{lem:IsGaussian.ext_iff}
  \uses{def:IsGaussian, def:covInnerBilin}
  \leanok
  \lean{ProbabilityTheory.IsGaussian.ext, ProbabilityTheory.IsGaussian.ext_iff}
Two Gaussian measures $\mu$ and $\nu$ on a separable Hilbert space are equal if and only if they have same mean and same covariance.
\end{lemma}

\begin{proof}\leanok
  \uses{thm:ext_of_charFun, lem:IsGaussian.charFun_eq}
The forward direction is immediate.

For the converse direction, it is enough to show that $\mu$ and $\nu$ have the same characteristic function by Theorem~\ref{thm:ext_of_charFun}. As they are both Gaussian, their characteristic functions only depend on their mean and covariance by Lemma~\ref{lem:IsGaussian.charFun_eq}. Thus they are equal.
\end{proof}


\begin{definition}[Standard Gaussian measure]\label{def:stdGaussian}
  \uses{def:gaussianReal}
  \leanok
  \lean{ProbabilityTheory.stdGaussian}
Let $(e_1, \ldots, e_d)$ be an orthonormal basis of $E$ and let $\mu$ be the standard Gaussian measure on $\mathbb{R}$.
The standard Gaussian measure on $E$ is the pushforward measure of the product measure $\mu \times \ldots \times \mu$ by the map $x \mapsto \sum_{i=1}^d x_i \cdot e_i$.
\end{definition}

The fact that this definition does not depend on the choice of basis will be a consequence of the fact that its characteristic function does not depend on the basis.


\begin{lemma}\label{lem:integral_eval_pi}
  \leanok
  \lean{ProbabilityTheory.integral_eval_pi}
For $\mu_1, \ldots, \mu_d$ probability measures on $\mathbb{R}$ and $f : \mathbb{R} \to \mathbb{R}$ integrable with respect to $\mu_i$, we have
\begin{align*}
  \int_x f(x_i) \, d(\mu_1 \times \ldots \times \mu_d)(x)
  = \int_x f(x) \, d\mu_i
  \: .
\end{align*}
\end{lemma}

\begin{proof}\leanok
As $f$ is integrable, we can use Fubini theorem to obtain that
$$\int f(x_i) \, d(\mu_1 \times \ldots \times \mu_d)(x) = \int f(x) \, d\mu_i(x) \times \prod_{j \ne i} \int 1 \, d\mu_j(x) = \int f(x) \, d\mu_i(x)$$
because the $\mu_j$s are probability measures.
\end{proof}


\begin{lemma}\label{lem:isCentered_stdGaussian}
  \uses{def:stdGaussian}
  \leanok
  \lean{ProbabilityTheory.isCentered_stdGaussian}
The standard Gaussian measure on $E$ is centered, i.e., $\mu[L] = 0$ for every $L \in E^*$.
\end{lemma}

\begin{proof}\leanok
  \uses{lem:integral_eval_pi}

\end{proof}


\begin{lemma}\label{lem:isProbabilityMeasure_stdGaussian}
  \uses{def:stdGaussian}
  \leanok
  \lean{ProbabilityTheory.isProbabilityMeasure_stdGaussian}
The standard Gaussian measure is a probability measure.
\end{lemma}

\begin{proof}\leanok

\end{proof}


\begin{lemma}\label{lem:charFun_stdGaussian}
  \uses{def:stdGaussian, def:charFun}
  \leanok
  \lean{ProbabilityTheory.charFun_stdGaussian}
The characteristic function of the standard Gaussian measure on $E$ is given by
\begin{align*}
  \hat{\mu}(t) = \exp\left(-\frac{1}{2} \Vert t \Vert^2 \right) \: .
\end{align*}
\end{lemma}

\begin{proof}\leanok
  \uses{lem:charFun_gaussianReal}
Denote by $\nu$ the standard Gaussian measure on $\mathbb{R}$. This is a straightforward computation:
\begin{align*}
  \hat{\mu}(t) = \int \exp\left(i\langle t, \sum_{j=1}^d x_j \cdot e_j \rangle\right) d(\nu \times \ldots \times \nu)(dx) &= \int \exp\left(\sum_{j=1}^d ix_j\langle t, e_j \rangle\right) d(\nu \times \ldots \times \nu)(dx) \\
  &= \int \prod_{j=1}^d \exp\left(ix_j\langle t, e_j \rangle\right) d(\nu \times \ldots \times \nu)(dx) \\
  &= \prod_{j=1}^d \int \exp\left(ix\langle t, e_j \rangle\right) d\nu(x) \\
  &= \prod_{j=1}^d \exp\left(-\frac{\langle t, e_j \rangle^2}{2}\right) \\
  &= \exp\left(-\frac{1}{2} \Vert t \Vert^2 \right).
\end{align*}
\end{proof}


\begin{lemma}\label{lem:isGaussian_stdGaussian}
  \uses{def:stdGaussian, def:IsGaussian}
  \leanok
  \lean{ProbabilityTheory.isGaussian_stdGaussian}
The standard Gaussian measure on $E$ is a Gaussian measure.
\end{lemma}

\begin{proof}\leanok
  \uses{lem:isGaussian_iff_gaussian_charFun, lem:charFun_stdGaussian, lem:isProbabilityMeasure_stdGaussian}
Since the standard Gaussian is a probability measure (hence finite), we can apply Lemma~\ref{lem:isGaussian_iff_gaussian_charFun} that states that it suffices to show that the characteristic function has a particular form.
That form is given by Lemma~\ref{lem:charFun_stdGaussian}, taking $m=0$ and $C = \langle\cdot, \cdot\rangle$.
\end{proof}


\begin{lemma}\label{lem:integral_id_stdGaussian}
  \uses{def:stdGaussian}
  \leanok
  \lean{ProbabilityTheory.integral_id_stdGaussian}
The mean of the standard Gaussian measure is $0$.
\end{lemma}

\begin{proof}\leanok
  \uses{lem:integral_eval_pi}

\end{proof}


\begin{lemma}\label{lem:covMatrix_stdGaussian}
  \uses{def:stdGaussian, def:covMatrix}
  \leanok
  \lean{ProbabilityTheory.covMatrix_stdGaussian}
The covariance matrix of the standard Gaussian measure is the identity matrix.
\end{lemma}

\begin{proof}\leanok
  \uses{lem:isGaussian_iff_gaussian_charFun, lem:charFun_stdGaussian}
From Lemma~\ref{lem:charFun_stdGaussian}, we know that for all $t \in \mathbb{R}$,
$$\hat{\mu}(t) = \exp\left(-\frac{\|t\|^2}{2}\right) = \exp\left(-\frac{\langle t, \mathrm{I}t\rangle}{2}\right).$$
As the identity is positive semidefinite, we deduce from Lemma~\ref{lem:isGaussian_iff_gaussian_charFun} that $\Sigma_\mu$ is the identity matrix.
\end{proof}


\begin{definition}[Multivariate Gaussian]\label{def:multivariateGaussian}
  \uses{def:stdGaussian}
  \leanok
  \lean{ProbabilityTheory.multivariateGaussian}
The multivariate Gaussian measure on $\mathbb{R}^d$ with mean $m \in \mathbb{R}^d$ and covariance matrix $\Sigma \in \mathbb{R}^{d \times d}$, with $\Sigma$ positive semidefinite, is the pushforward measure of the standard Gaussian measure on $\mathbb{R}^d$ by the map $x \mapsto m + \Sigma^{1/2} x$.
We denote this measure by $\mathcal{N}(m, \Sigma)$.
\end{definition}


\begin{lemma}\label{lem:integral_id_multivariateGaussian}
  \uses{def:multivariateGaussian}
  \leanok
  \lean{ProbabilityTheory.integral_id_multivariateGaussian}
The mean of the multivariate Gaussian measure $\mathcal{N}(m, \Sigma)$ is $m$.
\end{lemma}

\begin{proof}\leanok
  \uses{lem:integral_id_stdGaussian}

\end{proof}


\begin{lemma}\label{lem:covMatrix_multivariateGaussian}
  \uses{def:multivariateGaussian}
  \leanok
  \lean{ProbabilityTheory.covInnerBilin_multivariateGaussian}
The covariance matrix of the multivariate Gaussian measure $\mathcal{N}(m, \Sigma)$ is $\Sigma$.
\end{lemma}

\begin{proof}\leanok
  \uses{lem:covMatrix_stdGaussian}

\end{proof}


\begin{lemma}\label{lem:isGaussian_multivariateGaussian}
  \uses{def:multivariateGaussian, def:IsGaussian}
  \leanok
  \lean{ProbabilityTheory.isGaussian_multivariateGaussian}
A multivariate Gaussian measure is a Gaussian measure.
\end{lemma}

\begin{proof}\leanok
  \uses{lem:isGaussian_stdGaussian, lem:isGaussian_add_const, lem:isGaussian_map}
The multivariate Gaussian measure is the pushforward of the standard Gaussian measure by an affine map, and is thus Gaussian by Lemma~\ref{lem:isGaussian_add_const} and Lemma~\ref{lem:isGaussian_map}.
\end{proof}


\begin{theorem}\label{thm:charFun_multivariateGaussian}
  \uses{def:multivariateGaussian, def:charFun}
  \leanok
  \lean{ProbabilityTheory.charFun_multivariateGaussian}
The characteristic function of a multivariate Gaussian measure $\mathcal{N}(m, \Sigma)$ is given by
\begin{align*}
  \hat{\mu}(t) = \exp\left(i \langle m, t \rangle - \frac{1}{2} \langle t, \Sigma t \rangle\right)
  \: .
\end{align*}
\end{theorem}

\begin{proof}\leanok
  \uses{lem:isGaussian_multivariateGaussian, lem:IsGaussian.charFun_eq, lem:integral_id_multivariateGaussian, lem:covMatrix_multivariateGaussian}
Since the multivariate Gaussian measure is a Gaussian measure, we can apply Lemma~\ref{lem:IsGaussian.charFun_eq} to it.
It suffices then to show that the mean and the covariance matrix of the multivariate Gaussian measure are equal to $m$ and $\Sigma$, respectively.
This is given by Lemma~\ref{lem:integral_id_multivariateGaussian} and Lemma~\ref{lem:covMatrix_multivariateGaussian}.
\end{proof}


\section{Gaussian processes}
\label{sec:gaussian_processes}

\begin{definition}[Gaussian process]\label{def:IsGaussianProcess}
  \uses{def:IsGaussian}
  \leanok
  \lean{ProbabilityTheory.IsGaussianProcess}
A process $X : T \to \Omega \to E$ is Gaussian if for every finite subset $t_1, \ldots, t_n \in T$, the random vector $(X_{t_1}, \ldots, X_{t_n})$ has a Gaussian distribution.
\end{definition}


\begin{lemma}\label{lem:isGaussianProcess_of_modification}
  \uses{def:IsGaussianProcess}
  \leanok
  \lean{ProbabilityTheory.IsGaussianProcess.modification}
Let $X, Y : T \to \Omega \to E$ be two stochastic processes that are modifications of each other (that is, for all $t \in T$, $X_t =_{a.e.} Y_t$).
If $X$ is a Gaussian process, then $Y$ is a Gaussian process as well.
\end{lemma}

\begin{proof}
  \uses{lem:map_eq_of_modification}
Being a Gaussian process is defined in terms of the distribution of finite-dimensional random vectors.
By Lemma~\ref{lem:map_eq_of_modification}, the random vector $(Y_{t_1}, \ldots, Y_{t_n})$ has the same distribution as the random vector $(X_{t_1}, \ldots, X_{t_n})$ for all $t_1, \ldots, t_n \in T$.
\end{proof}

\chapter{Projective family of the Brownian motion}
\label{chap:projective_family}


\section{Kolmogorov extension theorem}

This theorem has been formalized in the repository \href{https://github.com/RemyDegenne/kolmogorov_extension4}{kolmogorov\_extension4}.

\begin{definition}[Projective family]\label{def:IsProjectiveMeasureFamily}
  \mathlibok
  \lean{MeasureTheory.IsProjectiveMeasureFamily}
A family of measures $P$ indexed by finite sets of $T$ is projective if, for finite sets $J \subseteq I$, the projection from $E^I$ to $E^J$ maps $P_I$ to $P_J$.
\end{definition}


\begin{definition}[Projective limit]\label{def:IsProjectiveLimit}
  \uses{def:IsProjectiveMeasureFamily}
  \mathlibok
  \lean{MeasureTheory.IsProjectiveLimit}
A measure $\mu$ on $E^T$ is the projective limit of a projective family of measures $P$ indexed by finite sets of $T$ if, for every finite set $I \subseteq T$, the projection from $E^T$ to $E^I$ maps $\mu$ to $P_I$.
\end{definition}


\begin{theorem}[Kolmogorov extension theorem]\label{thm:kolmogorovExtension}
  \uses{def:IsProjectiveLimit, def:IsProjectiveMeasureFamily}
  \leanok
  \lean{MeasureTheory.projectiveLimit, MeasureTheory.IsProjectiveLimit.unique, MeasureTheory.isProjectiveLimit_projectiveLimit, MeasureTheory.isFiniteMeasure_projectiveLimit, MeasureTheory.isProbabilityMeasure_projectiveLimit}
Let $\mathcal{X}$ be a Polish space, equipped with the Borel $\sigma$-algebra, and let $T$ be an index set.
Let $P$ be a projective family of finite measures on $\mathcal{X}$.
Then the projective limit $\mu$ of $P$ exists, is unique, and is a finite measure on $\mathcal{X}^T$.
Moreover, if $P_I$ is a probability measure for every finite set $I \subseteq T$, then $\mu$ is a probability measure.
\end{theorem}

\begin{proof}\leanok

\end{proof}


\section{Projective family of Gaussian measures}

We build a projective family of Gaussian measures indexed by $\mathbb{R}_+$.
In order to do so, we need to define specific Gaussian measures on finite index sets $\{t_1, \ldots, t_n\}$.
We want to build a multivariate Gaussian measure on $\mathbb{R}^n$ with mean $0$ and covariance matrix $C_{ij} = \min(t_i, t_j)$ for $1 \leq i,j \leq n$.

% \paragraph{First method: Gaussian increments}

% In this method, we build the Gaussian measure by adding independent Gaussian increments.

% \begin{definition}(Gaussian increment)\label{def:gaussianIncrement}
% For $v \ge 0$, the map from $\mathbb{R}$ to the probability measures on $\mathbb{R}$ defined by $x \mapsto \mathcal{N}(x, v)$ is a Markov kernel.
% We call that kernel the \emph{Gaussian increment} with variance $v$ and denote it by $\kappa^G_v$.
% \end{definition}

% TODO: perhaps the equality $\mathcal{N}(x, v) = \delta_x \ast \mathcal{N}(0, v)$ is useful to show that it is a kernel?

% \begin{definition}\label{def:gaussianFromIncrements}
%   \uses{def:gaussianIncrement}
% Let $0 \le t_1 \le \ldots \le t_n$ be non-negative reals.
% Let $\mu_0$ be the real Gaussian distribution $\mathcal{N}(0, t_1)$.
% For $i \in \{1, \ldots, n-1\}$, let $\kappa_i$ be the Markov kernel from $\mathbb{R}$ to $\mathbb{R}$ defined by $\kappa_i(x) = \mathcal{N}(x, t_{i+1} - t_i)$ (the Gaussian increment $\kappa^G_{t_{i+1} - t_i}$).
% Let $P_{t_1, \ldots, t_n}$ be the measure on $\mathbb{R}^n$ defined by $\mu_0 \otimes \kappa_1 \otimes \ldots \otimes \kappa_{n-1}$.
% \end{definition}

% TODO: explain the notation $\otimes$ in the lemma above: $\kappa_{n-1}$ takes the value at $n-1$ only to produce the distribution at $n$.

% \begin{lemma}\label{lem:isGaussian_gaussianFromIncrements}
%   \uses{def:gaussianFromIncrements, def:IsGaussian}
% $P_{t_1, \ldots, t_n}$ is a Gaussian measure on $\mathbb{R}^n$ with mean $0$ and covariance matrix $C_{ij} = \min(t_i, t_j)$ for $1 \leq i,j \leq n$.
% \end{lemma}

% \begin{proof}

% \end{proof}


% \paragraph{Second method: covariance matrix}

We prove that the matrix $C_{ij} = \min(t_i, t_j)$ is positive semidefinite, which means that there exists a Gaussian distribution with mean 0 and covariance matrix $C$.

\begin{definition}[Gram matrix]\label{def:gramMatrix}
Let $v_1, \ldots, v_n$ be vectors in an inner product space $E$.
The Gram matrix of $v_1, \ldots, v_n$ is the matrix in $\mathbb{R}^{n \times n}$ with entries $G_{ij} = \langle v_i, v_j \rangle$ for $1 \leq i,j \leq n$.
\end{definition}


\begin{lemma}\label{lem:posSemidef_gramMatrix}
  \uses{def:gramMatrix}
A gram matrix is positive semidefinite.
\end{lemma}

\begin{proof}
Symmetry is obvious from the definition.
Let $x \in E$. Then
\begin{align*}
  \langle x, G x \rangle
  &= \sum_{i,j} x_i x_j \langle v_i, v_j \rangle
  \\
  &= \langle \sum_i x_i v_i, \sum_j x_j v_j \rangle
  \\
  &= \left\Vert \sum_i x_i v_i \right\Vert^2
  \\
  &\ge 0
  \: .
\end{align*}
\end{proof}


\begin{lemma}\label{lem:C_eq_gramMatrix}
  \uses{def:gramMatrix}
Let $I = \{t_1, \ldots, t_n\}$ be a finite subset of $\mathbb{R}_+$.
For $i \le n$, let $v_i = \mathbb{I}_{[0, t_i]}$ be the indicator function of the interval $[0, t_i]$, as an element of $L^2(\mathbb{R})$.
Then the Gram matrix of $v_1, \ldots, v_n$ is equal to the matrix $C_{ij} = \min(t_i, t_j)$ for $1 \leq i,j \leq n$.
\end{lemma}

\begin{proof}
By definition of the inner product in $L^2(\mathbb{R})$,
\begin{align*}
  \langle v_i, v_j \rangle
  &= \int_{\mathbb{R}} \mathbb{I}_{[0, t_i]}(x) \mathbb{I}_{[0, t_j]}(x) \: dx
  = \min(t_i, t_j)
  \: .
\end{align*}
\end{proof}


\begin{lemma}\label{lem:posSemidef_brownianCov}
For $I = \{t_1, \ldots, t_n\}$ a finite subset of $\mathbb{R}_+$, let $C \in \mathbb{R}^{n \times n}$ be the matrix $C_{ij} = \min(t_i, t_j)$ for $1 \leq i,j \leq n$.
Then $C$ is positive semidefinite.
\end{lemma}

\begin{proof}\uses{lem:C_eq_gramMatrix, lem:posSemidef_gramMatrix}
$C$ is a Gram matrix by Lemma~\ref{lem:C_eq_gramMatrix}.
By Lemma~\ref{lem:posSemidef_gramMatrix}, it is positive semidefinite.
\end{proof}


\paragraph{Definition of the projective family and extension}

\begin{definition}[Projective family of the Brownian motion]\label{def:gaussianProjectiveFamily}
  \uses{def:multivariateGaussian, lem:posSemidef_brownianCov}
For $I = \{t_1, \ldots, t_n\}$ a finite subset of $\mathbb{R}_+$, let $P^B_I$ be the multivariate Gaussian measure on $\mathbb{R}^n$ with mean $0$ and covariance matrix $C_{ij} = \min(t_i, t_j)$ for $1 \leq i,j \leq n$.
We call the family of measures $P^B_I$ the \emph{projective family of the Brownian motion}.
\end{definition}


\begin{lemma}\label{lem:isProjectiveMeasureFamily_gaussianProjectiveFamily}
  \uses{def:gaussianProjectiveFamily, def:IsProjectiveMeasureFamily}
The projective family of the Brownian motion is a projective family of measures.
\end{lemma}

\begin{proof}
  \uses{lem:isGaussian_map, lem:isGaussian_multivariateGaussian, lem:covMatrix_map}
Let $J \subseteq I$ be finite subsets of $\mathbb{R}_+$.
We need to show that the restriction from $\mathbb{R}^I$ to $\mathbb{R}^J$ (denote it by $\pi_{IJ}$) maps $P^B_I$ to $P^B_J$.

The restriction is a continuous linear map from $\mathbb{R}^I$ to $\mathbb{R}^J$.
The map of a Gaussian measure by a continuous linear map is Gaussian (Lemma~\ref{lem:isGaussian_map}).
It thus suffices to show that the mean and covariance matrix of the map are equal to the ones of $P^B_J$.

The mean of the map is $0$, since the mean of $P^B_I$ is $0$ and the map is linear.

For the covariance matrix and $i, j \in J$, by Lemma~\ref{lem:covMatrix_map} we have
\begin{align*}
  \langle e_i, \Sigma_{\pi_{IJ*}\mu} e_j\rangle
  &= \langle \pi_{IJ}^\dagger(e_i), \Sigma_\mu \pi_{IJ}^\dagger(e_j)\rangle
  \: .
\end{align*}
$\pi_{IJ}^\dagger(u)$ is the vector of $\mathbb{R}^I$ with coordinates $(\pi_{IJ}^\dagger(u))_i = u_i$ if $i \in J$ and $(\pi_{IJ}^\dagger(u))_i = 0$ otherwise.
This gives the same covariance matrix as the one of $P^B_J$.
\end{proof}


\begin{definition}\label{def:gaussianLimit}
  \uses{thm:kolmogorovExtension, lem:isProjectiveMeasureFamily_gaussianProjectiveFamily}
We denote by TODO the projective limit of the projective family of the Brownian motion given by Theorem~\ref{thm:kolmogorovExtension}.
This is a probability measure on $\mathbb{R}^{\mathbb{R}_+}$.
\end{definition}


% \begin{definition}\label{def:}
% Let $\Omega = \mathbb{R}^{\mathbb{R}_+}$ and consider the probability space $(\Omega, TODO)$.
% The identity on that space is a function $\Omega \to \mathbb{R}_+ \to \mathbb{R}$.
% We can reorder the arguments to define a process $X : \mathbb{R}_+ \to \Omega \to \mathbb{R}$.
% That process is a Gaussian process with covariance function $C(t,s) = \min(t,s)$.
% \end{definition}

\chapter{Kolmogorov-Chentsov Theorem}
\label{chap:kolmogorov_chentsov}

\section{Covers}

\begin{definition}[$\varepsilon$-cover]\label{def:IsCover}
  \leanok
  \lean{IsCover}
  Let $E$ be a set with a distance function $d_E$. Let $\varepsilon \ge 0$.
  A set $C \subseteq E$ is an $\varepsilon$-cover of a set $A \subseteq E$ if for every $x \in A$, there exists $y \in C$ such that $d_E(x, y) < \varepsilon$.
\end{definition}

\begin{definition}[External covering number]\label{def:externalCoveringNumber}
  \uses{def:IsCover}
  \leanok
  \lean{externalCoveringNumber}
  Let $E$ be a set with a distance function $d_E$.
  The external covering number of a set $A \subseteq E$ for $\varepsilon \ge 0$ is the smallest cardinality of an $\varepsilon$-cover of $A$.
\end{definition}

\begin{definition}[Internal covering number]\label{def:internalCoveringNumber}
  \uses{def:IsCover}
  \leanok
  \lean{internalCoveringNumber}
  Let $E$ be a set with a distance function $d_E$.
  The internal covering number of a set $A \subseteq E$ for $\varepsilon \ge 0$ is the smallest cardinality of an $\varepsilon$-cover of $A$ which is a subset of $A$.
\end{definition}

\chapter{Brownian motion}
\label{chap:brownian}


\section{Stochastic process with continuous paths}

\begin{definition}[pre-Brownian process]\label{def:preBrownian}
  \uses{def:gaussianLimit}
  \leanok
  \lean{ProbabilityTheory.preBrownian}
Let $\Omega = \mathbb{R}^{\mathbb{R}_+}$ and consider the probability space $(\Omega, P_B)$ (where $P_B$ is the measure defined in Definition~\ref{def:gaussianLimit}).
The identity on that space is a function $\Omega \to \mathbb{R}_+ \to \mathbb{R}$.
We reorder the arguments to define a stochastic process $X : \mathbb{R}_+ \to \Omega \to \mathbb{R}$, which we call the pre-Brownian process.
\end{definition}


\begin{lemma}\label{lem:isGaussianProcess_preBrownian}
  \uses{def:preBrownian, def:IsGaussianProcess}
  \leanok
  \lean{ProbabilityTheory.isGaussianProcess_preBrownian}
  The pre-Brownian process $X$ of Definition~\ref{def:preBrownian} is a Gaussian process.
\end{lemma}

\begin{proof}\leanok
  \uses{lem:isGaussian_multivariateGaussian}

\end{proof}


\begin{lemma}\label{lem:map_sub_preBrownian}
  \uses{def:preBrownian}
  \leanok
  \lean{ProbabilityTheory.map_sub_preBrownian}
Let $X$ be the pre-Brownian process of Definition~\ref{def:preBrownian}.
Then, for all $s, t \in \mathbb{R}_+$, the random variable $X_t - X_s$ is a Gaussian random variable with mean $0$ and variance $|t - s|$.
\end{lemma}

\begin{proof}

\end{proof}


\begin{lemma}\label{lem:isKolmogorovProcess_preBrownian}
  \uses{def:preBrownian}
  \leanok
  \lean{ProbabilityTheory.isKolmogorovProcess_preBrownian}
The pre-Brownian process $X$ of Definition~\ref{def:preBrownian} satisfies the Kolmogorov condition for exponents $(2n,n)$ with constant $(2n - 1)!!$ for all $n \in \mathbb{N}$.
That is, for all $s, t \in \mathbb{R}_+$, we have
\begin{align*}
  \mathbb{E} \left[ |X_t - X_s|^{2n} \right] \le (2n - 1)!! |t - s|^n
  \: .
\end{align*}
\end{lemma}

\begin{proof}
  \uses{lem:centralMoment_two_mul_gaussianReal, lem:map_sub_preBrownian}
$X_t - X_s$ is a Gaussian random variable with mean $0$ and variance $|t - s|$ (Lemma~\ref{lem:map_sub_preBrownian}).
Thus, by Lemma~\ref{lem:centralMoment_two_mul_gaussianReal}, we have
\begin{align*}
  \mathbb{E} \left[ |X_t - X_s|^{2n} \right]
  = (2n - 1)!! |t - s|^n
  \: .
\end{align*}
\end{proof}


\begin{definition}[Brownian motion]\label{def:brownian}
  \uses{thm:localized_holder_modification_sup, def:preBrownian, lem:isKolmogorovProcess_preBrownian, lem:hasBoundedCoveringNumberCover_nnreal}
  \leanok
By Theorem~\ref{thm:localized_holder_modification_sup}, there exists a modification $B$ of the pre-Brownian process such that all the paths of $B$ are Hölder continuous of all orders $\gamma \in (0, 1/2)$.
We call $B$ the \emph{Brownian motion} on $\mathbb{R}_+$.
\end{definition}


\begin{lemma}\label{lem:isGaussianProcess_brownian}
  \uses{def:brownian, def:IsGaussianProcess}
  \leanok
  \lean{ProbabilityTheory.isGaussianProcess_brownian}
The Brownian motion is a Gaussian process.
\end{lemma}

\begin{proof}\leanok
  \uses{lem:isGaussianProcess_of_modification, lem:isGaussianProcess_preBrownian}
The pre-Brownian process is a Gaussian process by Lemma~\ref{lem:isGaussianProcess_preBrownian}.
The Brownian motion is a modification of the pre-Brownian process by Definition~\ref{def:brownian}.
Thus, the Brownian motion is a Gaussian process as well by Lemma~\ref{lem:isGaussianProcess_of_modification}.
\end{proof}


\begin{lemma}\label{lem:isHolderWith_brownian}
  \uses{def:brownian}
  \leanok
  \lean{ProbabilityTheory.isHolderWith_brownian}
The paths of the Brownian motion are Hölder continuous of all orders $\gamma \in (0, 1/2)$.
\end{lemma}

\begin{proof}

\end{proof}


\begin{lemma}\label{lem:continuous_brownian}
  \uses{def:brownian}
  \leanok
  \lean{ProbabilityTheory.continuous_brownian}
The paths of the Brownian motion are continuous.
\end{lemma}

\begin{proof}
  \uses{lem:isHolderWith_brownian}

\end{proof}


\begin{lemma}\label{lem:law_brownian_apply}
  \uses{def:brownian}
For $t \in \mathbb{R}_+$, the law of $B_t$ (the Brownian motion at time $t$) is the real Gaussian measure $\mathcal{N}(0,t)$.
\end{lemma}

\begin{proof}

\end{proof}

\section{Wiener measure on the continuous functions}

We want to turn the Brownian motion into a measure on the continuous functions $C(\mathbb{R}_+, \mathbb{R})$ with the Borel sigma-algebra generated by the compact-open topology.


\begin{definition}[Auxiliary Wiener measure]\label{def:wienerMeasureAux}
  \uses{def:brownian, def:gaussianLimit, lem:continuous_brownian}
  \leanok
  \lean{ProbabilityTheory.wienerMeasureAux}
The pushforward of the measure $P_B$ of Definition~\ref{def:gaussianLimit} by the Brownian motion $B$ is a measure on the continuous functions on $\mathbb{R}^{\mathbb{R}_+}$, with the sigma-algebra induced by the product sigma-algebra on $\mathbb{R}^{\mathbb{R}_+}$.
\end{definition}

\textbf{Lean remark}: the auxiliary Wiener measure is a measure on the subtype \texttt{\{f  // Continuous f\}}. This is not the same type as $C(\mathbb{R}_+, \mathbb{R})$.


\begin{theorem}\label{thm:ContinuousMap.borel_eq_iSup_comap_eval}
  \leanok
  \lean{ProbabilityTheory.ContinuousMap.borel_eq_iSup_comap_eval}
The borel sigma-algebra on $C(\mathbb{R}_+, \mathbb{R})$ coming from the compact-open topology is equal to the smallest sigma-algebra for which the evaluation maps $f \mapsto f(t)$ are measurable for every $t \in \mathbb{R}_+$.
\end{theorem}

\begin{proof}
Possible ref: \href{https://math.stackexchange.com/questions/4789531/when-does-the-borel-sigma-algebra-of-compact-convergence-coincide-with-the-pr}{stackexchange question}.
\end{proof}


\begin{definition}\label{def:MeasurableEquiv.continuousMap}
  \uses{thm:ContinuousMap.borel_eq_iSup_comap_eval}
  \leanok
  \lean{ProbabilityTheory.MeasurableEquiv.continuousMap}
The identity is a measurable equivalence between the continuous functions of $\mathbb{R}^{\mathbb{R}_+}$ with the subset sigma-algebra obtained from the product sigma-algebra, and $C(\mathbb{R}_+, \mathbb{R})$ with the Borel sigma-algebra coming from the compact-open topology.

Mathematically this says nothing more than the equality of sigma-algebras of Theorem~\ref{thm:ContinuousMap.borel_eq_iSup_comap_eval} but in Lean we have two different types so we need an equivalence.
\end{definition}


\begin{definition}[Wiener measure]\label{def:wienerMeasure}
  \uses{def:MeasurableEquiv.continuousMap, def:wienerMeasureAux}
  \leanok
  \lean{ProbabilityTheory.wienerMeasure}
The Wiener measure on $C(\mathbb{R}_+, \mathbb{R})$ with the Borel sigma-algebra is the map of the auxiliary Wiener measure by the measurable equivalence of definition~\ref{def:MeasurableEquiv.continuousMap}.
\end{definition}


TODO: add the main properties of the Brownian motion and the Wiener measure.
We need to be able to tell that we have built the correct objects.



\part{Stochastic integral}

\paragraph{Overview}

We describe the construction of a stochastic integral.

\paragraph{Status} The formalization is ongoing.

\paragraph{Formalization authors} Anyone is welcome to contribute!

\chapter{Debut Theorem}
\label{chap:debut_theorem}

\section{Monotone class theorem}

% This section may be used also in other chapters of the blueprint. Therefore, depending on where it is used and also how big the proof becomes, we may want to move it somewhere else.
TODO: find the right generality (and some reference) to state the monotone class theorem and write the informal proof. It may be possible to adapt the following theorem: \href{https://leanprover-community.github.io/mathlib4_docs/Mathlib/MeasureTheory/PiSystem.html#MeasurableSpace.DynkinSystem.generateFrom_eq}{MeasurableSpace.DynkinSystem.generateFrom\_eq}.

\begin{definition}[Monotone class]\label{def:monotone_class}
  Let $\mathcal{M}$ be a collection of subsets of a set $X$. We say that $\mathcal{M}$ is a monotone class if it is closed under countable monotone unions and countable monotone intersections, i.e.:
  \begin{enumerate}
    \item if \( A_1, A_2, \ldots \in M \) and \( A_1 \subseteq A_2 \subseteq \cdots \), then
    \( \bigcup_{i=1}^\infty A_i \in M \),
    \item if \( B_1, B_2, \ldots \in M \) and \( B_1 \supseteq B_2 \supseteq \cdots \), then
    \( \bigcap_{i=1}^\infty B_i \in M \).
  \end{enumerate}
  Given a collection $\mathcal{F}$ of subsets of $X$, we call the smallest monotone class containing $\mathcal{F}$ the monotone class generated by $\mathcal{F}$.
\end{definition}

\begin{theorem}[Monotone class theorem]\label{thm:monotone_class}
  Let \(G\) be an algebra of subsets of a set \(X\). Then the monotone class generated by \(G\) coincides with the $\sigma$-algebra generated by \(G\).
\end{theorem}

\begin{proof}
  TODO
\end{proof}

\section{Debut}
The following proof is based on "R.F. Bass, The measurability of hitting times, Electron. Commun. Probab. {\bf 15} (2010), 99--105; MR2606507"
and the successive "R.F. Bass. "Correction to "The measurability of hitting times"." Electron. Commun. Probab. 16 189 - 191, 2011. \url{https://doi.org/10.1214/ECP.v16-1627}"
which is a rather clever and short proof evading the classical proof which uses more complex structures. Note that there exists also an Arxiv version of the paper with the corrections applied (\url{https://arxiv.org/pdf/1001.3619}), we will mostly reference this unified version.


Standard notation in this chapter:
$(\Omega, \mathcal{F} , P)$ is a probability space;
$\mathcal{S}$ is a topological space;
$\pi:\mathbb{R}_{\geq 0}\times\Omega\rightarrow \Omega$ is the projection; $P^*$ is the outer measure associated with $P$.

\begin{definition}[Progressively measurable set]\label{def:progr_meas_set}
  \leanok
  \uses{def:ProgMeasurable}
  \lean{MeasureTheory.ProgMeasurableSet}
A subset of $[0, \infty) \times \Omega$ is progressively measurable if its indicator is a progressively measurable process.
\end{definition}

\begin{definition}[Debut of a set]\label{def:debut_set}
  \leanok
  \lean{MeasureTheory.Debut}
Let $E \subseteq{} [0, \infty) \times \Omega $, define $D_E = \inf\left\lbrace t \geq 0\ :\ (t, \omega) \in E\right\rbrace$, the debut of $E$.
\end{definition}

\begin{definition}[$\mathcal{K}^0$]\label{def:subsets_compact_RNN_times_measurable}
  \leanok
  \lean{MeasureTheory.𝓚₀}
Let $t>0$. Let $\mathcal{K}^0(t)$ be the collection of subsets of $[0, t] \times \Omega$ of the form $K \times C$, where $K$ is a compact
subset of $[0, t]$ and $C \in \mathcal{F}_t$.
\end{definition}

\begin{definition}[$\mathcal{K}$]\label{def:fin_union_RNN_times_measurable}
  \leanok
  \lean{MeasureTheory.𝓚}
  \uses{def:subsets_compact_RNN_times_measurable}
Let $t>0$. Let $\mathcal{K}(t)$ be the collection of finite unions of elements of $\mathcal{K}^0(t)$.
\end{definition}

\begin{definition}[$\mathcal{K}_\delta$]\label{def:count_inter_of_fin_union_RNN_times_measurable}
  \leanok
  \lean{MeasureTheory.𝓚δ}
  \uses{def:fin_union_RNN_times_measurable}
Let $t>0$. Let $\mathcal{K}_\delta(t)$ be the collection of countable intersections of elements of $\mathcal{K}(t)$.
\end{definition}

\begin{definition}[$t$-approximable set]\label{def:t_approx_set}
  \leanok
  \lean{MeasureTheory.Approximation}
  \uses{def:count_inter_of_fin_union_RNN_times_measurable}
Let $t>0$.
We say $A \in \mathcal{B}[0, t] \times \mathcal{F}_t$ is
$t$-approximable if given $\epsilon > 0$, there exists $B \in \mathcal{K}_\delta (t)$ with $B \subseteq{} A$ and
$$P^∗ (\pi(A)) \leq P^∗ (\pi(B)) + \epsilon,$$
where $\pi$ is the projection over $\Omega$.
\end{definition}

\begin{lemma}\label{lem:iInf_snd_eq_snd_iInf}
  \leanok
  \lean{MeasureTheory.iInf_snd_eq_snd_iInf_of_mem_𝓚δ}
  \uses{def:count_inter_of_fin_union_RNN_times_measurable}
If $B \in \mathcal{K}_\delta (t)$, $\forall n\in \mathbb{N}$, $B_n \in \mathcal{K}^\delta(t)$ and $B_n \searrow B$, then $\pi(B) = \bigcap_{n\in\mathbb{N}} \pi(B_n)$.
\end{lemma}

\begin{proof}
  % See the proof of Lemma 2.2 in the corrected paper.

  For each $\omega \in \Omega$ and each set $C \subseteq [0,+\infty) \times \Omega$, let
  \[S(C)(\omega) = \{s \leq t : (s,\omega) \in C\}.\]

  If $B \in \mathcal{K}_\delta(t)$, then $S(B)(\omega)$
  is compact. In fact, there exists a sequence $A_n \in \mathcal{K}(t)$ such that
  $A_n \searrow B$, therefore $S(A_n)(\omega)$ is compact and $S(A_n)(\omega) \searrow S(B)(\omega)$. In particular, $S(B_n)(\omega)$ is compact for each $n$.

  Now we divide the proof in two cases.

  One possibility is that
  $\bigcap_{n \in \mathbb{N}} S(B_n)(\omega) \neq \emptyset$; in this case, if $s \in \bigcap_{n \in \mathbb{N}} S(B_n)(\omega)$,
  then $(s,\omega) \in B_n$ for each $n$, and so $(s,\omega) \in B$. Therefore,
  $\omega \in \pi(B_n)$ for each $n$ and $\omega \in \pi(B)$.

  The other possibility is that $\bigcap_{n \in \mathbb{N}} S(B_n)(\omega) = \emptyset$.
  Since the sequence $S(B_n)(\omega)$ is a decreasing sequence of compact sets,
    $S(B_n)(\omega) = \emptyset$ for some $n$,
  for otherwise $\bigcap_{n \in \mathbb{N}} S(B_n)(\omega)$ would also be nonempty.
  Therefore $\omega \notin \pi(B_n)$ and $\omega \notin \pi(B)$.

  We conclude that $\omega \in \pi(B)$ if and only if $\omega \in \bigcap_{n \in \mathbb{N}} \pi(B_n)$, hence $\pi(B) = \bigcap_{n \in \mathbb{N}} \pi(B_n)$.
\end{proof}

\begin{lemma}\label{lem:K_delta_of_inter_K_delta}
  \leanok
  \lean{MeasureTheory.measurableSet_snd_of_mem_𝓚δ}
  \uses{def:count_inter_of_fin_union_RNN_times_measurable}
If $B \in \mathcal{K}_\delta (t)$, then $\pi(B) \in \mathcal{F}_t$.
\end{lemma}

\begin{proof}
  \uses{lem:iInf_snd_eq_snd_iInf}
  % See the proof of Lemma 2.2 in the corrected paper.
  By definition of $\mathcal{K}_\delta(t)$, $B = \bigcap_{n\in\mathbb{N}} B_n$ where $B_n \in \mathcal{K}(t)$. Therefore, by Lemma~\ref{lem:iInf_snd_eq_snd_iInf}, \[\pi(B) = \pi \left(\bigcap_{n\in\mathbb{N}} B_n\right) = \bigcap_{n\in\mathbb{N}} \pi(B_n) \in \mathcal{F}_t .\]
\end{proof}

\begin{lemma}\label{lem:measurable_of_t_approx}
  \leanok
  \lean{MeasureTheory.Approximation.measurableSet_snd}
  \uses{def:t_approx_set}
  If $A$ is $t$-approximable, then $\pi(A) \in \mathcal{F}_t$.
\end{lemma}

\begin{proof}
  \uses{lem:K_delta_of_inter_K_delta}
  % See the proof of Lemma 2.3 in the corrected paper.
  Choose $A_n \in \mathcal{K}_\delta(t)$ with $A_n \subseteq A$ and
  $P(\pi(A_n)) \to P^*(\pi(A))$. Let $B_n = A_1 \cup \cdots \cup A_n$ and
  let $B = \bigcup_{n \in \mathbb{N}} B_n$. Then
  $B_n \in \mathcal{K}_\delta(t)$, $B_n \nearrow B$, and $P(\pi(B_n)) \geq P(\pi(A_n)) \to
  P^*(\pi(A))$.
  Moreover, by Lemma~\ref{lem:K_delta_of_inter_K_delta}, $\pi(B_n) \in \mathcal{F}_t$.
  It follows that $\pi(B_n) \nearrow \pi(B)$, and so $\pi(B) \in \mathcal{F}_t$ and
  $$P(\pi(B)) = \lim_{n \to \infty} P(\pi(B_n)) = P^*(\pi(A)).$$

  For each $n$, there exists $C_n \in \mathcal{F}$ such that
  $\pi(A) \subseteq C_n$ and $P(C_n) \leq P^*(\pi(A)) + 1/n$. Setting
  $C = \bigcap_{n \in \mathbb{N}} C_n$, we have $\pi(A) \subseteq C$ and $P^*(\pi(A)) = P(C)$.
  Therefore
  $\pi(B) \subseteq \pi(A) \subseteq C$ and $P(\pi(B)) = P^*(\pi(A)) = P(C)$.
  This implies that $\pi(A) \setminus \pi(B)$ is a $P$-null set, and by
  the completeness assumption, $\pi(A) = (\pi(A) \setminus \pi(B)) \cup \pi(B) \in \mathcal{F}_t$.
\end{proof}

\begin{lemma}\label{lem:exists_B_of_t_approx}
  \leanok
  \lean{MeasureTheory.Approximation.tendsto_measure_diff_B'}
  \uses{def:count_inter_of_fin_union_RNN_times_measurable, def:t_approx_set}
  Suppose $A$ is $t$-approximable. Then, given $\epsilon > 0$, there exists
  $B \in \mathcal{K}_\delta (t)$ such that $P(\pi(A) \setminus \pi(B)) < \epsilon$.
\end{lemma}

\begin{proof}
  \uses{lem:K_delta_of_inter_K_delta}
  % See the proof of Lemma 2.3 in the corrected paper.
  Let $B_n$ and $B$ be as in the proof of Lemma~\ref{lem:measurable_of_t_approx}.
  Then,
  $$\lim_{n \to \infty} P(\pi(A) \setminus \pi(B_n)) = P(\pi(A) \setminus \pi(B)) = 0.$$
\end{proof}

% consider removing this lemma altogether, I think we do not even need it in the proof
\begin{lemma}\label{lem:aux1a}
  \leanok
If $A \subseteq \Omega$, there exists $C \in \mathcal{F}$ such that $A \subseteq C$ and $P^∗(A) = P(C)$.
\end{lemma}

\begin{proof}\leanok
  This is just \verb|MeasureTheory.exists_measurable_superset|.
\end{proof}

\begin{lemma}\label{lem:aux1b}
Let $(A_n)_{n\in\mathbb{N}},A\subseteq \Omega$.
Suppose $A_n \nearrow A$. Then $P^∗(A) = \lim_{n\rightarrow \infty} P^∗(A_n)$.
\end{lemma}

\begin{proof}\leanok
  This is just \verb|Monotone.measure_iUnion| (this is a version with the sup, if needed there is also the version with the limit).
\end{proof}

\begin{definition}[$\mathcal{L}$-sets]\label{def:L_sets}
  \leanok
  \lean{MeasureTheory.𝓛₀, MeasureTheory.𝓛₁, MeasureTheory.𝓛, MeasureTheory.𝓛σ, MeasureTheory.𝓛σδ}
  \uses{def:count_inter_of_fin_union_RNN_times_measurable}
  From hereafter the following sets are needed:

  \begin{itemize}
  \item $\mathcal{L}_0(X) := \left\lbrace A \times B\ :\ A \subseteq X ,\ A \text{ compact},\ B \in \mathcal{K}(t)\right\rbrace$
  \item $\mathcal{L}_1(X )$ the class of finite unions of sets in $\mathcal{L}_0(X )$
  \item $\mathcal{L} (X )$ the class of intersections of countable decreasing sequences in $\mathcal{L}_1(X )$
  \item $\mathcal{L}_\sigma(X )$ be the class of unions of countable increasing sequences of sets in $\mathcal{L} (X )$
  \item $\mathcal{L}_{\sigma\delta}(X )$ the class of intersections of countable decreasing sequences of sets in $\mathcal{L}_\sigma(X )$
  \end{itemize}
\end{definition}

\begin{lemma}\label{lem:exists_cpct_Hausdorff}
  \leanok
  \lean{MeasureTheory.exists_mem_𝓛σδ_of_measurableSet}
  \uses{def:L_sets}
If $A \in \mathcal{B}[0, t] \times \mathcal{F}_t$, there exists a compact Hausdorff space $X$ and $B \in \mathcal{L}_{\sigma\delta}(X )$ such  that $A = \rho^X (B)$.

Where $\rho^X:X\times ([0,t]\times\Omega)\rightarrow [0,t]\times\Omega$ is the projection.
\end{lemma}

\begin{proof}
  \uses{thm:monotone_class}
  % See Lemma 2.5 in the corrected paper.
  %TODO: this proof needs to be expanded in multiple lemmas, for now I just copy pasted it from the paper, but probably we will need to separately define the set M, prove as a lemma that it is a monotone class, etc. We will also need to find the monotone class theorem or prove it ourselves, in the latter case I think we will need to have a section dedicated to it.

  TODO: Reorganize this proof, possibly divide it in multiple lemmas.

  If $A \in \mathcal{K}(t)$, we take $X = [0,1]$, the unit interval with the
  usual topology and $B = X \times A$. Thus the collection $\mathcal{M}$ of
  subsets of $\mathcal{B}[0,t] \times \mathcal{F}_t$
  for which the lemma is satisfied contains $\mathcal{K}(t)$. We will
  show that $\mathcal{M}$ is a monotone class.


  Suppose $A_n \in \mathcal{M}$ with $A_n \downarrow A$. There exist compact Hausdorff
  spaces $X_n$ and sets $B_n \in \mathcal{L}_{\sigma\delta}(X_n)$ such that $A_n = \rho^{X_n}(B_n)$.
  Let $X = \prod_{n=1}^\infty X_n$ be furnished with the product topology. Let
  $\tau_n: X \times [0,t] \times \Omega \to X_n \times [0,t] \times \Omega$ be defined by $\tau_n(x,(s,\omega))
  = (x_n,(s,\omega))$ if $x = (x_1,x_2, \ldots)$. Let $C_n = \tau_n^{-1}(B_n)$
  and let $C = \bigcap_{n \in \mathbb{N}} C_n$. It is easy to check that $\mathcal{L}(X)$ is closed under
  the operations of finite unions and intersections, from which it follows
  that $C \in \mathcal{L}_{\sigma\delta}(X)$. If $(s,\omega) \in A$, then for each $n$ there exists $x_n \in X_n$ such that $(x_n,(s,\omega)) \in B_n$. Note that
  $((x_1,x_2, \ldots),(s,\omega)) \in C$ and therefore $(s,\omega) \in \rho^X(C)$.
  It is straightforward that $\rho^X(C) \subseteq A$, and we conclude
  $A \in \mathcal{M}$.

  Now suppose $A_n \in \mathcal{M}$ with $A_n \uparrow A$. Let $X_n$ and $B_n$ be as before.
  Let $X' = \bigcup_{n=1}^\infty (X_n \times \{n\})$ with the topology generated by
  $\{G \times \{n\}: G \text{ open in } X_n\}$. Let $X$ be the one point
  compactification of $X'$. We can write $B_n = \bigcap_{m \in \mathbb{N}} B_{nm}$ with
  $B_{nm} \in \mathcal{L}_\sigma(X_n)$. Let
  $$C_{nm} = \{((x,n),(s,\omega)) \in X \times [0,t] \times \Omega: x \in X_n, (x,(s,\omega)) \in B_{nm}\},$$
  $C_n = \bigcap_{m \in \mathbb{N}} C_{nm}$, and $C = \bigcup_{n \in \mathbb{N}} C_n$.
  Then $C_{nm} \in \mathcal{L}_\sigma(X)$ and so $C_n \in \mathcal{L}_{\sigma\delta}(X)$.

  If $((x,p),(s,\omega)) \in \bigcap_{m \in \mathbb{N}} \bigcup_{n \in \mathbb{N}} C_{nm}$, then
  for each $m$ there exists $n_m$ such that $((x,p),(s,\omega)) \in C_{n_mm}$.
  This is
  only possible if $n_m = p$ for each $m$. Thus $((x,p), (s, \omega)) \in \bigcap_{m \in \mathbb{N}} C_{pm} = C_p \subseteq C$.
  The other inclusion is easier and we thus obtain $C = \bigcap_{m \in \mathbb{N}}\bigcup_{n \in \mathbb{N}} C_{nm}$,
  which implies $C \in \mathcal{L}_{\sigma\delta}(X)$. We check that
  $A = \rho^X(C)$ along the same lines, and therefore $A \in \mathcal{M}$.


  If $\mathcal{I}^0(t)$ is the collection of sets of the form $[a,b) \times C$, where
  $a < b \leq t$ and $C \in \mathcal{F}_t$, and $\mathcal{I}(t)$ is the collection of finite
  unions of sets in $\mathcal{I}^0(t)$, then $\mathcal{I}(t)$ is an algebra of sets. We
  note that $\mathcal{I}(t)$ generates the $\sigma$-field $\mathcal{B}[0,t] \times \mathcal{F}_t$. A set
  in $\mathcal{I}^0(t)$ of the form $[a,b) \times C$ is the union of sets in $\mathcal{K}^0(t)$
  of the form $[a, b-(1/m)] \times C$, and it
  follows that every set in $\mathcal{I}(t)$ is the increasing union of sets
  in $\mathcal{K}(t)$. Since $\mathcal{M}$ is a monotone
  class containing $\mathcal{K}(t)$, then $\mathcal{M}$ contains $\mathcal{I}(t)$.
  By the monotone class theorem (Theorem~\ref{thm:monotone_class}), $\mathcal{M} = \mathcal{B}[0,t] \times \mathcal{F}_t$.
\end{proof}

\begin{lemma}\label{lem:t_approx_of_Borel_measurable}
  \leanok
  \lean{MeasureTheory.Approximation.of_mem_prod_borel}
  \uses{def:t_approx_set}
If $A \in \mathcal{B}[0, t] \times \mathcal{F}_t$, then $A$ is $t$-approximable.
\end{lemma}

\begin{proof}
  \uses{lem:exists_cpct_Hausdorff}
  % See the proof of Lemma 2.6 in the corrected paper.
  %TODO: expand this proof, this may need some auxiliary lemmas. For now I just copy pasted the proof from the paper

  TODO: Reorganize this proof, possibly divide it in multiple lemmas.

  We first prove that if $H \in \mathcal{L}(X)$, then $\rho^X(H) \in \mathcal{K}_\delta$. If $H \in \mathcal{L}_1(X)$,
  this is clear. Suppose that $H_n \downarrow H$ with each $H_n \in \mathcal{L}_1(X)$.
  If $(s,\omega) \in \bigcap_{n \in \mathbb{N}} \rho^X(H_n)$, there exist
  $x_n \in X$ such that $(x_n,(s,\omega)) \in H_n$. Then there exists a subsequence
  such that $x_{n_k} \to x_\infty$ by the compactness of $X$. Now $(x_{n_k},(s,\omega)) \in H_{n_k}
  \subseteq H_m$ for $n_k$ larger than $m$. For fixed $\omega$, $\{(x,s): (x,(s,\omega)) \in H_m\}$
  is compact, so $(x_\infty,(s,\omega)) \in H_m$ for all $m$. This implies
  $(x_\infty,(s,\omega)) \in H$. The other inclusion is easier and therefore $\bigcap_{n \in \mathbb{N}} \rho^X(H_n) = \rho^X(H)$.
  Since $\rho^X(H_n) \in \mathcal{K}_\delta(t)$, then $\rho^X(H) \in \mathcal{K}_\delta(t)$.
  We also observe that for fixed $\omega$, $\{(x,s):(x,(s,\omega)) \in H\}$
  is compact.

  Now suppose $A \in \mathcal{B}[0,t] \times \mathcal{F}_t$. Then by Lemma~\ref{lem:exists_cpct_Hausdorff}
  there exists a compact Hausdorff space $X$ and $B \in \mathcal{L}_{\sigma\delta}(X)$ such that $A = \rho^X(B)$. We can write
  $B = \bigcap_{n \in \mathbb{N}} B_n$ and $B_n = \bigcup_{m \in \mathbb{N}} B_{nm}$ with $B_n \downarrow B$, $B_{nm} \uparrow B_n$, and $B_{nm} \in \mathcal{L}(X)$.

  Let $a = P^*(\pi(A)) = P^*(\pi \circ \rho^X(B))$ and let $\epsilon > 0$.
  By Lemma~\ref{lem:aux1b},
  $$P^*(\pi \circ \rho^X(B \cap B_{1m})) \uparrow P^*(\pi \circ \rho^X(B \cap B_1))
  = P^*(\pi \circ \rho^X(B)) = a.$$
  Take $m$ large enough so that $P^*(\pi \circ \rho^X(B \cap B_{1m})) > a - \epsilon$,
  let $C_1 = B_{1m}$, and $D_1 = B \cap C_1$.

  We proceed by induction. Suppose we are given sets $C_1, \ldots, C_{n-1}$ and
  sets $D_1, \ldots, D_{n-1}$ with $D_{n-1} = B \cap \left(\bigcap_{i=1}^{n-1} C_i\right)$, $P^*(\pi
  \circ \rho^X(D_{n-1})) > a - \epsilon$, and each $C_i = B_{im_i}$ for
  some $m_i$. Since $D_{n-1} \subseteq B \subseteq B_n$, by Lemma~\ref{lem:aux1b}
  $$P^*(\pi \circ \rho^X(D_{n-1} \cap B_{nm}))
  \uparrow P^*(\pi \circ \rho^X(D_{n-1} \cap B_n))
  = P^*(\pi \circ \rho^X(D_{n-1})).$$
  We can take $m$ large enough so that
  $P^*(\pi \circ \rho^X(D_{n-1} \cap B_{nm})) > a - \epsilon$, let $C_n = B_{nm}$, and $D_n = D_{n-1} \cap C_n$.

  If we let $G_n = C_1 \cap \cdots \cap C_n$ and $G = \bigcap_{n \in \mathbb{N}} G_n = \bigcap_{n \in \mathbb{N}} C_n$, then
  each $G_n$ is in $\mathcal{L}(X)$, hence $G \in \mathcal{L}(X)$. Since $C_n \subseteq B_n$, then
  $G \subseteq \bigcap_{n \in \mathbb{N}} B_n = B$.
  Each $G_n \in \mathcal{L}(X)$ and so by the first paragraph of this proof, for each
  fixed $\omega$ and $n$, $\{(x,s): (x,(s,\omega)) \in G_n\}$
  is compact. Hence, by a proof very similar to that of Lemma~\ref{lem:iInf_snd_eq_snd_iInf},
  $\pi \circ \rho^X(G_n) \downarrow \pi \circ \rho^X(G)$.
  Using the first paragraph of this proof and Lemma~\ref{lem:iInf_snd_eq_snd_iInf},
  we see that $$P(\pi \circ \rho^X(G))
  = \lim_{n \to \infty} P(\pi \circ \rho^X(G_n)) \geq \lim_{n \to \infty} P^*(\pi \circ \rho^X(D_n)) \geq a - \epsilon.$$

  Using the first paragraph of this proof once again, we see that $A$ is $t$-approximable.
\end{proof}

\begin{theorem}\label{thm:debut_of_progr_meas_is_stop_time}
  \leanok
  \lean{MeasureTheory.Debut.isStoppingTime}
  \uses{def:progr_meas_set, def:debut_set}
If $E$ is a progressively measurable set, then $D_E$ is a stopping time.
\end{theorem}

\begin{proof}
  \uses{def:t_approx_set, lem:t_approx_of_Borel_measurable, lem:measurable_of_t_approx}
  % See the proof of Theorem 2.1 in the corrected paper.
  % TODO: It seems we need right continuity of the filtration, should we state it as an assumption?
  Let $E$ be a progressively measurable set and let $A_u = E \cap ([0,u] \times \Omega)$.
  By Lemma~\ref{lem:t_approx_of_Borel_measurable}, $A_u$ is $u$-approximable.
  By Lemma~\ref{lem:measurable_of_t_approx}, $\pi(A_u) \in \mathcal{F}_u$.
  Now fix $t$. If $\omega \in \{D_E \leq t\}$, we see that $\omega \in \pi(A_u)$ for all $u > t$.
  Conversely, if $\omega \in \pi(A_u)$ for all $u > t$, then $\omega \in \{D_E \leq t\}$.
  If $u_1 < u_2$, then $A_{u_1} \subseteq A_{u_2}$ and hence $\pi(A_{u_1}) \subseteq
  \pi(A_{u_2})$. Therefore
  $$\{D_E \leq t\} = \bigcap_{u > t} \pi(A_u) \in \bigcap_{u > t} \mathcal{F}_u = \mathcal{F}_t.$$
  Because $t$ was arbitrary, we conclude $D_E$ is a stopping time.
\end{proof}

\section{Hitting times}

% TODO: do we want to do the same distinction that the paper makes between the hitting time and the entry time?
% In the paper they are defined as follows:
% Entry time: inf{t \geq 0: X_t \in B}
% Hitting time: inf{t > 0: X_t \in B}
% What we call hitting time in Mathlib is actually a generalization of the entry time, but does not completely cover the hitting time. In case we would like to have both we may want to define another version of the hitting time with the strict inequality (essentially with Set.Ioc or Set.Ioo instead of Set.Icc).
% For the moment this does not seem necessary, and in any case the proof for the hitting time relies on the one for the entry time and takes a limit, so we can always add it later if needed.

\begin{theorem}\label{thm:hitting_is_stopping_time}
  \leanok
  \lean{MeasureTheory.hitting_isStoppingTime'}
  \uses{def:IsStoppingTime, def:hittingAfter, def:ProgMeasurable, def:rightContinuous}
  \notready
If $X : T \to \Omega \to E$ is a progressively measurable process with respect to a right-continuous filtration and $B$ is a Borel-measurable subset of $E$, then the hitting time of $X$ in $B$ is a stopping time.
\end{theorem}

\begin{proof}
  \uses{thm:debut_of_progr_meas_is_stop_time, lem:rightContinuous_basic}
  % See the proof of Theorem 2.7 in the corrected paper.
  Since $B$ is a Borel subset of $\mathcal{S}$ and $X$ is progressively measurable,
  then $\mathbf{1}_B(X_t)$ is also progressively measurable. The hitting time is then the debut of the set
  $E = \{(s,\omega) : \mathbf{1}_B(X_s(\omega)) = 1\}$, and therefore is a stopping time by Theorem~\ref{thm:debut_of_progr_meas_is_stop_time}.
\end{proof}


\section{Corollaries}

TODO: need to generalize the \texttt{leastGE} definition to general index sets that are not necessarily $\mathbb{N}$.


\begin{definition}\label{def:leastGE}
  \mathlibok
  \lean{MeasureTheory.leastGE}
For a process $X : ι \to Ω \to ℝ$ and a real number $a$, define the random time
\begin{align*}
  \tau_{X \ge a} = \inf\{t \in ι \mid X_t \ge a\} \: ,
\end{align*}
in which the infimum is infinite if the set is empty.
\end{definition}


\begin{lemma}\label{lem:isStoppingTime_leastGE}
  \uses{def:IsStoppingTime, def:leastGE}
If $X : ι \to Ω \to ℝ$ is a progressively measurable process with respect to a right-continuous filtration, then for any $a \in \mathbb{R}$, the random time $\tau_{X \ge a}$ is a stopping time.
\end{lemma}

\begin{proof}
  \uses{thm:hitting_is_stopping_time}
This is a direct application of Theorem~\ref{thm:hitting_is_stopping_time} with the set $B = [a, +\infty)$.
\end{proof}


\begin{corollary}\label{cor:isStoppingTime_leastGE_of_rightContinuous}
  \uses{def:IsStoppingTime, def:leastGE, def:rightContinuous, def:adapted, def:RightContinuous}
If $X : ι \to Ω \to ℝ$ is a right-continuous and adapted process with respect to a right-continuous filtration, then for any $a \in \mathbb{R}$, the random time $\tau_{X \ge a}$ is a stopping time.
\end{corollary}

\begin{proof}
  \uses{lem:isStoppingTime_leastGE, lem:Adapted.progMeasurable_of_rightContinuous}
This follows from Lemma~\ref{lem:isStoppingTime_leastGE} since $X$ is progressively measurable by Lemma~\ref{lem:Adapted.progMeasurable_of_rightContinuous}.
\end{proof}

\chapter{Doob-Meyer Theorem}
\label{chap:doob_meyer}


This chapter starts with a short review of the properties of the Doob decomposition of an adapted process indexed on a discrete set, and then follows \cite{Beiglböck_Schachermayer_Veliyev_2012} which gives an elementary and short proof of the Doob-Meyer theorem.


\section{Doob decomposition in discrete time}


\begin{definition}\label{def:predictablePart}
  \uses{def:filtration}
  \mathlibok
  \lean{MeasureTheory.predictablePart}
Let $X : \mathbb{N} \to \Omega \to E$ be a process indexed by $\mathbb{N}$, for $E$ a Banach space.
Let $(\mathcal{F}_n)_{n\in\mathbb{N}}$ be a filtration on $\Omega$.
The predictable part of $X$ is the process $A : \mathbb{N} \to \Omega \to E$ defined by $A_0 = 0$ and for $n \ge 0$,
\begin{align*}
  A_{n+1}
  &= A_n + \mathbb{E}\left[ X_{n+1} - X_n | \mathcal{F}_n \right]
  \: .
\end{align*}
\end{definition}


\begin{definition}\label{def:martingalePart}
  \uses{def:predictablePart}
  \mathlibok
  \lean{MeasureTheory.martingalePart}
Let $X : \mathbb{N} \to \Omega \to E$ be a process indexed by $\mathbb{N}$, for $E$ a Banach space.
Let $(\mathcal{F}_n)_{n\in\mathbb{N}}$ be a filtration on $\Omega$ and let $A$ be the predictable part of $X$ for that filtration.
The martingale part of $X$ is the process $M : \mathbb{N} \to \Omega \to E$ defined by $M_n = X_n - A_n$.
\end{definition}


\begin{lemma}\label{lem:predictable_predictablePart}
  \uses{def:predictablePart, def:predictable, def:adapted}
The predictable part of an adapted process is a predictable process.
\end{lemma}

\begin{proof}
This is almost \texttt{MeasureTheory.adapted\_predictablePart} but we need to formulate it with the new predictable definition.
\end{proof}


\begin{lemma}\label{lem:martingale_martingalePart}
  \uses{def:martingalePart, def:Martingale}
  \mathlibok
  \lean{MeasureTheory.martingale_martingalePart}
Suppose that the filtration is sigma-finite.
Then the martingale part of an adapted process $X$ such that $X_n$ is integrable for all $n$ is a martingale.
\end{lemma}

\begin{proof}\leanok

\end{proof}


\begin{lemma}\label{lem:nondecreasing_predictablePart_of_submartingale}
  \uses{def:predictablePart, def:Submartingale}
The predictable part of a submartingale is a nondecreasing process.
\end{lemma}

\begin{proof}
  \uses{def:Submartingale}
Let $X$ be a submartingale and let $A$ be its predictable part. Then for all $n \geq 0$,
\begin{align*}
  A_{n+1} &= A_n + \mathbb{E}\left[ X_{n+1} - X_n | \mathcal{F}_n \right] \\
  &\ge A_n
  \: .
\end{align*}
The last inequality follows from the submartingale property of $X$.
Thus, $(A_n)_{n \in \mathbb{N}}$ is nondecreasing.
\end{proof}


\section{Cadlag modifications of (local) martingales}



\begin{definition}[Dyadics]\label{def:dyadics}
For $T>0$, let $\mathcal{D}_n^T = \left\lbrace \frac{k}{2^n}T \mid k=0,\cdots 2^n\right\rbrace$ be the set of dyadics at scale $n$ and let $\mathcal{D}^T=\bigcup_{n\in\mathbb{N}}\mathcal{D}_n^T$ be the set of all dyadics of $[0,T]$.
\end{definition}


\begin{lemma}\label{lem:martingale_exists_dyadic_limit_left}
  \uses{def:dyadics, def:Martingale}
  Let $X=(X_t)_{t\in\mathcal{D}}$ be a martingale indexed by the dyadics. Then almost surely, for every $t\geq 0$ the limit
  $$
  \lim_{\stackrel{s\rightarrow t^-}{s\in\mathcal{D}}}X_s(\omega)
  $$
  exists and is finite.
\end{lemma}

\begin{proof}
  See 8.2.1 of Pascucci.
\end{proof}


\begin{lemma}\label{lem:martingale_exists_dyadic_limit_right}
  \uses{def:dyadics, def:Martingale}
  Let $X=(X_t)_{t\in\mathcal{D}}$ be a martingale indexed by the dyadics. Then almost surely, for every $t\geq 0$ the limit
  $$
  \lim_{\stackrel{s\rightarrow t^+}{s\in\mathcal{D}}}X_s(\omega)
  $$
  exists and is finite.
\end{lemma}

\begin{proof}
  See 8.2.1 of Pascucci.
\end{proof}


\begin{lemma}\label{lem:mg_is_cadlag}
  \uses{def:usualConditions, def:Martingale}
  Let the filtered probability space satisfy the usual conditions.
  Then every martingale $X$ admits a modification that is still a martingale with cadlag trajectories.
\end{lemma}

\begin{proof}
  \uses{lem:martingale_exists_dyadic_limit_right,lem:martingale_exists_dyadic_limit_left}
  See 8.2.3 of Pascucci.
\end{proof}


\begin{lemma}\label{lem:exists_cadlag_mod_of_nonneg_submg}
  \uses{def:usualConditions, def:Submartingale}
  Let the filtered probability space satisfy the usual conditions.
  Then every nonnegative submartingale $X$ admits a modification that is still a nonnegative submartingale with cadlag trajectories.
\end{lemma}

\begin{proof}
  \uses{lem:martingale_exists_dyadic_limit_right,lem:martingale_exists_dyadic_limit_left}
  See 8.2.3 of Pascucci.
\end{proof}


\begin{lemma}\label{lem:exists_cadlag_mod_of_local_mg}
  \uses{def:usualConditions}
  Let the filtered probability space satisfy the usual conditions.
  Then every local martingale $X$ admits a modification that is still a local martingale with cadlag trajectories.
\end{lemma}

\begin{proof}
  \uses{lem:mg_is_cadlag}
\end{proof}



\section{Komlòs Lemma}



Firstly we will need Komlos' Lemma.
%technically a more general version with Cesaro sums exists, but it is not needed for this case
%(see "J. Komlòs, A generalization of a problem of Steinhaus, Acta Math. Acad. Sci. Hungar. 18 (1967) 217–229").


\begin{lemma}\label{lem:komlos_aux}
  Let $H$ be a Hilbert space and $(f_n)_{n\in\mathbb{N}}$ a bounded sequence in $H$. Then there exist functions $g_n\in convex(f_n,f_{n+1},\cdots)$ such that $(g_n)_{n\in\mathbb{N}}$ converges in $H$.
\end{lemma}

\begin{proof}
  Let $r_n = \inf(\|g\|_2:g\in convex(f_n, f_{n+1},\ldots))$.
  Let $A=\sup_{n\geq1} r_n$. $A$ is finite by boundedness of $(f_n)_{n\in\mathbb{N}}$ and
  for each $n$ we  may pick some $g_n\in convex(f_n, f_{n+1},\ldots)$ such that $ \|g_n\|_2\leq A+1/n$ by $\inf$ and $\sup$ definitions.
  Let $\epsilon>0$.
  By construction $(r_n)_{n\in\mathbb{N}}$ is increasing. By properties of $\sup$ there exists $\bar{n}$ such that $r_{\bar{n}}\geq A-\epsilon$ and such that $\frac{1}{\bar{n}}\leq\epsilon$.
  Let $m\geq k\geq \bar{n}$. $(g_k+g_m)/2 \in convex(f_k,f_{k+1},\ldots)$. It follows since $(r_n)_{n\in\mathbb{N}}$ is increasing that
  $\|(g_k+g_m)/2\|_2\geq A-\epsilon$.
  Hence due to the ordering of $m,k,\bar{n}$
  $$ \|g_k-g_m\|_2^2=2 \|g_k\|_2^2+2\|g_m\|_2^2- \|g_k+g_m\|_2^2
  \leq 4(A+\frac{1}{\bar{n}})^2-4(A-\epsilon)^2\leq 16A\epsilon.$$ By completeness, $(g_n)_{n\geq1}$  converges in $\|.\|_2$.
\end{proof}


\begin{lemma}\label{lem:convex_of_converg_seq_is_converg}
  Let $X$ be a normed vector space (over $\mathbb{R}$).
  %For topological spaces we need that a convex combinations of elements of neighborhoods are still in the neighborhood (just like balls). Also for metric space we want that $d(ax,ay)\leq a d(x,y)$ (in lean dist_pair_smul).
  Let $(x_n)_{n\in\mathbb{N}}$ be a sequence in $X$ converging to $x$ w.r.t. the topology of $X$.
  Let $(N_n)_{n\in\mathbb{N}}$ be a sequence in $\mathbb{N}$ such that $n\leq N_n$ for every $n\in\mathbb{N}$ (maybe here we could have $N_n$ increasing WLOG).
  Let $(a_{n,m})_{n\in\mathbb{N},m\in\left\lbrace n,\cdots,N_n\right\rbrace}$ be a triangular array in $\mathbb{R}$ such that $0\leq a_{n,m}\leq 1$ and $\sum_{m=n}^{N_n}a_{n,m}=1$.
  Then $(\sum_{m=n}^{N_n}a_{n,m}x_m)_{n\in\mathbb{N}}$ converges to $x$ uniformly w.r.t. the triangular array.
\end{lemma}

\begin{proof}
  Let $\epsilon>0$.
  By convergence of $x_n$ we have $\exists \bar{n}$ such that $\forall n\geq\bar{n}$ $|x_n-x|\leq \epsilon$.
  By triangular inequality it follows that
  $$
  |\sum_{m=n}^{N_n}a_{n,m}x_m - x|\leq \sum_{m=n}^{N_n}a_{n,m}|x_m-x|\leq\epsilon.
  $$
\end{proof}


\begin{lemma}\label{lem:komlos_convex_aux}
  For $i,n\in\mathbb{N}$ set $f_{n}^{(i)}:=f_n \mathbb{1}_{(|f_n|\leq i)}$ such that $f_{n}^{(i)}\in L^2$.
  There exists the sequence of convex weights $\lambda_n^{n}, \ldots, \lambda_{N_n}^{n}$ such that the functions
  $ (\lambda_n^{n} f_n^{(i)} + \ldots+\lambda_{N_n}^{n} f_{N_n}^{(i)})_{n\in\mathbb{N}}$
  converge in $L^2$ for every $i\in\mathbb{N}$ uniformly.
\end{lemma}

\begin{proof}
  \uses{lem:komlos_aux, lem:convex_of_converg_seq_is_converg}
  Firstly by lemma \ref{lem:komlos_aux} over $(f_n^{(1)})_{n\in\mathbb{N}}$ there exist convex weights $\prescript{1}{}{\lambda}^n_n,\cdots,\prescript{1}{}{\lambda}^n_{N^1_n}$ such that
  $g^1_n=\sum_{m=n}^{N^1_n}\prescript{1}{}{\lambda}^n_mf_m^{(1)}$ converges to some $g^1$.
  Secondly apply the lemma to $(\tilde{g}^2_n=\sum_{m=n}^{N^1_n}\prescript{1}{}{\lambda}^n_mf^{(2)}_m)_{n\in\mathbb{N}}$, there exists convex weights $\tilde{\lambda}^n_n,\cdots,\tilde{\lambda}^n_{\tilde{N}_n}$ such that
  $g^2_n=\sum_{m=n}^{\tilde{N}_n}\tilde{\lambda}^n_m\tilde{g}_m^{(2)}=\sum_{m=n}^{N^2_n}\prescript{2}{}{\lambda}^n_mf_m^{(2)}$ converges to some $g^2$.
  Notice that $\sum_{m=n}^{N^2_n}\prescript{2}{}{\lambda}^n_mf_m^{(1)}=\sum_{m=n}^{\tilde{N}_n}\tilde{\lambda}^n_m\tilde{g}_m^{(1)}$ and thus this sequence by lemma \ref{lem:convex_of_converg_seq_is_converg} converges still to $g^1$.
  By iteration we may define $\prescript{i}{}{\lambda}^n_n,\cdots,\prescript{i}{}{\lambda}^n_{N^i_n}$ convex weights such that if used on $(f^j_n)_{n\in\mathbb{N}}$ they make the sequence convergent if $1\leq j\leq i$.
  At this point consider $\lambda^n_m=\prescript{n}{}{\lambda}^n_m$.
  Since $\forall m\geq i$ $\sum_{j=n}^{N^m_n}\prescript{m}{}{\lambda}^n_j f^{(i)}_j\rightarrow g^i$ and even better
  $\forall\epsilon>0$ $\exists\bar{n}$, $\forall n\geq\bar{n}$, $\forall m\geq i$ $|\sum_{j=n}^{N^m_n}\prescript{m}{}{\lambda}^n_j f^{(i)}_j - g^i|\leq\epsilon$
  (this works by lemma \ref{lem:convex_of_converg_seq_is_converg} uniformity of convergence w.r.t. triangular array) this concludes.
\end{proof}


\begin{lemma}[Komlòs Lemma]\label{lem:komlos}
  Let $( f_n)_{n\in\mathbb{N}}$ be a uniformly integrable sequence of functions on a probability space $(\Omega , \mathcal{F} , P)$.
  Then there exist functions $g_n \in convex( f_n, f_{n+1}, \cdots)$ such that $(g_n)_{n\in\mathbb{N}}$ converges in  $L^1 (\Omega )$.
\end{lemma}

\begin{proof}
  \uses{lem:komlos_convex_aux}
  For $i,n\in\mathbb{N}$ set $f_{n}^{(i)}:=f_n \mathbb{1}_{(|f_n|\leq i)}$ such that $f_{n}^{(i)}\in L^2$.
  Using \ref{lem:komlos_convex_aux} there exist for every $n$ convex weights $\lambda_n^{n}, \ldots, \lambda_{N_n}^{n}$ such that the functions
  $ \lambda_n^{n} f_n^{(i)} + \ldots+\lambda_{N_n}^{n} f_{N_n}^{(i)}$ converge in $L^2$ for every $i\in\mathbb{N}$.
  By uniform integrability, $\lim_{i\to \infty}\| f^{(i)}_n- f_n\|_1=0$, uniformly with respect to $n$.
  Hence, once again, uniformly with respect to $n$,
  $$ \textstyle\lim_{i\to\infty}\|  (\lambda_n^{n} f_n^{(i)} + \ldots+\lambda_{N_n}^{n} f_{N_n}^{(i)})-(\lambda_n^{n} f_n + \ldots+\lambda_{N_n}^{n} f_{N_n})\|_1= 0.$$
  Thus $(\lambda_n^{n} f_n + \ldots+\lambda_{N_n}^{n} f_{N_n})_{n\geq 1}$  is a Cauchy sequence in $L^1$.
\end{proof}



\section{Doob-Meyer decomposition}



For uniqueness of Doob-Meyer Decomposition we will need theorem \ref{thm:IsLocalMartingale.eq_zero_of_finiteVariation}.

We now start the construction for the existence part.
Let $T>0$ and recall that $\mathcal{D}_n^T=\left\lbrace \frac{k}{2^n}T \mid k=0,\cdots 2^n\right\rbrace$.

TODO: everywhere below, $S$ is a cadlag submartingale of class D on $[0,T]$?


\begin{definition}\label{def:Doob_Meyer_class}
  \uses{def:IsStoppingTime}
$D$ is the class of all adapted processes $(S_t)_{0\leq t\leq T}$ such that the set $\{S_\tau \mid \tau \text{ is a stopping time}\}$ is uniformly integrable.
\end{definition}


\begin{definition}[A]\label{def:A}
  \uses{def:dyadics, def:predictablePart}
Define $A_0=0$ and for $t\in\mathcal{D}_n^T$ positive,
\begin{align*}
A^n_t
&=A^n_{t-T2^{-n}} + \mathbb{E}\left[ S_t-S_{t-T2^{-n}}|\mathcal{F}_{t-T2^{-n}}\right]
\: .
\end{align*}
\end{definition}


\begin{definition}[M]\label{def:M}
  \uses{def:A, def:martingalePart}
For $t\in\mathcal{D}_n^T$, define $M^n_t = S_t-A^n_t$~.
\end{definition}


\begin{lemma}\label{lem:Doob_Meyer_Finite_Predictable}
  \uses{def:A, def:predictable}
  $(A^n_t)_{t\in\mathcal{D}_n^T}$ is a predictable process.
\end{lemma}

\begin{proof}
  \uses{lem:predictable_nat_iff, lem:predictable_predictablePart}
  Trivial
\end{proof}


\begin{lemma}\label{lem:Doob_Meyer_Finite_Martingale}
  \uses{def:M, def:Martingale}
  $(M^n_t)_{t\in\mathcal{D}_n^T}$ is a martingale.
\end{lemma}

\begin{proof}
  \uses{lem:martingale_martingalePart}
  Trivial
\end{proof}


\begin{lemma}\label{lem:Predict_Part_Increasing}
  \uses{def:A}
  $(A^n_t)_{t\in\mathcal{D}_n^T}$ is an increasing process.
\end{lemma}

\begin{proof}
  \uses{def:Submartingale, lem:nondecreasing_predictablePart_of_submartingale}
$S$ is a submartingale:
\begin{align*}
  A^n_{t+T2^{-n}} - A^n_t
  &= \mathbb{E}\left[ S_{t+T2^{-n}}-S_t|\mathcal{F}_t\right] \ge 0
  \: .
\end{align*}
\end{proof}


\begin{definition}[Hitting time for $A$]\label{def:hittingAGT}
  \uses{def:A}
Let $c>0$. Define the hitting time on $\mathcal{D}^T_n$
\begin{align*}
  \tau_n(c)
  &= \inf\{t \in \mathcal{D}^T_n \mid A^n_{t + 2^{-n}T} > c\} \wedge T
  \: .
\end{align*}
\end{definition}


\begin{lemma}\label{lem:IsStoppingTime_hittingAGT}
  \uses{def:IsStoppingTime,def:hittingAGT}
  $\tau_n(c)$ is a stopping time.
\end{lemma}

\begin{proof}
Since $A^n_{t}$ is predictable, $A^n_{t + 2^{-n}T}$ is adapted.
The hitting time of an adapted process is a stopping time (we use the discrete time version of that result here, not the full Début theorem).
\end{proof}


\begin{lemma}\label{lem:A_hittingAGT_le}
  \uses{def:hittingAGT}
$A^n_{\tau_n(c)} \le c$ and if $\tau_n(c) < T$ then $A^n_{\tau_n(c)+T2^{-n}} > c$.
\end{lemma}

\begin{proof}

\end{proof}


\begin{lemma}\label{lem:A_hittingAGT_sub_ge}
  \uses{def:hittingAGT}
Let $a, b > 0$ with $a \le b$. If $\tau_n(b) < T$ then $A^n_{\tau_n(b)+T2^{-n}} - A^n_{\tau_n(a)} \ge b - a$.
\end{lemma}

\begin{proof}
  \uses{lem:A_hittingAGT_le}

\end{proof}


\begin{lemma}\label{lem:A_uniform_integrable}
  \uses{def:A}
  The sequence $(A^n_T)_{n\in\mathbb{N}}$ is uniformly integrable (bounded in $L^1$ norm).
\end{lemma}

\begin{proof}
  \uses{lem:Doob_Meyer_Finite_Predictable,lem:Predict_Part_Increasing,lem:Doob_Meyer_Finite_Martingale,lem:IsStoppingTime_hittingAGT,lem:A_hittingAGT_sub_ge}
  WLOG $S_T=0$ and $S_t\leq 0$ (else consider $S_t-\mathbb{E}\left[S_T\vert\mathcal{F}_{t}\right]$).

  We have that $0=S_T=M^n_T+A^n_T$. Thus
  \begin{equation}\label{equation_DM_e1}
  M^n_T=-A^n_T.
  \end{equation}
  Since $M^n$ is a martingale it follows by optional sampling that for any $(\mathcal{F}_t)_{t\in\mathcal{D}_n}$ stopping time $\tau$
  \begin{equation}\label{equation_DM_e2}
  S_\tau=M^n_\tau+A^n_\tau = \mathbb{E}[M^n_T\vert\mathcal{F}_\tau]+A^n_\tau\stackrel{\eqref{equation_DM_e1}}{=} -\mathbb{E}[A^n_T\vert\mathcal{F}_\tau]+A^n_\tau.
  \end{equation}
  Let $c>0$. By Lemma~\ref{lem:IsStoppingTime_hittingAGT}, $\tau_n(c)$ (Definition~\ref{def:hittingAGT}) is a stopping time.
  % Define the last time when $A^n$ has always been inside $[0,c]$, by the Début Theorem \ref{thm:hitting_is_stopping_time} and the fact that $A^n$ is predictable the following is a stopping time
  % $$
  % \tau_n(c)=\inf\left(\frac{j-1}{2^n}T\vert\, A^n_{jT2^{-n}}>c\right)\wedge T.
  % $$
  By construction $A^n_{\tau_n(c)}\leq c$. It follows that
  \begin{equation}\label{equation_DM_e3}
  S_{\tau_n(c)}\stackrel{\eqref{equation_DM_e2}}{=}-\mathbb{E}[A^n_T\vert\mathcal{F}_{\tau_n(c)}]+A^n_{\tau_n(c)}\leq -\mathbb{E}[A^n_T\vert\mathcal{F}_{\tau_n(c)}]+c.
  \end{equation}
  Since $(A^n_T>c)=(\tau_n(c)<T)$ we have
  \begin{align}\nonumber
  \int_{(A^n_T>c)}A^n_TdP&=\int_{(\tau_n(c)<T)}A^n_TdP\stackrel{\mathrm{Tower}}{=}\int_{(\tau_n(c)<T)}\mathbb{E}[A^n_T\vert\mathcal{F}_{\tau_n(c)}]dP\\
  &\stackrel{\eqref{equation_DM_e3}}{\leq} cP(\tau_n(c)<T)-\int_{\tau_n(c)<T}S_{\tau_n(c)}dP.\label{equation_DM_e4}
  \end{align}
  Now we notice that $(\tau_n(c)<T)\subseteq (\tau_n(c/2)<T)$, thus
  \begin{align}\nonumber
  \int_{\tau_n(c/2)<T}-S_{\tau_n(c/2)}dP
  &\stackrel{\eqref{equation_DM_e2}}{=}\int_{(\tau_n(c/2))<T}\mathbb{E}[A^n_T\vert\mathcal{F}_{\tau_n(c/2)}]-A^n_{\tau_n(c/2)}dP \nonumber
  \\
  &\stackrel{\mathrm{Tower}}{=}\int_{(\tau_n(c/2)<T)}A^n_t-A^n_{\tau_n(c/2)}dP\nonumber
  \\
  &\geq \int_{(\tau_n(c)<T)}A^n_t-A^n_{\tau_n(c/2)}dP\nonumber
  \\
  \intertext{(over the event $(\tau_n(c)<T)$ $A^n_T\geq c$ and $A^n_{\tau_n(c/2)}\leq c/2$, thus $A^n_T-A^n_{\tau_n(c/2)}\geq c/2$)}
  &\geq \frac{c}{2}P(\tau_n(c)<T).\label{equation_DM_e5}
  \end{align}
  It follows
  $$
  \int_{(A^n_T>c)}A^n_TdP\stackrel{\eqref{equation_DM_e4}}{\leq}cP(\tau_n(c)<T)-\int_{\tau_n(c)<T}S_{\tau_n(c)}dP\stackrel{\eqref{equation_DM_e5}}{\leq}-2\int_{\tau_n(c/2)<T}S_{\tau_n(c/2)}dP-\int_{\tau_n(c)<T}S_{\tau_n(c)}dP.
  $$
  We may notice that
  $$
  P(\tau_n(c)<T)=P(A^n_T>c)\stackrel{Markov}{\leq}\frac{\mathbb{E}[A^n_T]}{c}=-\frac{\mathbb{E}[M^n_T]}{c}\stackrel{mg}{=}-\frac{\mathbb{E}[S_0]}{c}
  $$
  which goes to $0$ uniformly in $n$ as $c$ goes to infinity.
  This implies that $\int_{(A^n_T>c)}A^n_TdP$ is uniformly bounded in $n$ due to the fact that $S$ is of class $D$. And so also the $L^1$ norm is uniformly bounded.
\end{proof}


\begin{lemma}\label{lem:M_uniform_integrable}
  The sequence $(M^n_T)_{n\in\mathbb{N}}$ is uniformly integrable (bounded in $L^1$ norm).
\end{lemma}

\begin{proof}
  \uses{lem:A_uniform_integrable}
  $M^n_T=S_T-A^n_T$, also $S$ is of class $D$ and $A^n_T$ is uniformly integrable.
\end{proof}


\begin{lemma}\label{lem:incr_fun_lim_right_cont_limsup_ineq}
  If $f_n, f : [0, 1] \rightarrow \mathbb{R}$ are increasing functions such that $f$ is right continuous and
  $\lim_n f_n(t) = f (t)$ for $t \in\mathcal{D}^T$, then  $\limsup_n  f_n(t) \leq f (t)$ for all $t \in [0, T]$.
\end{lemma}

\begin{proof}
  Let $t\in[0,T]$ and $s\in\mathcal{D}^T$ such that $t<s$. We have
  $$
  \limsup_n f_n(t)\leq \limsup_n f_n(s)=f(s).
  $$
  Since the above is true uniformly in $s$ in particular since $f$ is right-continuous
  $$
  \limsup_n f_n(t)\leq\lim_{\stackrel{s\rightarrow t^+}{s\in\mathcal{D}^T}}f(s)=f(t).
  $$
\end{proof}


\begin{lemma}\label{lem:incr_fun_lim_right_cont_lim_eq}
  If $f_n, f : [0, 1] \rightarrow \mathbb{R}$ are increasing functions such that $f$ is right continuous and
  $\lim_n f_n(t) = f (t)$ for $t \in\mathcal{D^T}$, if $f$ is continuous in $t\in[0,T]$ then $\lim_n  f_n(t) = f (t)$.
\end{lemma}

\begin{proof}
  \uses{lem:incr_fun_lim_right_cont_limsup_ineq}
  By lemma \ref{lem:incr_fun_lim_right_cont_limsup_ineq} it is enough to show that $\liminf_n f_n(t)\geq f(t)$.
  Let $s\in\mathcal{D}^T$ such that $t>s$. We have
  $$
  \liminf_n f_n(t)\geq \liminf_n f_n(s)=f(s).
  $$
  Since the above is true uniformly in $s$ in particular since $f$ is continuous in $t$
  $$
  \liminf_n f_n(t)\geq\lim_{\stackrel{s\rightarrow t^-}{s\in\mathcal{D}^T}}f(s)=f(t).
$$
\end{proof}

Define $M^n_t$ on $[0,T]$ using $M^n_t=\mathbb{E}[M^n_T\vert\mathcal{F}_t]$.

\begin{lemma}\label{lem:M_n_cadlag_mg}
  $M^n_t$ admits a modification which is a cadlag martingale.
\end{lemma}

\begin{proof}
  \uses{lem:mg_is_cadlag}
  By theorem \ref{lem:mg_is_cadlag}
\end{proof}

From this point onwards $M^n_t$ will be redefined as the modification from lemma \ref{lem:M_n_cadlag_mg}.

\begin{lemma}\label{lem:M_cal_converges_L1}
  There are convex weights $\lambda^n_n,\cdots,\lambda^n_{N_n}$ such that
  $\mathcal{M}^n_T\stackrel{L^1}{\rightarrow}M$, where $\mathcal{M}^n:=\lambda^n_nM^n+\cdots +\lambda^n_{N_n}M^{N_n}.$
\end{lemma}
\begin{proof}
  \uses{lem:M_uniform_integrable,lem:komlos}
  By lemma \ref{lem:M_uniform_integrable} $(M^n_T)_{n\in\mathbb{N}}$ is uniformly bounded in $L^1$, thus by lemma \ref{lem:komlos} there are convex weights $\lambda^n_n,\cdots,\lambda^n_{N_n}$ such that
  $\mathcal{M}^n_T\stackrel{L^1}{\rightarrow}M$, where $\mathcal{M}^n:=\lambda^n_nM^n+\cdots +\lambda^n_{N_n}M^{N_n}.$
\end{proof}

\begin{lemma}\label{lem:M_cal_cadlag}
  $\mathcal{M}^n$ is cadlag.
\end{lemma}
\begin{proof}
  \uses{lem:M_n_cadlag_mg,lem:M_cal_converges_L1}
  By construction and \ref{lem:M_n_cadlag_mg}
\end{proof}

Let \begin{equation}\label{equation_DM_e6} M_t = \mathbb{E}[M\vert\mathcal{F}_t].\end{equation}

\begin{lemma}\label{lem:M_cadlag_mg}
  $M_t$ admits a martingale cadlag modification.
\end{lemma}
\begin{proof}
  \uses{lem:M_cal_converges_L1, lem:mg_is_cadlag}
  By construction $M_t$ is a martingale and thus by theorem \ref{lem:mg_is_cadlag} admits a cadlag martingale modification
  ($M_t$ is a version of $\mathbb{E}[M\vert\mathcal{F}_t]$ and thus passing to modification does not pose any problem).
\end{proof}

From this point onwards $M^n_t$ will be redefined as the modification from lemma \ref{lem:M_cadlag_mg}.
Define
\begin{itemize}
  \item Extend now $A^n$ as a left continuous process $A^n_s:=\sum_{t\in\mathcal{D}^T_n}A^n_t\mathbb{1}_{]t-2^{-n},t]}(s)$
  \item $\mathcal{A}^n=\lambda^n_nA^n+\cdots +\lambda^n_{N_n}A^{N_n}$
  \item $A_t=S_t-M_t$
\end{itemize}

\begin{lemma}\label{lem:M1_komlos}
  For every $t\in[0,T]$ we have $\mathcal{M}^n_t\stackrel{L^1}{\rightarrow}M_t$.
\end{lemma}
\begin{proof}
  \uses{lem:M_cal_converges_L1}
  We may notice that by Jensen's inequality, the tower lemma and lemma \ref{lem:M_cal_converges_L1}
  \begin{gather}\nonumber
    \mathbb{E}[|\mathcal{M}^n_t-M_t|]=\mathbb{E}[|\mathbb{E}[\mathcal{M}^n_T-M\vert\mathcal{F}_t]|]\leq \mathbb{E}[|\mathcal{M}^n_T-M|]\rightarrow0,\\
    \Rightarrow\mathcal{M}^n_t\stackrel{L^1}{\rightarrow} M_t,\quad \forall t\in[0,T].\label{equation_DM_e7}
  \end{gather}
\end{proof}

\begin{lemma}\label{lem:A_cal_conv_A_on_D_T}
  There exists a set $E\subseteq\Omega$, $P(E)=0$ and a subsequence $k_n$ such that $\lim_n\mathcal{A}^{k_n}_t(\omega)=A_t(\omega)$ for every $t\in\mathcal{D}^T,\omega\in\Omega\setminus E$.
\end{lemma}
\begin{proof}
  \uses{lem:M1_komlos}
  By Lemma \ref{lem:M1_komlos}
  $$
  \mathcal{A}^n_t=S_t-\mathcal{M}^n_t\stackrel{L^1}{\rightarrow}S_t-M_t=A_t,\quad\forall t\in\mathcal{D}^T.
  $$
  $\mathcal{D}^T$ is countable we can arrange the elements as $(t_n)_{n\in\mathbb{N}}$.
  Given $t_0\in\mathcal{D}^T$ there exists a subsequence $k^{0}_n$ for which $\mathcal{A}^{k^{0}_n}_{t_0}$ converges to $A_{t_0}$ over the set $\Omega\setminus E_{0}$ where $P(E_{0})=0$.
  Suppose we have a sequence $k^m_n$ for which $\mathcal{A}^{k^j_n}_{t_j}$ converges to $A_{t_j}$ over the set $\Omega\setminus E_{m}$ where $P(E_{m})=0$ for each $j=0,\cdots,m$.
  From this subsequence we may extract a new subsequence $k^{m+1}_n$ for which $\mathcal{A}^{k^{m+1}_n}_{t_{m+1}}$ converges to $A_{t_{m+1}}$ over the set $\Omega\setminus E_{m+1}$ where $P(E_{m+1})=0$.
  By construction over this subsequence the convergence for $t_0,\cdots,t_m$ still applies.
  With a diagonal argument we obtain the final result with $E=\bigcup_n E_n$.
\end{proof}

\begin{lemma}\label{lem:A_increasing}
  $(A_t)_{t\in[0,T]}$ is an increasing process.
\end{lemma}
\begin{proof}
  \uses{lem:A_cal_conv_A_on_D_T, lem:Predict_Part_Increasing, lem:M_cadlag_mg}
  Since $\mathcal{A}^n_t$ is increasing on $\mathcal{D}^T$ by lemma \ref{lem:A_cal_conv_A_on_D_T} also $A$ is almost surely increasing on $\mathcal{D}^T$.
  Since $S,M$ are cadlag also $A$ is cadlag (thus right-continuous). It follows that $A$ must be increasing on $[0,T]$.
\end{proof}

\begin{lemma}\label{lem:lim_Exp_A_n_tau_is_Exp_A_tau}
  Let $\tau$ be an $(\mathcal{F}_t)_{t\in[0,T]}$ stopping time. We have $\lim_n\mathbb{E}[A^n_\tau]=\mathbb{E}[A_\tau]$.
\end{lemma}
\begin{proof}
  \uses{lem:M_cadlag_mg, lem:M_n_cadlag_mg}
  Let $\sigma_n:=\inf\left(t\in\mathcal{D}^T_n\vert t>\tau\right)$. By construction of $A^n$ we have $A^n_\tau=A^n_{\sigma_n}$.
  Also $\sigma_n\searrow\tau$. Since $S$ is of class $D$ and cadlag we have
  \begin{align*}
    \mathbb{E}[A^n_\tau]&=\mathbb{E}[A^n_{\sigma_n}]=\mathbb{E}[S_{\sigma_n}]-\mathbb{E}[M^n_{\sigma_n}]=\mathbb{E}[S_{\sigma_n}]-\mathbb{E}[M^n_0]=\\
    &=\mathbb{E}[S_{\sigma_n}]-\mathbb{E}[S_0]\rightarrow \mathbb{E}[S_\tau]-\mathbb{E}[M_0]=\mathbb{E}[S_\tau]-\mathbb{E}[M_\tau]=\mathbb{E}[A_\tau].
  \end{align*}
\end{proof}

\begin{lemma}\label{lem:limsup_A_n_tau_is_A_tau_ae}
  Let $\tau$ be an $(\mathcal{F}_t)_{t\in[0,T]}$ stopping time. We have $\limsup_n \mathcal{A}_\tau^n = A_\tau$.
\end{lemma}
\begin{proof}
  \uses{lem:lim_Exp_A_n_tau_is_Exp_A_tau, lem:incr_fun_lim_right_cont_limsup_ineq, lem:Predict_Part_Increasing, lem:A_cal_conv_A_on_D_T}
  Firstly we notice that $\liminf_n \mathbb{E}[A_\tau^n]  \leq \limsup_n  \mathbb{E}  [\mathcal{A}_\tau^n  ]  \leq \mathbb{E}[\limsup_n  \mathcal{A}_\tau^n  ]  \leq \mathbb{E}[ A_\tau ]$,
  where the first inequality is justified by the definition of limsup and liminf and the fact that
  $$
  \sup_{k\geq n}\mathbb{E}[\mathcal{A}^k_\tau]\geq \sum_{m=k}^{N_k}\lambda^k_m\mathbb{E}[A^m_\tau]\geq \sum_{m=k}^{N_k}\lambda^k_m\inf_{j\geq n}\mathbb{E}[A^j_\tau]=\inf_{k\geq n}\mathbb{E}[A^k_\tau]
  $$
  the third inequality by \ref{lem:incr_fun_lim_right_cont_limsup_ineq}.
  Let's prove the second inequality: observe that
  $$
  \mathcal{A}^n_\tau= A_1+\mathcal{A}^n_\tau-A_1\leq A_1+(\mathcal{A}^n_\tau-A_1)_+,
  $$
  thus it follows that $\mathcal{A}^n_\tau - (\mathcal{A}^n_\tau-A_1)_+\leq A_1$; since $A_1$ is an integrable guardian the inverse Fatou Lemma may be applied to show together with limsup properties that
  \begin{align*}
    \limsup_n\mathbb{E}[\mathcal{A}^n_\tau]+0 &= \limsup_n\mathbb{E}[\mathcal{A}^n_\tau]+\liminf_n-\mathbb{E}[(\mathcal{A}^n_\tau-A_1)_+] \leq \limsup_n\mathbb{E}[\mathcal{A}^n_\tau-(\mathcal{A}^n_\tau-A_1)_+]\leq\\
    &\leq \mathbb{E}[\limsup_n\mathcal{A}^n_\tau-(\mathcal{A}^n_\tau-A_1)_+]\leq \mathbb{E}[\limsup_n\mathcal{A}^n_\tau]-\mathbb{E}[\liminf_n(\mathcal{A}^n_\tau-A_1)_+]\leq\mathbb{E}[\limsup_n\mathcal{A}^n_\tau],
    \end{align*}
  where the first equality is justified by the fact that $\mathcal{A}^n_\tau\leq\mathcal{A}^n_1\rightarrow A_1$ almost surely.
  Due to lemma \ref{lem:lim_Exp_A_n_tau_is_Exp_A_tau} and \ref{lem:incr_fun_lim_right_cont_limsup_ineq} the first sequence of inequalities is a sequence of equalities, thus
  we know that $A_\tau- \limsup_n \mathcal{A}_\tau^n $ is an a.s. nonnegative function with null expected value, and thus it must be almost everywhere null.
\end{proof}

\begin{theorem}\label{thm:Doob_Meyer}
  Let $S = (S_t )_{0\leq t\leq T}$ be a cadlag submartingale of class $D$.
  Then, $S$ can be written in a unique way in the form  $S = M + A$ where $M$ is a cadlag martingale and $A$ is a predictable increasing process starting at $0$.
\end{theorem}
\begin{proof}
  \uses{lem:A_increasing, lem:M_cadlag_mg, lem:limsup_A_n_tau_is_A_tau_ae, lem:incr_fun_lim_right_cont_lim_eq}
  By construction $M$ is a cadlag martingale and $A_0=0$ and by lemma \ref{lem:A_increasing} $A$ is increasing. It suffices to show that $A$ is predictable.
  $A^n,\mathcal{A}^n$ are left continuous and adapted, and thus they are predictable (measurable wrt the predictable sigma algebra (the one generated by left-cont adapted processes)).
  It is enough to show that $\omega-a.e.$, $\forall t\in[0,T]$, $\limsup_n\mathcal{A}^n_t(\omega)=A_t(\omega)$.

  By lemma \ref{lem:incr_fun_lim_right_cont_lim_eq} that is true for any continuity point of $A$. Since $A$ is increasing it can only have a finite amount of jumps larger than $1/k$ for any $k\in\mathbb{N}$.
  Consider now $\tau_{q,k}$ the family of stopping times equal to the $q$-th time that the process $A_t$ has a jump higher than $1/k$. This is a countable family.
  Given a time $t$ and a trajectory $\omega$ there are only two possibilities: either $A$ is continuous or not at time $t$ along $\omega$.
  If $A$ is continuous at time $t$ we have $\limsup_n\mathcal{A}^n_t(\omega)=A_t(\omega)$, if it jumps there exists $q(\omega),k(\omega)$ such that $t=\tau_{q(\omega),k(\omega)}(\omega)$.
  Due to lemma \ref{lem:limsup_A_n_tau_is_A_tau_ae} we know that $\limsup_n A^n_{\tau_{q,k}} = A_{\tau_{q,k}}$ for each $q,k$ almost surely. Thus, since it is an intersection of a countable amount of almost sure
  events $\forall\omega\in\Omega'$ with $P(\Omega')=1$, for each $q,k$ $\limsup_n A^n_{\tau_{q,k}}(\omega) = A_{\tau_{q,k}}(\omega)$ ($\omega$ does not depend upon $q,k$).
  Consequently, $\forall\omega\in\Omega'$ we have $\limsup_n\mathcal{A}^n_t(\omega)=\limsup_n\mathcal{A}^n_{\tau_{q(\omega),k(\omega)}}(\omega)=A_{\tau_{q(\omega),k(\omega)}}(\omega)=A_t(\omega)$
\end{proof}



\section{Local version of the Doob-Meyer decomposition}



\begin{lemma}\label{lem:IsLocalSubmartingale.local_doobMeyerClass}
  \uses{def:IsLocalSubmartingale, def:Doob_Meyer_class}
Every local submartingale $X$ with $X_0 = 0$ is locally of class D.
\end{lemma}

\begin{proof}
  \uses{lem:local_induction}
By Lemma~\ref{lem:local_induction}, it suffices to show that if $X$ is a submartingale with $X_0=0$ then it is locally of class D.

TODO
\end{proof}


\begin{theorem}[Doob-Meyer decomposition]\label{thm:local_doobMeyer}
  \uses{def:IsLocalSubmartingale, def:predictable, def:IsLocalMartingale}
An adapted process $X$ is a cadlag local submartingale iff $X = M + A$ where $M$ is a cadlag local martingale and $A$ is a predictable, cadlag, locally integrable and increasing process starting at $0$.
The processes $M$ and $A$ are uniquely determined by $X$ a.s.
\end{theorem}

\begin{proof}
  \uses{lem:IsLocalSubmartingale.local_doobMeyerClass, thm:Doob_Meyer}

\end{proof}

\chapter{Stochastic integral}

The lecture notes at \href{https://dec41.user.srcf.net/h/III_L/stochastic_calculus_and_applications/}{this link} are a good reference for this chapter.

\section{Total variation and Lebesgue-Stieltjes integral}

TODO: in Mathlib, we can integrate with respect to the measure given by a right-continuous monotone function (\texttt{StieltjesFunction.measure}). This will be useful to integrate against the quadratic variation of a local martingale.
However, we will also want to integrate with respect to a signed measure given by a càdlàg function with finite variation.
We need new definitions for that one, starting with the definition of the total variation of a function.


\section{Local martingales}

TODO: filtrations should be assumed right-continuous and complete whenever needed.

TODO: this section follows Kallenberg's book and uses $\mathbb{R}_+$ as the time index.
Some of the definitions and results could possibly be generalized.

In this section, $E$ denotes a complete normed space.

First, recall the definitions of a martingale, a stopping time and a stopped process, which are already in Mathlib.


\begin{definition}\label{def:Martingale}
  \mathlibok
  \lean{MeasureTheory.Martingale}
Let $\mathcal{F}$ be a filtration on a measurable space $\Omega$ with measure $P$ indexed by $T$.
A family of functions $M : T \to \Omega \to E$ is a martingale with respect to a filtration $\mathcal{F}$ if $M$ is adapted with respect to $\mathcal{F}$ and for all $i \le j$, $P[M_j \mid \mathcal{F}_i] = M_i$ almost surely.
\end{definition}


\begin{definition}\label{def:IsStoppingTime}
  \mathlibok
  \lean{MeasureTheory.IsStoppingTime}
A stopping time with respect to some filtration $\mathcal{F}$ indexed by $T$ is a function $\tau : \Omega \to T$ such that for all $i$, the preimage of $\{j \mid j \le i\}$ along $\tau$ is measurable with respect to $\mathcal{F}_i$.
\end{definition}


\begin{definition}\label{def:stoppedProcess}
  \mathlibok
  \lean{MeasureTheory.stoppedProcess}
Let $X : T \to \Omega \to E$ be a stochastic process and let $\tau : \Omega \to T$.
The stopped process with respect to $\tau$ is defined by
\begin{align*}
  (X^{\tau})_t = \begin{cases}
    X_t & \text{if } t \le \tau \\
    X_{\tau} & \text{otherwise}
  \end{cases}
\end{align*}
\end{definition}


\begin{definition}\label{def:localMartingale}
  \uses{def:Martingale, def:IsStoppingTime, def:stoppedProcess}
Let $\mathcal{F} = (\mathcal{F}_t)_{t \in \mathbb{R}_+}$ be a filtration on a measurable space $\Omega$.
A local martingale with respect to $\mathcal{F}$ is a stochastic process $M : \mathbb{R}_+ \to \Omega \to E$ adapted to $\mathcal{F}$ such that there exists a localizing sequence $(\tau_n)_{n \in \mathbb{N}}$ such that the following conditions hold:
\begin{itemize}
  \item $\tau_n$ is a stopping time for every $n \in \mathbb{N}$,
  \item $\tau_n$ is non-decreasing and $\tau_n \to \infty$ as $n \to \infty$ (a.s.),
  \item for all $n \in \mathbb{N}$, the stopped and centered process $M^{\tau_n} - M_0$ is a martingale with respect to $\mathcal{F}$.
\end{itemize}
\end{definition}


\begin{definition}\label{def:quadraticVariation}
  \uses{def:localMartingale}
For any continuous local martingale $M$, there exists a continuous process $[M]$ with $[M]_0 = 0$ such that $M^2 - [M]$ is a local martingale. That process is a.s. unique and is called the \emph{quadratic variation} of $M$.
\end{definition}


\begin{definition}\label{def:covariation}
  \uses{def:localMartingale}
For any continuous local martingales $M$ and $N$, there exists a continuous process $[M,N]$ with $[M,N]_0 = 0$ such that $MN - [M,N]$ is a local martingale. That process is a.s. unique and is called the \emph{covariation} of $M$ and $N$.
\end{definition}




\section{Stochastic integral}


\begin{definition}\label{def:predictableStepProcess}
  \uses{def:IsStoppingTime}
Let $(\tau_n)_{n \in \mathbb{N}}$ be a sequence of stopping times which is a.s. non-decreasing and such that $\tau_n \to \infty$ as $n \to \infty$.
Let $(\eta_n)_{n \in \mathbb{N}}$ be a sequence of $\mathcal{F}_{\tau_n}$-measurable random variables.
Then the predictable step process for that sequence is the process $V : \mathbb{R}_+ \to \Omega \to E$ defined by
\begin{align*}
  V_t = \sum_{n=0}^\infty \eta_n \mathbb{1}_{(\tau_n, \tau_{n+1}]}(t)
  \: .
\end{align*}
\end{definition}


\begin{definition}\label{def:elementaryStochasticIntegral}
  \uses{def:predictableStepProcess}
Let $V$ be a predictable step process and let $X$ be a stochastic process.
The \emph{elementary stochastic integral} process $V \cdot X : \mathbb{R}_+ \to \Omega \to E$ is defined by
\begin{align*}
  (V \cdot X)_t
  &= \sum_{n=0}^\infty \eta_n (X^t_{\tau_{n+1}} - X^t_{\tau_n})
  \: .
\end{align*}
\end{definition}


\bibliographystyle{amsalpha}
\bibliography{bib}
