% In this file you should put the actual content of the blueprint.
% It will be used both by the web and the print version.
% It should *not* include the \begin{document}
%
% If you want to split the blueprint content into several files then
% the current file can be a simple sequence of \input. Otherwise It
% can start with a \section or \chapter for instance.

\paragraph{Abstract}

Our goal is to formalize Brownian motions in Lean using \texttt{Mathlib}.
There are two main parts to this formalization:
\begin{itemize}
  \item develop the theory of Gaussian distributions and build a projective family of Gaussian distributions and define its projective limit by the Kolmogorov extension theorem,
  \item prove the Kolmogorov-Chentsov continuity theorem.
\end{itemize}

\paragraph{Notation}

$T$ denotes an index set (for a stochastic process).

$\Omega$ is a measurable space. $\mathbb{P}$ is the probability measure on $\Omega$ and $\mathbb{E}$ is the expectation operator with respect to $\mathbb{P}$.

$E, F$ are normed vector spaces. Most often $F$ is a Banach space and $E$ is a Hilbert space.

\chapter{Characteristic function and covariance}

\section{Characteristic functions}
\label{sec:characteristic_function}


\begin{definition}[Characteristic function]\label{def:charFunDual}
  \mathlibok
  \lean{MeasureTheory.charFunDual}
The characteristic function of a measure $\mu$ on a normed space $E$ is the function $E^* \to \mathbb{C}$ defined by
\begin{align*}
  \hat{\mu}(L) = \int_E e^{i L(x)} \: d\mu(x) \: .
\end{align*}
\end{definition}


\begin{theorem}\label{thm:ext_of_charFunDual}
  \uses{def:charFunDual}
  \mathlibok
  \lean{MeasureTheory.Measure.ext_of_charFunDual}
In a separable Banach space, if two finite measures have same characteristic function, they are equal.
\end{theorem}

\begin{proof}\leanok

\end{proof}


\begin{definition}[Characteristic function]\label{def:charFun}
  \mathlibok
  \lean{MeasureTheory.charFun}
The characteristic function of a measure $\mu$ on an inner product space $E$ is the function $E \to \mathbb{C}$ defined by
\begin{align*}
  \hat{\mu}(t) = \int_E e^{i \langle t, x \rangle} \: d\mu(x) \: .
\end{align*}
This is equal to the normed space version of the characteristic function applied to the linear map $x \mapsto \langle t, x \rangle$.
\end{definition}


\begin{theorem}\label{thm:ext_of_charFun}
  \uses{def:charFun}
  \mathlibok
  \lean{MeasureTheory.Measure.ext_of_charFun}
In a separable Hilbert space, if two finite measures have same characteristic function, they are equal.
\end{theorem}

\begin{proof}\leanok

\end{proof}


\begin{lemma}\label{lem:charFun_map_eq_charFunDual_smul}
  \uses{def:charFun, def:charFunDual}
  \mathlibok
  \lean{MeasureTheory.charFun_map_eq_charFunDual_smul}
Let $\mu$ be a measure on $F$ and let $L \in F^*$. Then
\begin{align*}
  \widehat{L_*\mu}(x) &= \hat{\mu}(x \cdot L) \: .
\end{align*}
\end{lemma}

\begin{proof}\leanok

\end{proof}


\begin{lemma}\label{lem:charFunDual_map}
  \uses{def:charFunDual}
  \mathlibok
  \lean{MeasureTheory.charFunDual_map}
Let $\mu$ be a measure on a normed space $E$ and let $L$ be a continuous linear map from $E$ to $F$.
Then for all $L' \in F^*$,
\begin{align*}
  \widehat{L_*\mu}(L') = \hat{\mu}(L' \circ L) \: .
\end{align*}
\end{lemma}

\begin{proof}\leanok

\end{proof}



\section{Covariance}
\label{sec:covariance}

Let $F$ be a Banach space and $E$ be a Hilbert space.

\begin{definition}[Covariance]\label{def:covarianceBilin}
  \mathlibok
  \lean{ProbabilityTheory.covarianceBilin, ProbabilityTheory.covarianceBilin_apply, ProbabilityTheory.covarianceBilin_apply'}
The covariance bilinear form of a measure $\mu$ on $F$ with finite second moment is the continuous bilinear form $C_\mu : F^* \times F^* \to \mathbb{R}$ with
\begin{align*}
  C_\mu(L_1, L_2)
  &= \int_x (L_1(x) - L_1(m_\mu)) (L_2(x) - L_2(m_\mu)) \: d\mu(x)
  \\
  &= \int_x L_1(x - m_\mu) L_2(x- m_\mu) \: d\mu(x)
  \: .
\end{align*}
\end{definition}

\begin{lemma}\label{lem:covarianceBilin_same_eq_variance}
  \uses{def:covarianceBilin}
  \mathlibok
  \lean{ProbabilityTheory.covarianceBilin_same_eq_variance}
For $\mu$ a measure on $F$ with finite second moment and $L \in F^*$, $C_\mu(L, L) = \mathbb{V}_\mu[L]$.
\end{lemma}

\begin{proof}\leanok

\end{proof}


\begin{definition}[Covariance in a Hilbert space]\label{def:covInnerBilin}
  \leanok
  \lean{ProbabilityTheory.covInnerBilin}
The covariance bilinear form of a finite measure $\mu$ with finite second moment on a Hilbert space $E$ is the continuous bilinear form $C_\mu : E \times E \to \mathbb{R}$ with
\begin{align*}
  C'_\mu(x, y) = \int_z \langle x, z - m_\mu \rangle \langle y, z - m_\mu \rangle \: d\mu(z) \: .
\end{align*}
This is $C_\mu$ applied to the linear maps $L_x, L_y \in E^*$ defined by $L_x(z) = \langle x, z \rangle$ and $L_y(z) = \langle y, z \rangle$.
\end{definition}


\begin{lemma}\label{lem:covInnerBilin_map}
  \uses{def:covInnerBilin}
  \leanok
  \lean{ProbabilityTheory.covInnerBilin_map}
Let $E$ and $F$ be two Hilbert spaces with $F$ finite dimensional, $\mu$ a finite measure on $E$ with finite second moment, and $L : E \to F$ a continuous linear map.
Then the covariance bilinear form of the measure $L_*\mu$ is given by
\begin{align*}
  C'_{L_*\mu}(u, v)
  &= C'_\mu(L^\dagger(u), L^\dagger(v))
  \: ,
\end{align*}
in which $L^\dagger : F \to E$ is the adjoint of $L$.
\end{lemma}

\begin{proof}\leanok
\begin{align*}
  C'_{L_*\mu}(u, v)
  &= (L_*\mu)\left[\langle u, x - m_{L_*\mu}\rangle \langle x - m_{L_*\mu}, v \rangle\right]
  \\
  &= \mu\left[\langle u, L(x) - L(m_\mu)\rangle \langle L(x) - L(m_\mu), v \rangle \right]
  \\
  &= \mu\left[\langle L^\dagger(u), x - m_\mu\rangle \langle x - m_\mu, L^\dagger(v) \rangle \right]
  \\
  &= C'_\mu(L^\dagger(u), L^\dagger(v))
  \: .
\end{align*}
\end{proof}


\begin{definition}[Covariance matrix]\label{def:covMatrix}
  \uses{def:IsGaussian, lem:covarianceBilin_same_eq_variance}
  \leanok
  \lean{ProbabilityTheory.covMatrix, ProbabilityTheory.posSemidef_covMatrix}
The covariance matrix of a finite measure $\mu$ with finite second moment on a finite dimensional inner product space $E$ is the positive semidefinite matrix $\Sigma_\mu$ such that for $u, v \in E$,
\begin{align*}
  \langle u, \Sigma_\mu v\rangle = \mu[\langle u, x - m_\mu \rangle \langle x - m_\mu, v \rangle] \: .
\end{align*}
This is the covariance bilinear form $C'_\mu(u, v)$, as a matrix.
\end{definition}


\begin{lemma}\label{lem:covMatrix_map}
  \uses{def:covMatrix}
Let $E$ and $F$ be two finite dimensional inner product spaces, $\mu$ a measure on $E$ with finite second moment, and $L : E \to F$ a continuous linear map.
Then the covariance matrix of the measure $L_*\mu$ has entries
\begin{align*}
  \langle e_i, \Sigma_{L_*\mu} e_j\rangle
  &= \langle L^\dagger(e_i), \Sigma_\mu L^\dagger(e_j)\rangle
  \: ,
\end{align*}
in which $L^\dagger : F \to E$ is the adjoint of $L$.
\end{lemma}

\begin{proof}
  \uses{lem:covInnerBilin_map}

\end{proof}

\chapter{Stochastic processes}
\label{chap:process}

Let $T$ be an index set and $\Omega$ a measurable space, with measure $\mathbb{P}$.
A stochastic process is a function $X : T \to \Omega \to E$, where $E$ is another measurable space, such that for all $t \in T$, $X_t : \Omega \to E$ is $\mathbb{P}$-a.e. measurable.


\begin{definition}[Law of a stochastic process]\label{def:processLaw}
  \leanok
The law of a stochastic process $X$ is the measure on the measurable space $E^T$ obtained by pushing forward the measure $\mathbb{P}$ by the map $\omega \mapsto X(\cdot, \omega)$.
\end{definition}

\textbf{Lean remark}: we don't use a Lean definition for the law, but write the map in full.

\begin{definition}[Modification]\label{def:modification}
  \leanok
We say that a stochastic process $Y$ is a \emph{modification} of another stochastic process $X$ if for all $t \in T$, $Y_t =_{\mathbb{P}\text{-a.e.}} X_t$.
\end{definition}

\textbf{Lean remark}: we don't use a Lean definition for being a modification, but write explicitly the condition $\forall t \in T,\ Y_t =_{\mathbb{P}\text{-a.e.}} X_t$~.

\begin{definition}[Indistinguishable]\label{def:indistinguishable}
  \leanok
We say that a stochastic processes $Y$ is a \emph{indistinguishable} from $X$ if $\mathbb{P}$-a.e., for all $t \in T$, $X_t = Y_t$.
\end{definition}

A summary of the next few lemmas is this:
\begin{itemize}
  \item indistinguishable $\implies$ modification $\implies$ same law,
  \item modification and continuous with $T$ separable $\implies$ indistinguishable.
\end{itemize}


\begin{lemma}\label{lem:Indistinguishable.Modification}
  \uses{def:indistinguishable, def:modification}
  \leanok
  \lean{modification_of_indistinduishable}
If $Y$ is indistinguishable from $X$, then $Y$ is a modification of $X$.
\end{lemma}

\begin{proof}\leanok
Obvious.
\end{proof}


\begin{lemma}\label{lem:map_eq_of_modification}
  \uses{def:modification}
  \leanok
  \lean{finite_distributions_eq}
Let $X, Y : T \to \Omega \to E$ be two stochastic processes that are modifications of each other.
Then for all $t_1, \ldots, t_n \in T$, the random vector $(X_{t_1}, \ldots, X_{t_n})$ has the same distribution as the random vector $(Y_{t_1}, \ldots, Y_{t_n})$.
That is, $X$ and $Y$ have same finite-dimensional distributions.
\end{lemma}

\begin{proof}\leanok
By the modification property, almost surely $X_{t_i} = Y_{t_i}$ for all $i \in [n]$.
Thus the function $\omega \mapsto (X_{t_1}(\omega), \ldots, X_{t_n}(\omega))$ is equal to $\omega \mapsto (Y_{t_1}(\omega), \ldots, Y_{t_n}(\omega))$ almost surely, hence the maps of $\mathbb{P}$ by these two functions are equal.
\end{proof}


\begin{lemma}\label{lem:map_eq_iff}
  \uses{def:processLaw}
  \leanok
  \lean{finite_distributions_eq_iff_same_law}
Let $X, Y : T \to \Omega \to E$ be two stochastic processes.
Then $X$ and $Y$ have same finite-dimensional distributions if and only if they have the same law.
\end{lemma}

\begin{proof}\leanok
TODO: consider the $\pi$-system of cylinder sets.
\end{proof}


\begin{lemma}\label{lem:indistinguishable_of_modification_of_continuous}
  \uses{def:modification, def:indistinguishable}
  \leanok
  \lean{indistinduishable_of_modification}
Let $T$ and $E$ be topological spaces and suppose that $T$ is separable Hausdorff.
Let $X, Y : T \to \Omega \to E$ be two stochastic processes that are modifications of each other and are almost surely continuous.
Then $X$ and $Y$ are indistinguishable.
\end{lemma}

\begin{proof}\leanok
Since $T$ is separable, it has a countable dense subset $D$.
Since $D$ is countable,
\begin{align*}
  (\forall t \in D, \mathbb{P}\text{-a.e.}, X_t = Y_t)
  \iff (\mathbb{P}\text{-a.e.}, \forall t \in D, X_t = Y_t)
\end{align*}
Hence by the modification property we have that almost surely, for all $t \in D$, $X_t = Y_t$.
Then almost surely $X$ and $Y$ are continuous functions which are equal on a dense subset of $T$: those two functions are equal everywhere.
\end{proof}

\chapter{Gaussian distributions}
\label{chap:gaussian}

\section{Gaussian measures}
\label{sec:gaussian_measures}

\subsection{Real Gaussian measures}

\begin{definition}[Real Gaussian measure]\label{def:gaussianReal}
  \mathlibok
  \lean{ProbabilityTheory.gaussianReal}
  The real Gaussian measure with mean $\mu \in \mathbb{R}$ and variance $\sigma^2 > 0$ is the measure on $\mathbb{R}$ with density $\frac{1}{\sqrt{2 \pi \sigma^2}} \exp\left(-\frac{(x - \mu)^2}{2 \sigma^2}\right)$ with respect to the Lebesgue measure.
  The real Gaussian measure with mean $\mu \in \mathbb{R}$ and variance $0$ is the Dirac measure $\delta_\mu$.
  We denote this measure by $\mathcal{N}(\mu, \sigma^2)$.
\end{definition}


\begin{lemma}\label{lem:charFun_gaussianReal}
  \uses{def:gaussianReal, def:charFun}
  \mathlibok
  \lean{ProbabilityTheory.charFun_gaussianReal}
The characteristic function of a real Gaussian measure with mean $\mu$ and variance $\sigma^2$ is given by
$x \mapsto \exp\left(i \mu x - \frac{\sigma^2 x^2}{2}\right)$.
\end{lemma}

\begin{proof}\leanok

\end{proof}


\begin{lemma}\label{lem:centralMoment_two_mul_gaussianReal}
  \uses{def:gaussianReal}
  \leanok
  \lean{ProbabilityTheory.centralMoment_two_mul_gaussianReal}
The central moment of order $2n$ of a real Gaussian measure $\mathcal{N}(\mu, \sigma^2)$ is given by
\begin{align*}
  \mathbb{E}[(X - \mu)^{2n}] = \sigma^{2n} (2n - 1)!! \: ,
\end{align*}
in which $(2n - 1)!! = (2n - 1)(2n - 3) \cdots 3 \cdot 1$ is the double factorial of $2n - 1$.
\end{lemma}

\begin{proof}\leanok
\begin{align*}
	\mathbb{E}[(X - \mu)^{2n}] &= \int_{-\infty}^\infty (x - \mu)^{2n} \frac{1}{\sqrt{2 \pi \sigma^2}} e^{-\frac{(x - \mu)^2}{2 \sigma^2}} \mathrm dx \\
	&= \int_{-\infty}^\infty x^{2n} \frac{1}{\sqrt{2 \pi \sigma^2}} e^{-\frac{x^2}{2 \sigma^2}} \mathrm dx \\
	&= 2 \int_{0}^\infty x^{2n} \frac{1}{\sqrt{2 \pi \sigma^2}} e^{-\frac{x^2}{2 \sigma^2}} \mathrm dx \\
	&= 2 \int_{0}^\infty {\sqrt{2 \sigma^2 x}}^{2n} \frac{1}{\sqrt{2 \pi \sigma^2}} e^{-x)} \frac{\sigma^2}{\sqrt{2 \sigma^2 x'}} \mathrm dx \\
	&= \frac{\sigma^{2n} 2^n}{\sqrt{\pi}} \int_{0}^\infty x^{n - 1/2} e{-x} \mathrm dx \\
	&= \frac{\sigma^{2n} 2^n}{\sqrt{\pi}} \Gamma(n + 1/2) \\
	&= \frac{\sigma^{2n} 2^n}{\Gamma(1/2)} \left( \prod_{k=0}^{n-1} (k + 1/2) \right) \Gamma(1/2) \\
	&= \sigma^{2n} \prod_{k=0}^{n-1} (2k + 1) \\
	&= \sigma^{2n} (2n - 1)!!
\end{align*}
\end{proof}


\subsection{Gaussian measures on a Banach space}

That kind of generality is not needed for this project, but we happen to have results about Gaussian measures on a Banach space in Mathlib, so we will use them.
The main reference for this section is \cite{hairer2009introduction}.

Let $F$ be a separable Banach space.

\begin{definition}[Gaussian measure]\label{def:IsGaussian}
  \uses{def:gaussianReal}
  \mathlibok
  \lean{ProbabilityTheory.IsGaussian}
A measure $\mu$ on $F$ is Gaussian if for every continuous linear form $L \in F^*$, the pushforward measure $L_* \mu$ is a Gaussian measure on $\mathbb{R}$.
\end{definition}


\begin{lemma}\label{lem:IsGaussian.IsProbabilityMeasure}
  \uses{def:IsGaussian}
  \mathlibok
A Gaussian measure is a probability measure.
\end{lemma}

\begin{proof}\leanok

\end{proof}


\begin{theorem}\label{thm:isGaussian_iff_charFunDual_eq}
  \uses{def:IsGaussian, def:charFunDual}
  \mathlibok
  \lean{ProbabilityTheory.isGaussian_iff_charFunDual_eq}
A finite measure $\mu$ on $F$ is Gaussian if and only if for every continuous linear form $L \in F^*$, the characteristic function of $\mu$ at $L$ is
\begin{align*}
  \hat{\mu}(L) = \exp\left(i \mu[L] - \mathbb{V}_\mu[L] / 2\right) \: ,
\end{align*}
in which $\mathbb{V}_\mu[L]$ is the variance of $L$ with respect to $\mu$.
\end{theorem}

\begin{proof}\uses{thm:ext_of_charFunDual, lem:charFun_gaussianReal}\leanok

\end{proof}



\paragraph{Transformations of Gaussian measures}

\begin{lemma}\label{lem:isGaussian_map}
  \uses{def:IsGaussian}
  \mathlibok
  \lean{ProbabilityTheory.isGaussian_map}
Let $F, G$ be two Banach spaces, let $\mu$ be a Gaussian measure on $F$ and let $T : F \to G$ be a continuous linear map.
Then $T_*\mu$ is a Gaussian measure on $G$.
\end{lemma}

\begin{proof}\leanok

\end{proof}


\begin{lemma}\label{lem:isGaussian_add_const}
  \uses{def:IsGaussian}
  \leanok
  % This is an instance without name in the code, hence we don't give a \lean{...}.
Let $\mu$ be a Gaussian measure on $F$ and let $c \in F$.
Then the measure $\mu$ translated by $c$ (the map of $\mu$ by $x \mapsto x + c$) is a Gaussian measure on $F$.
\end{lemma}

\begin{proof}\leanok

\end{proof}


\begin{lemma}\label{lem:isGaussian_conv}
  \uses{def:IsGaussian}
  \mathlibok
  %\lean{ProbabilityTheory.isGaussian_conv} -- need a Mathlib update
The convolution of two Gaussian measures is a Gaussian measure.
\end{lemma}

\begin{proof}\leanok

\end{proof}



\paragraph{Fernique's theorem}


\begin{theorem}\label{thm:exists_integrable_exp_sq_of_map_rotation_eq_self}
  \leanok
  % In a Mathlib PR
Let $\mu$ be a finite measure on $F$ such that $\mu \times \mu$ is invariant under the rotation of angle $-\frac{\pi}{4}$.
Then there exists $C > 0$ such that the function $x \mapsto \exp (C \Vert x \Vert ^ 2)$ is integrable with respect to $\mu$.
\end{theorem}

\begin{proof}\leanok

\end{proof}


\begin{lemma}\label{lem:IsGaussian.map_rotation_eq_self}
  \uses{def:IsGaussian}
  \leanok
  % In a Mathlib PR
For a Gaussian measure $\mu$, $\mu \times \mu$ is invariant by rotation.
\end{lemma}

\begin{proof}\leanok
  \uses{lem:isGaussian_conv}

\end{proof}


\begin{theorem}[Fernique's theorem]\label{thm:IsGaussian.exists_integrable_exp_sq}
  \uses{def:IsGaussian}
  \leanok
  \lean{ProbabilityTheory.IsGaussian.exists_integrable_exp_sq}
For a Gaussian measure, there exists $C > 0$ such that the function $x \mapsto \exp (C \Vert x \Vert ^ 2)$ is integrable.
\end{theorem}

\begin{proof}\leanok
  \uses{thm:isGaussian_iff_charFunDual_eq, lem:IsGaussian.IsProbabilityMeasure, thm:exists_integrable_exp_sq_of_map_rotation_eq_self, lem:IsGaussian.map_rotation_eq_self}

\end{proof}


\begin{lemma}\label{lem:IsGaussian.memLp_id}
  \uses{def:IsGaussian}
  \leanok
  \lean{ProbabilityTheory.IsGaussian.memLp_id}
A Gaussian measure $\mu$ has finite moments of all orders.
In particular, there is a well defined mean $m_\mu := \mu[\mathrm{id}]$, and for all $L \in F^*$, $\mu[L] = L(m_\mu)$.
\end{lemma}

\begin{proof}\leanok
  \uses{thm:IsGaussian.exists_integrable_exp_sq}

\end{proof}

A Gaussian measure has finite second moment by Lemma~\ref{lem:IsGaussian.memLp_id}, hence its covariance bilinear form is well defined.


\subsection{Gaussian measures on a finite dimensional Hilbert space}

We specialize directly from Banach space to finite dimensional Hilbert space since that's what we need in this project, although there are results for Gaussian measures on infinite dimensional Hilbert spaces that would worth stating.

\begin{lemma}\label{lem:isGaussian_iff_charFun_eq}
  \uses{def:IsGaussian, def:charFunDual, def:charFun}
  \leanok
  \lean{ProbabilityTheory.isGaussian_iff_charFun_eq}
A finite measure $\mu$ on a Hilbert space $E$ is Gaussian if and only if for every $t \in E$, the characteristic function of $\mu$ at $t$ is
\begin{align*}
  \hat{\mu}(t) =  \exp\left(i \mu[\langle t, \cdot \rangle] - \mathbb{V}_\mu[\langle t, \cdot \rangle] / 2\right) \: .
\end{align*}
\end{lemma}

\begin{proof}\leanok
  \uses{thm:isGaussian_iff_charFunDual_eq}
By Theorem~\ref{thm:isGaussian_iff_charFunDual_eq}, $\mu$ is Gaussian iff for every continuous linear form $L \in E^*$, the characteristic function of $\mu$ at $L$ is
\begin{align*}
  \hat{\mu}(L) = \exp\left(i \mu[L] - \mathbb{V}_\mu[L] / 2\right) \: .
\end{align*}
Every continuous linear form $L \in E^*$ can be written as $L(x) = \langle t, x \rangle$ for some $t \in E$, hence we have that $\mu$ is Gaussian iff for every $t \in E$,
\begin{align*}
  \hat{\mu}(t) = \exp\left(i \mu[\langle t, \cdot \rangle] - \mathbb{V}_\mu[\langle t, \cdot \rangle] / 2\right) \: .
\end{align*}
\end{proof}

Let $E$ be a separable Hilbert space. We denote by $\langle \cdot, \cdot \rangle$ the inner product on $E$ and by $\Vert \cdot \Vert$ the associated norm.

\begin{lemma}\label{lem:IsGaussian.charFun_eq}
  \uses{def:IsGaussian, def:charFun, def:covInnerBilin}
  \leanok
  \lean{ProbabilityTheory.IsGaussian.charFun_eq}
The characteristic function of a Gaussian measure $\mu$ on $E$ is given by
\begin{align*}
  \hat{\mu}(t) = \exp\left(i \langle t, m_\mu \rangle - \frac{1}{2} C'_\mu(t, t)\right) \: .
\end{align*}
\end{lemma}

\begin{proof}\leanok
  \uses{lem:isGaussian_iff_charFun_eq, lem:IsGaussian.memLp_id, lem:covarianceBilin_same_eq_variance}
By Lemma~\ref{lem:isGaussian_iff_charFun_eq}, for every $t \in E$,
\begin{align*}
  \hat{\mu}(t) = \exp\left(i \mu[\langle t, \cdot \rangle] - \mathbb{V}_\mu[\langle t, \cdot \rangle] / 2\right) \: .
\end{align*}
By Lemma~\ref{lem:IsGaussian.memLp_id}, $\mu$ has finite first moment and $\mu[\langle t, \cdot \rangle] = \langle t, m_\mu \rangle$. By the same lemma, $\mu$ has finite second moment and for any $t$ we have $\mathbb{V}_\mu[\langle t, \cdot\rangle] = C'_\mu(t, t)$.
\end{proof}

\begin{lemma}\label{lem:isGaussian_iff_gaussian_charFun}
  \uses{def:IsGaussian, def:charFun, def:covMatrix}
  \leanok
  \lean{ProbabilityTheory.isGaussian_iff_gaussian_charFun, ProbabilityTheory.gaussian_charFun_congr}
A finite measure $\mu$ on $E$ is Gaussian if and only if there exists $m \in E$ and $C$ positive semidefinite such that for all $t \in E$, the characteristic function of $\mu$ at $t$ is
\begin{align*}
  \hat{\mu}(t) = \exp\left(i \langle t, m \rangle - \frac{1}{2} C(t, t)\right) \: ,
\end{align*}
If that's the case, then $m = m_\mu$ and $C = C'_\mu$.
\end{lemma}

Note that this lemma does not say that there exists a Gaussian measure for any such $m$ and $C$.
We will prove that later.

\begin{proof}\leanok
  \uses{lem:IsGaussian.charFun_eq, lem:charFun_map_eq_charFunDual_smul, thm:ext_of_charFun}
Lemma~\ref{lem:IsGaussian.charFun_eq} states that the characteristic function of a Gaussian measure has the wanted form.

Suppose now that there exists $m \in E$ and $C$ positive semidefinite such that for all $t \in E$, $\hat{\mu}(t) = \exp\left(i \langle t, m \rangle - \frac{1}{2} C(t, t)\right)$.

We need to show that for all $L \in E^*$, $L_*\mu$ is a Gaussian measure on $\mathbb{R}$.
Such an $L$ can be written as $\langle u, \cdot \rangle$ for some $u \in E$.
Let then $u \in E$. We compute the characteristic function of $\langle u, \cdot\rangle_*\mu$ at $x \in \mathbb{R}$ with Lemma~\ref{lem:charFun_map_eq_charFunDual_smul}:
\begin{align*}
  \widehat{\langle u, \cdot\rangle_*\mu}(x)
  &= \hat{\mu}(x \cdot u)
  \\
  &= \exp\left(i x \langle u, m \rangle - \frac{1}{2} x^2 C(u, u)\right)
  \: .
\end{align*}
This is the characteristic function of a Gaussian measure on $\mathbb{R}$ with mean $\langle u, m \rangle$ and variance $C(u, u)$.
By Theorem~\ref{thm:ext_of_charFun}, $\langle u, \cdot\rangle_*\mu$ is Gaussian, hence $\mu$ is Gaussian.

By Lemma~\ref{lem:IsGaussian.charFun_eq}, we deduce that for any $t \in E$ we have
$$\exp\left(i\langle t, m \rangle - \frac{1}{2} C(t, t)\right) = \exp\left(i\langle t, m_\mu \rangle - \frac{1}{2} C'_\mu(t, t)\right).$$
In particular, for any $t$ there exists $n_t \in \mathbb{Z}$ such that
$$i\langle t, m \rangle - \frac{1}{2} C(t, t) = i\langle t, m_\mu \rangle - \frac{1}{2} C'_\mu(t, t) + 2i\pi n_t.$$
We deduce that $n$ is a continuous map from $E$ to $\mathbb{Z}$, and thus must be constant because $E$ is connected. By looking at the value at $t = 0$, we deduce that for any $t$, $n_t = 0$. Looking at real and imaginary parts we obtain that for any $t$,
$$\langle t, m \rangle = \langle t, m_\mu \rangle \quad \text{and} \quad C(t, t) = C'_\mu(t, t).$$
We immediately deduce that $m = m_\mu$. Moreover, because $C$ and $C'_\mu$ are symmetric, they are characterized by their values on the diagonal. Indeed, for any $x, y$,
$$C(x, y) = \frac{1}{2} (C(x + y, x + y) - C(x, x) - C(y, y)).$$
We deduce that $C = C'_\mu$.
\end{proof}

\begin{lemma}\label{lem:IsGaussian.ext_iff}
  \uses{def:IsGaussian, def:covInnerBilin}
  \leanok
  \lean{ProbabilityTheory.IsGaussian.ext, ProbabilityTheory.IsGaussian.ext_iff}
Two Gaussian measures $\mu$ and $\nu$ on a separable Hilbert space are equal if and only if they have same mean and same covariance.
\end{lemma}

\begin{proof}\leanok
  \uses{thm:ext_of_charFun, lem:IsGaussian.charFun_eq}
The forward direction is immediate.

For the converse direction, it is enough to show that $\mu$ and $\nu$ have the same characteristic function by Theorem~\ref{thm:ext_of_charFun}. As they are both Gaussian, their characteristic functions only depend on their mean and covariance by Lemma~\ref{lem:IsGaussian.charFun_eq}. Thus they are equal.
\end{proof}


\begin{definition}[Standard Gaussian measure]\label{def:stdGaussian}
  \uses{def:gaussianReal}
  \leanok
  \lean{ProbabilityTheory.stdGaussian}
Let $(e_1, \ldots, e_d)$ be an orthonormal basis of $E$ and let $\mu$ be the standard Gaussian measure on $\mathbb{R}$.
The standard Gaussian measure on $E$ is the pushforward measure of the product measure $\mu \times \ldots \times \mu$ by the map $x \mapsto \sum_{i=1}^d x_i \cdot e_i$.
\end{definition}

The fact that this definition does not depend on the choice of basis will be a consequence of the fact that its characteristic function does not depend on the basis.


\begin{lemma}\label{lem:integral_eval_pi}
  \leanok
  \lean{ProbabilityTheory.integral_eval_pi}
For $\mu_1, \ldots, \mu_d$ probability measures on $\mathbb{R}$ and $f : \mathbb{R} \to \mathbb{R}$ integrable with respect to $\mu_i$, we have
\begin{align*}
  \int_x f(x_i) \, d(\mu_1 \times \ldots \times \mu_d)(x)
  = \int_x f(x) \, d\mu_i
  \: .
\end{align*}
\end{lemma}

\begin{proof}\leanok
As $f$ is integrable, we can use Fubini theorem to obtain that
$$\int f(x_i) \, d(\mu_1 \times \ldots \times \mu_d)(x) = \int f(x) \, d\mu_i(x) \times \prod_{j \ne i} \int 1 \, d\mu_j(x) = \int f(x) \, d\mu_i(x)$$
because the $\mu_j$s are probability measures.
\end{proof}


\begin{lemma}\label{lem:isCentered_stdGaussian}
  \uses{def:stdGaussian}
  \leanok
  \lean{ProbabilityTheory.isCentered_stdGaussian}
The standard Gaussian measure on $E$ is centered, i.e., $\mu[L] = 0$ for every $L \in E^*$.
\end{lemma}

\begin{proof}\leanok
  \uses{lem:integral_eval_pi}

\end{proof}


\begin{lemma}\label{lem:isProbabilityMeasure_stdGaussian}
  \uses{def:stdGaussian}
  \leanok
  \lean{ProbabilityTheory.isProbabilityMeasure_stdGaussian}
The standard Gaussian measure is a probability measure.
\end{lemma}

\begin{proof}\leanok

\end{proof}


\begin{lemma}\label{lem:charFun_stdGaussian}
  \uses{def:stdGaussian, def:charFun}
  \leanok
  \lean{ProbabilityTheory.charFun_stdGaussian}
The characteristic function of the standard Gaussian measure on $E$ is given by
\begin{align*}
  \hat{\mu}(t) = \exp\left(-\frac{1}{2} \Vert t \Vert^2 \right) \: .
\end{align*}
\end{lemma}

\begin{proof}\leanok
  \uses{lem:charFun_gaussianReal}
Denote by $\nu$ the standard Gaussian measure on $\mathbb{R}$. This is a straightforward computation:
\begin{align*}
  \hat{\mu}(t) = \int \exp\left(i\langle t, \sum_{j=1}^d x_j \cdot e_j \rangle\right) d(\nu \times \ldots \times \nu)(dx) &= \int \exp\left(\sum_{j=1}^d ix_j\langle t, e_j \rangle\right) d(\nu \times \ldots \times \nu)(dx) \\
  &= \int \prod_{j=1}^d \exp\left(ix_j\langle t, e_j \rangle\right) d(\nu \times \ldots \times \nu)(dx) \\
  &= \prod_{j=1}^d \int \exp\left(ix\langle t, e_j \rangle\right) d\nu(x) \\
  &= \prod_{j=1}^d \exp\left(-\frac{\langle t, e_j \rangle^2}{2}\right) \\
  &= \exp\left(-\frac{1}{2} \Vert t \Vert^2 \right).
\end{align*}
\end{proof}


\begin{lemma}\label{lem:isGaussian_stdGaussian}
  \uses{def:stdGaussian, def:IsGaussian}
  \leanok
  \lean{ProbabilityTheory.isGaussian_stdGaussian}
The standard Gaussian measure on $E$ is a Gaussian measure.
\end{lemma}

\begin{proof}\leanok
  \uses{lem:isGaussian_iff_gaussian_charFun, lem:charFun_stdGaussian, lem:isProbabilityMeasure_stdGaussian}
Since the standard Gaussian is a probability measure (hence finite), we can apply Lemma~\ref{lem:isGaussian_iff_gaussian_charFun} that states that it suffices to show that the characteristic function has a particular form.
That form is given by Lemma~\ref{lem:charFun_stdGaussian}, taking $m=0$ and $C = \langle\cdot, \cdot\rangle$.
\end{proof}


\begin{lemma}\label{lem:integral_id_stdGaussian}
  \uses{def:stdGaussian}
  \leanok
  \lean{ProbabilityTheory.integral_id_stdGaussian}
The mean of the standard Gaussian measure is $0$.
\end{lemma}

\begin{proof}\leanok
  \uses{lem:integral_eval_pi}

\end{proof}


\begin{lemma}\label{lem:covMatrix_stdGaussian}
  \uses{def:stdGaussian, def:covMatrix}
  \leanok
  \lean{ProbabilityTheory.covMatrix_stdGaussian}
The covariance matrix of the standard Gaussian measure is the identity matrix.
\end{lemma}

\begin{proof}\leanok
  \uses{lem:isGaussian_iff_gaussian_charFun, lem:charFun_stdGaussian}
From Lemma~\ref{lem:charFun_stdGaussian}, we know that for all $t \in \mathbb{R}$,
$$\hat{\mu}(t) = \exp\left(-\frac{\|t\|^2}{2}\right) = \exp\left(-\frac{\langle t, \mathrm{I}t\rangle}{2}\right).$$
As the identity is positive semidefinite, we deduce from Lemma~\ref{lem:isGaussian_iff_gaussian_charFun} that $\Sigma_\mu$ is the identity matrix.
\end{proof}


\begin{definition}[Multivariate Gaussian]\label{def:multivariateGaussian}
  \uses{def:stdGaussian}
  \leanok
  \lean{ProbabilityTheory.multivariateGaussian}
The multivariate Gaussian measure on $\mathbb{R}^d$ with mean $m \in \mathbb{R}^d$ and covariance matrix $\Sigma \in \mathbb{R}^{d \times d}$, with $\Sigma$ positive semidefinite, is the pushforward measure of the standard Gaussian measure on $\mathbb{R}^d$ by the map $x \mapsto m + \Sigma^{1/2} x$.
We denote this measure by $\mathcal{N}(m, \Sigma)$.
\end{definition}


\begin{lemma}\label{lem:integral_id_multivariateGaussian}
  \uses{def:multivariateGaussian}
  \leanok
  \lean{ProbabilityTheory.integral_id_multivariateGaussian}
The mean of the multivariate Gaussian measure $\mathcal{N}(m, \Sigma)$ is $m$.
\end{lemma}

\begin{proof}\leanok
  \uses{lem:integral_id_stdGaussian}

\end{proof}


\begin{lemma}\label{lem:covMatrix_multivariateGaussian}
  \uses{def:multivariateGaussian}
  \leanok
  \lean{ProbabilityTheory.covInnerBilin_multivariateGaussian}
The covariance matrix of the multivariate Gaussian measure $\mathcal{N}(m, \Sigma)$ is $\Sigma$.
\end{lemma}

\begin{proof}\leanok
  \uses{lem:covMatrix_stdGaussian}

\end{proof}


\begin{lemma}\label{lem:isGaussian_multivariateGaussian}
  \uses{def:multivariateGaussian, def:IsGaussian}
  \leanok
  \lean{ProbabilityTheory.isGaussian_multivariateGaussian}
A multivariate Gaussian measure is a Gaussian measure.
\end{lemma}

\begin{proof}\leanok
  \uses{lem:isGaussian_stdGaussian, lem:isGaussian_add_const, lem:isGaussian_map}
The multivariate Gaussian measure is the pushforward of the standard Gaussian measure by an affine map, and is thus Gaussian by Lemma~\ref{lem:isGaussian_add_const} and Lemma~\ref{lem:isGaussian_map}.
\end{proof}


\begin{theorem}\label{thm:charFun_multivariateGaussian}
  \uses{def:multivariateGaussian, def:charFun}
  \leanok
  \lean{ProbabilityTheory.charFun_multivariateGaussian}
The characteristic function of a multivariate Gaussian measure $\mathcal{N}(m, \Sigma)$ is given by
\begin{align*}
  \hat{\mu}(t) = \exp\left(i \langle m, t \rangle - \frac{1}{2} \langle t, \Sigma t \rangle\right)
  \: .
\end{align*}
\end{theorem}

\begin{proof}\leanok
  \uses{lem:isGaussian_multivariateGaussian, lem:IsGaussian.charFun_eq, lem:integral_id_multivariateGaussian, lem:covMatrix_multivariateGaussian}
Since the multivariate Gaussian measure is a Gaussian measure, we can apply Lemma~\ref{lem:IsGaussian.charFun_eq} to it.
It suffices then to show that the mean and the covariance matrix of the multivariate Gaussian measure are equal to $m$ and $\Sigma$, respectively.
This is given by Lemma~\ref{lem:integral_id_multivariateGaussian} and Lemma~\ref{lem:covMatrix_multivariateGaussian}.
\end{proof}


\section{Gaussian processes}
\label{sec:gaussian_processes}

\begin{definition}[Gaussian process]\label{def:IsGaussianProcess}
  \uses{def:IsGaussian}
  \leanok
  \lean{ProbabilityTheory.IsGaussianProcess}
A process $X : T \to \Omega \to E$ is Gaussian if for every finite subset $t_1, \ldots, t_n \in T$, the random vector $(X_{t_1}, \ldots, X_{t_n})$ has a Gaussian distribution.
\end{definition}


\begin{lemma}\label{lem:isGaussianProcess_of_modification}
  \uses{def:IsGaussianProcess}
  \leanok
  \lean{ProbabilityTheory.IsGaussianProcess.modification}
Let $X, Y : T \to \Omega \to E$ be two stochastic processes that are modifications of each other (that is, for all $t \in T$, $X_t =_{a.e.} Y_t$).
If $X$ is a Gaussian process, then $Y$ is a Gaussian process as well.
\end{lemma}

\begin{proof}
  \uses{lem:map_eq_of_modification}
Being a Gaussian process is defined in terms of the distribution of finite-dimensional random vectors.
By Lemma~\ref{lem:map_eq_of_modification}, the random vector $(Y_{t_1}, \ldots, Y_{t_n})$ has the same distribution as the random vector $(X_{t_1}, \ldots, X_{t_n})$ for all $t_1, \ldots, t_n \in T$.
\end{proof}

\chapter{Projective family of the Brownian motion}
\label{chap:projective_family}


\section{Kolmogorov extension theorem}

This theorem has been formalized in the repository \href{https://github.com/RemyDegenne/kolmogorov_extension4}{kolmogorov\_extension4}.

\begin{definition}[Projective family]\label{def:IsProjectiveMeasureFamily}
  \mathlibok
  \lean{MeasureTheory.IsProjectiveMeasureFamily}
A family of measures $P$ indexed by finite sets of $T$ is projective if, for finite sets $J \subseteq I$, the projection from $E^I$ to $E^J$ maps $P_I$ to $P_J$.
\end{definition}


\begin{definition}[Projective limit]\label{def:IsProjectiveLimit}
  \uses{def:IsProjectiveMeasureFamily}
  \mathlibok
  \lean{MeasureTheory.IsProjectiveLimit}
A measure $\mu$ on $E^T$ is the projective limit of a projective family of measures $P$ indexed by finite sets of $T$ if, for every finite set $I \subseteq T$, the projection from $E^T$ to $E^I$ maps $\mu$ to $P_I$.
\end{definition}


\begin{theorem}[Kolmogorov extension theorem]\label{thm:kolmogorovExtension}
  \uses{def:IsProjectiveLimit, def:IsProjectiveMeasureFamily}
  \leanok
  \lean{MeasureTheory.projectiveLimit, MeasureTheory.IsProjectiveLimit.unique, MeasureTheory.isProjectiveLimit_projectiveLimit, MeasureTheory.isFiniteMeasure_projectiveLimit, MeasureTheory.isProbabilityMeasure_projectiveLimit}
Let $\mathcal{X}$ be a Polish space, equipped with the Borel $\sigma$-algebra, and let $T$ be an index set.
Let $P$ be a projective family of finite measures on $\mathcal{X}$.
Then the projective limit $\mu$ of $P$ exists, is unique, and is a finite measure on $\mathcal{X}^T$.
Moreover, if $P_I$ is a probability measure for every finite set $I \subseteq T$, then $\mu$ is a probability measure.
\end{theorem}

\begin{proof}\leanok

\end{proof}


\section{Projective family of Gaussian measures}

We build a projective family of Gaussian measures indexed by $\mathbb{R}_+$.
In order to do so, we need to define specific Gaussian measures on finite index sets $\{t_1, \ldots, t_n\}$.
We want to build a multivariate Gaussian measure on $\mathbb{R}^n$ with mean $0$ and covariance matrix $C_{ij} = \min(t_i, t_j)$ for $1 \leq i,j \leq n$.

% \paragraph{First method: Gaussian increments}

% In this method, we build the Gaussian measure by adding independent Gaussian increments.

% \begin{definition}(Gaussian increment)\label{def:gaussianIncrement}
% For $v \ge 0$, the map from $\mathbb{R}$ to the probability measures on $\mathbb{R}$ defined by $x \mapsto \mathcal{N}(x, v)$ is a Markov kernel.
% We call that kernel the \emph{Gaussian increment} with variance $v$ and denote it by $\kappa^G_v$.
% \end{definition}

% TODO: perhaps the equality $\mathcal{N}(x, v) = \delta_x \ast \mathcal{N}(0, v)$ is useful to show that it is a kernel?

% \begin{definition}\label{def:gaussianFromIncrements}
%   \uses{def:gaussianIncrement}
% Let $0 \le t_1 \le \ldots \le t_n$ be non-negative reals.
% Let $\mu_0$ be the real Gaussian distribution $\mathcal{N}(0, t_1)$.
% For $i \in \{1, \ldots, n-1\}$, let $\kappa_i$ be the Markov kernel from $\mathbb{R}$ to $\mathbb{R}$ defined by $\kappa_i(x) = \mathcal{N}(x, t_{i+1} - t_i)$ (the Gaussian increment $\kappa^G_{t_{i+1} - t_i}$).
% Let $P_{t_1, \ldots, t_n}$ be the measure on $\mathbb{R}^n$ defined by $\mu_0 \otimes \kappa_1 \otimes \ldots \otimes \kappa_{n-1}$.
% \end{definition}

% TODO: explain the notation $\otimes$ in the lemma above: $\kappa_{n-1}$ takes the value at $n-1$ only to produce the distribution at $n$.

% \begin{lemma}\label{lem:isGaussian_gaussianFromIncrements}
%   \uses{def:gaussianFromIncrements, def:IsGaussian}
% $P_{t_1, \ldots, t_n}$ is a Gaussian measure on $\mathbb{R}^n$ with mean $0$ and covariance matrix $C_{ij} = \min(t_i, t_j)$ for $1 \leq i,j \leq n$.
% \end{lemma}

% \begin{proof}

% \end{proof}


% \paragraph{Second method: covariance matrix}

We prove that the matrix $C_{ij} = \min(t_i, t_j)$ is positive semidefinite, which means that there exists a Gaussian distribution with mean 0 and covariance matrix $C$.

\begin{definition}[Gram matrix]\label{def:gramMatrix}
Let $v_1, \ldots, v_n$ be vectors in an inner product space $E$.
The Gram matrix of $v_1, \ldots, v_n$ is the matrix in $\mathbb{R}^{n \times n}$ with entries $G_{ij} = \langle v_i, v_j \rangle$ for $1 \leq i,j \leq n$.
\end{definition}


\begin{lemma}\label{lem:posSemidef_gramMatrix}
  \uses{def:gramMatrix}
A gram matrix is positive semidefinite.
\end{lemma}

\begin{proof}
Symmetry is obvious from the definition.
Let $x \in E$. Then
\begin{align*}
  \langle x, G x \rangle
  &= \sum_{i,j} x_i x_j \langle v_i, v_j \rangle
  \\
  &= \langle \sum_i x_i v_i, \sum_j x_j v_j \rangle
  \\
  &= \left\Vert \sum_i x_i v_i \right\Vert^2
  \\
  &\ge 0
  \: .
\end{align*}
\end{proof}


\begin{lemma}\label{lem:C_eq_gramMatrix}
  \uses{def:gramMatrix}
Let $I = \{t_1, \ldots, t_n\}$ be a finite subset of $\mathbb{R}_+$.
For $i \le n$, let $v_i = \mathbb{I}_{[0, t_i]}$ be the indicator function of the interval $[0, t_i]$, as an element of $L^2(\mathbb{R})$.
Then the Gram matrix of $v_1, \ldots, v_n$ is equal to the matrix $C_{ij} = \min(t_i, t_j)$ for $1 \leq i,j \leq n$.
\end{lemma}

\begin{proof}
By definition of the inner product in $L^2(\mathbb{R})$,
\begin{align*}
  \langle v_i, v_j \rangle
  &= \int_{\mathbb{R}} \mathbb{I}_{[0, t_i]}(x) \mathbb{I}_{[0, t_j]}(x) \: dx
  = \min(t_i, t_j)
  \: .
\end{align*}
\end{proof}


\begin{lemma}\label{lem:posSemidef_brownianCov}
For $I = \{t_1, \ldots, t_n\}$ a finite subset of $\mathbb{R}_+$, let $C \in \mathbb{R}^{n \times n}$ be the matrix $C_{ij} = \min(t_i, t_j)$ for $1 \leq i,j \leq n$.
Then $C$ is positive semidefinite.
\end{lemma}

\begin{proof}\uses{lem:C_eq_gramMatrix, lem:posSemidef_gramMatrix}
$C$ is a Gram matrix by Lemma~\ref{lem:C_eq_gramMatrix}.
By Lemma~\ref{lem:posSemidef_gramMatrix}, it is positive semidefinite.
\end{proof}


\paragraph{Definition of the projective family and extension}

\begin{definition}[Projective family of the Brownian motion]\label{def:gaussianProjectiveFamily}
  \uses{def:multivariateGaussian, lem:posSemidef_brownianCov}
For $I = \{t_1, \ldots, t_n\}$ a finite subset of $\mathbb{R}_+$, let $P^B_I$ be the multivariate Gaussian measure on $\mathbb{R}^n$ with mean $0$ and covariance matrix $C_{ij} = \min(t_i, t_j)$ for $1 \leq i,j \leq n$.
We call the family of measures $P^B_I$ the \emph{projective family of the Brownian motion}.
\end{definition}


\begin{lemma}\label{lem:isProjectiveMeasureFamily_gaussianProjectiveFamily}
  \uses{def:gaussianProjectiveFamily, def:IsProjectiveMeasureFamily}
The projective family of the Brownian motion is a projective family of measures.
\end{lemma}

\begin{proof}
  \uses{lem:isGaussian_map, lem:isGaussian_multivariateGaussian, lem:covMatrix_map}
Let $J \subseteq I$ be finite subsets of $\mathbb{R}_+$.
We need to show that the restriction from $\mathbb{R}^I$ to $\mathbb{R}^J$ (denote it by $\pi_{IJ}$) maps $P^B_I$ to $P^B_J$.

The restriction is a continuous linear map from $\mathbb{R}^I$ to $\mathbb{R}^J$.
The map of a Gaussian measure by a continuous linear map is Gaussian (Lemma~\ref{lem:isGaussian_map}).
It thus suffices to show that the mean and covariance matrix of the map are equal to the ones of $P^B_J$.

The mean of the map is $0$, since the mean of $P^B_I$ is $0$ and the map is linear.

For the covariance matrix and $i, j \in J$, by Lemma~\ref{lem:covMatrix_map} we have
\begin{align*}
  \langle e_i, \Sigma_{\pi_{IJ*}\mu} e_j\rangle
  &= \langle \pi_{IJ}^\dagger(e_i), \Sigma_\mu \pi_{IJ}^\dagger(e_j)\rangle
  \: .
\end{align*}
$\pi_{IJ}^\dagger(u)$ is the vector of $\mathbb{R}^I$ with coordinates $(\pi_{IJ}^\dagger(u))_i = u_i$ if $i \in J$ and $(\pi_{IJ}^\dagger(u))_i = 0$ otherwise.
This gives the same covariance matrix as the one of $P^B_J$.
\end{proof}


\begin{definition}\label{def:gaussianLimit}
  \uses{thm:kolmogorovExtension, lem:isProjectiveMeasureFamily_gaussianProjectiveFamily}
We denote by TODO the projective limit of the projective family of the Brownian motion given by Theorem~\ref{thm:kolmogorovExtension}.
This is a probability measure on $\mathbb{R}^{\mathbb{R}_+}$.
\end{definition}


% \begin{definition}\label{def:}
% Let $\Omega = \mathbb{R}^{\mathbb{R}_+}$ and consider the probability space $(\Omega, TODO)$.
% The identity on that space is a function $\Omega \to \mathbb{R}_+ \to \mathbb{R}$.
% We can reorder the arguments to define a process $X : \mathbb{R}_+ \to \Omega \to \mathbb{R}$.
% That process is a Gaussian process with covariance function $C(t,s) = \min(t,s)$.
% \end{definition}

\chapter{Kolmogorov-Chentsov Theorem}
\label{chap:kolmogorov_chentsov}

\section{Covers}

\begin{definition}[$\varepsilon$-cover]\label{def:IsCover}
  \leanok
  \lean{IsCover}
  Let $E$ be a set with a distance function $d_E$. Let $\varepsilon \ge 0$.
  A set $C \subseteq E$ is an $\varepsilon$-cover of a set $A \subseteq E$ if for every $x \in A$, there exists $y \in C$ such that $d_E(x, y) < \varepsilon$.
\end{definition}

\begin{definition}[External covering number]\label{def:externalCoveringNumber}
  \uses{def:IsCover}
  \leanok
  \lean{externalCoveringNumber}
  Let $E$ be a set with a distance function $d_E$.
  The external covering number of a set $A \subseteq E$ for $\varepsilon \ge 0$ is the smallest cardinality of an $\varepsilon$-cover of $A$.
\end{definition}

\begin{definition}[Internal covering number]\label{def:internalCoveringNumber}
  \uses{def:IsCover}
  \leanok
  \lean{internalCoveringNumber}
  Let $E$ be a set with a distance function $d_E$.
  The internal covering number of a set $A \subseteq E$ for $\varepsilon \ge 0$ is the smallest cardinality of an $\varepsilon$-cover of $A$ which is a subset of $A$.
\end{definition}

\chapter{Brownian motion}
\label{chap:brownian}


\section{Stochastic process with continuous paths}

\begin{definition}[pre-Brownian process]\label{def:preBrownian}
  \uses{def:gaussianLimit}
  \leanok
  \lean{ProbabilityTheory.preBrownian}
Let $\Omega = \mathbb{R}^{\mathbb{R}_+}$ and consider the probability space $(\Omega, P_B)$ (where $P_B$ is the measure defined in Definition~\ref{def:gaussianLimit}).
The identity on that space is a function $\Omega \to \mathbb{R}_+ \to \mathbb{R}$.
We reorder the arguments to define a stochastic process $X : \mathbb{R}_+ \to \Omega \to \mathbb{R}$, which we call the pre-Brownian process.
\end{definition}


\begin{lemma}\label{lem:isGaussianProcess_preBrownian}
  \uses{def:preBrownian, def:IsGaussianProcess}
  \leanok
  \lean{ProbabilityTheory.isGaussianProcess_preBrownian}
  The pre-Brownian process $X$ of Definition~\ref{def:preBrownian} is a Gaussian process.
\end{lemma}

\begin{proof}\leanok
  \uses{lem:isGaussian_multivariateGaussian}

\end{proof}


\begin{lemma}\label{lem:map_sub_preBrownian}
  \uses{def:preBrownian}
  \leanok
  \lean{ProbabilityTheory.map_sub_preBrownian}
Let $X$ be the pre-Brownian process of Definition~\ref{def:preBrownian}.
Then, for all $s, t \in \mathbb{R}_+$, the random variable $X_t - X_s$ is a Gaussian random variable with mean $0$ and variance $|t - s|$.
\end{lemma}

\begin{proof}

\end{proof}


\begin{lemma}\label{lem:isKolmogorovProcess_preBrownian}
  \uses{def:preBrownian}
  \leanok
  \lean{ProbabilityTheory.isKolmogorovProcess_preBrownian}
The pre-Brownian process $X$ of Definition~\ref{def:preBrownian} satisfies the Kolmogorov condition for exponents $(2n,n)$ with constant $(2n - 1)!!$ for all $n \in \mathbb{N}$.
That is, for all $s, t \in \mathbb{R}_+$, we have
\begin{align*}
  \mathbb{E} \left[ |X_t - X_s|^{2n} \right] \le (2n - 1)!! |t - s|^n
  \: .
\end{align*}
\end{lemma}

\begin{proof}
  \uses{lem:centralMoment_two_mul_gaussianReal, lem:map_sub_preBrownian}
$X_t - X_s$ is a Gaussian random variable with mean $0$ and variance $|t - s|$ (Lemma~\ref{lem:map_sub_preBrownian}).
Thus, by Lemma~\ref{lem:centralMoment_two_mul_gaussianReal}, we have
\begin{align*}
  \mathbb{E} \left[ |X_t - X_s|^{2n} \right]
  = (2n - 1)!! |t - s|^n
  \: .
\end{align*}
\end{proof}


\begin{definition}[Brownian motion]\label{def:brownian}
  \uses{thm:localized_holder_modification_sup, def:preBrownian, lem:isKolmogorovProcess_preBrownian, lem:hasBoundedCoveringNumberCover_nnreal}
  \leanok
By Theorem~\ref{thm:localized_holder_modification_sup}, there exists a modification $B$ of the pre-Brownian process such that all the paths of $B$ are Hölder continuous of all orders $\gamma \in (0, 1/2)$.
We call $B$ the \emph{Brownian motion} on $\mathbb{R}_+$.
\end{definition}


\begin{lemma}\label{lem:isGaussianProcess_brownian}
  \uses{def:brownian, def:IsGaussianProcess}
  \leanok
  \lean{ProbabilityTheory.isGaussianProcess_brownian}
The Brownian motion is a Gaussian process.
\end{lemma}

\begin{proof}\leanok
  \uses{lem:isGaussianProcess_of_modification, lem:isGaussianProcess_preBrownian}
The pre-Brownian process is a Gaussian process by Lemma~\ref{lem:isGaussianProcess_preBrownian}.
The Brownian motion is a modification of the pre-Brownian process by Definition~\ref{def:brownian}.
Thus, the Brownian motion is a Gaussian process as well by Lemma~\ref{lem:isGaussianProcess_of_modification}.
\end{proof}


\begin{lemma}\label{lem:isHolderWith_brownian}
  \uses{def:brownian}
  \leanok
  \lean{ProbabilityTheory.isHolderWith_brownian}
The paths of the Brownian motion are Hölder continuous of all orders $\gamma \in (0, 1/2)$.
\end{lemma}

\begin{proof}

\end{proof}


\begin{lemma}\label{lem:continuous_brownian}
  \uses{def:brownian}
  \leanok
  \lean{ProbabilityTheory.continuous_brownian}
The paths of the Brownian motion are continuous.
\end{lemma}

\begin{proof}
  \uses{lem:isHolderWith_brownian}

\end{proof}


\begin{lemma}\label{lem:law_brownian_apply}
  \uses{def:brownian}
For $t \in \mathbb{R}_+$, the law of $B_t$ (the Brownian motion at time $t$) is the real Gaussian measure $\mathcal{N}(0,t)$.
\end{lemma}

\begin{proof}

\end{proof}

\section{Wiener measure on the continuous functions}

We want to turn the Brownian motion into a measure on the continuous functions $C(\mathbb{R}_+, \mathbb{R})$ with the Borel sigma-algebra generated by the compact-open topology.


\begin{definition}[Auxiliary Wiener measure]\label{def:wienerMeasureAux}
  \uses{def:brownian, def:gaussianLimit, lem:continuous_brownian}
  \leanok
  \lean{ProbabilityTheory.wienerMeasureAux}
The pushforward of the measure $P_B$ of Definition~\ref{def:gaussianLimit} by the Brownian motion $B$ is a measure on the continuous functions on $\mathbb{R}^{\mathbb{R}_+}$, with the sigma-algebra induced by the product sigma-algebra on $\mathbb{R}^{\mathbb{R}_+}$.
\end{definition}

\textbf{Lean remark}: the auxiliary Wiener measure is a measure on the subtype \texttt{\{f  // Continuous f\}}. This is not the same type as $C(\mathbb{R}_+, \mathbb{R})$.


\begin{theorem}\label{thm:ContinuousMap.borel_eq_iSup_comap_eval}
  \leanok
  \lean{ProbabilityTheory.ContinuousMap.borel_eq_iSup_comap_eval}
The borel sigma-algebra on $C(\mathbb{R}_+, \mathbb{R})$ coming from the compact-open topology is equal to the smallest sigma-algebra for which the evaluation maps $f \mapsto f(t)$ are measurable for every $t \in \mathbb{R}_+$.
\end{theorem}

\begin{proof}
Possible ref: \href{https://math.stackexchange.com/questions/4789531/when-does-the-borel-sigma-algebra-of-compact-convergence-coincide-with-the-pr}{stackexchange question}.
\end{proof}


\begin{definition}\label{def:MeasurableEquiv.continuousMap}
  \uses{thm:ContinuousMap.borel_eq_iSup_comap_eval}
  \leanok
  \lean{ProbabilityTheory.MeasurableEquiv.continuousMap}
The identity is a measurable equivalence between the continuous functions of $\mathbb{R}^{\mathbb{R}_+}$ with the subset sigma-algebra obtained from the product sigma-algebra, and $C(\mathbb{R}_+, \mathbb{R})$ with the Borel sigma-algebra coming from the compact-open topology.

Mathematically this says nothing more than the equality of sigma-algebras of Theorem~\ref{thm:ContinuousMap.borel_eq_iSup_comap_eval} but in Lean we have two different types so we need an equivalence.
\end{definition}


\begin{definition}[Wiener measure]\label{def:wienerMeasure}
  \uses{def:MeasurableEquiv.continuousMap, def:wienerMeasureAux}
  \leanok
  \lean{ProbabilityTheory.wienerMeasure}
The Wiener measure on $C(\mathbb{R}_+, \mathbb{R})$ with the Borel sigma-algebra is the map of the auxiliary Wiener measure by the measurable equivalence of definition~\ref{def:MeasurableEquiv.continuousMap}.
\end{definition}


TODO: add the main properties of the Brownian motion and the Wiener measure.
We need to be able to tell that we have built the correct objects.

\chapter{Stochastic integral}

The lecture notes at \href{https://dec41.user.srcf.net/h/III_L/stochastic_calculus_and_applications/}{this link} are a good reference for this chapter.

\section{Total variation and Lebesgue-Stieltjes integral}

TODO: in Mathlib, we can integrate with respect to the measure given by a right-continuous monotone function (\texttt{StieltjesFunction.measure}). This will be useful to integrate against the quadratic variation of a local martingale.
However, we will also want to integrate with respect to a signed measure given by a càdlàg function with finite variation.
We need new definitions for that one, starting with the definition of the total variation of a function.


\section{Local martingales}

TODO: filtrations should be assumed right-continuous and complete whenever needed.

TODO: this section follows Kallenberg's book and uses $\mathbb{R}_+$ as the time index.
Some of the definitions and results could possibly be generalized.

In this section, $E$ denotes a complete normed space.

First, recall the definitions of a martingale, a stopping time and a stopped process, which are already in Mathlib.


\begin{definition}\label{def:Martingale}
  \mathlibok
  \lean{MeasureTheory.Martingale}
Let $\mathcal{F}$ be a filtration on a measurable space $\Omega$ with measure $P$ indexed by $T$.
A family of functions $M : T \to \Omega \to E$ is a martingale with respect to a filtration $\mathcal{F}$ if $M$ is adapted with respect to $\mathcal{F}$ and for all $i \le j$, $P[M_j \mid \mathcal{F}_i] = M_i$ almost surely.
\end{definition}


\begin{definition}\label{def:IsStoppingTime}
  \mathlibok
  \lean{MeasureTheory.IsStoppingTime}
A stopping time with respect to some filtration $\mathcal{F}$ indexed by $T$ is a function $\tau : \Omega \to T$ such that for all $i$, the preimage of $\{j \mid j \le i\}$ along $\tau$ is measurable with respect to $\mathcal{F}_i$.
\end{definition}


\begin{definition}\label{def:stoppedProcess}
  \mathlibok
  \lean{MeasureTheory.stoppedProcess}
Let $X : T \to \Omega \to E$ be a stochastic process and let $\tau : \Omega \to T$.
The stopped process with respect to $\tau$ is defined by
\begin{align*}
  (X^{\tau})_t = \begin{cases}
    X_t & \text{if } t \le \tau \\
    X_{\tau} & \text{otherwise}
  \end{cases}
\end{align*}
\end{definition}


\begin{definition}\label{def:localMartingale}
  \uses{def:Martingale, def:IsStoppingTime, def:stoppedProcess}
Let $\mathcal{F} = (\mathcal{F}_t)_{t \in \mathbb{R}_+}$ be a filtration on a measurable space $\Omega$.
A local martingale with respect to $\mathcal{F}$ is a stochastic process $M : \mathbb{R}_+ \to \Omega \to E$ adapted to $\mathcal{F}$ such that there exists a localizing sequence $(\tau_n)_{n \in \mathbb{N}}$ such that the following conditions hold:
\begin{itemize}
  \item $\tau_n$ is a stopping time for every $n \in \mathbb{N}$,
  \item $\tau_n$ is non-decreasing and $\tau_n \to \infty$ as $n \to \infty$ (a.s.),
  \item for all $n \in \mathbb{N}$, the stopped and centered process $M^{\tau_n} - M_0$ is a martingale with respect to $\mathcal{F}$.
\end{itemize}
\end{definition}


\begin{definition}\label{def:quadraticVariation}
  \uses{def:localMartingale}
For any continuous local martingale $M$, there exists a continuous process $[M]$ with $[M]_0 = 0$ such that $M^2 - [M]$ is a local martingale. That process is a.s. unique and is called the \emph{quadratic variation} of $M$.
\end{definition}


\begin{definition}\label{def:covariation}
  \uses{def:localMartingale}
For any continuous local martingales $M$ and $N$, there exists a continuous process $[M,N]$ with $[M,N]_0 = 0$ such that $MN - [M,N]$ is a local martingale. That process is a.s. unique and is called the \emph{covariation} of $M$ and $N$.
\end{definition}




\section{Stochastic integral}


\begin{definition}\label{def:predictableStepProcess}
  \uses{def:IsStoppingTime}
Let $(\tau_n)_{n \in \mathbb{N}}$ be a sequence of stopping times which is a.s. non-decreasing and such that $\tau_n \to \infty$ as $n \to \infty$.
Let $(\eta_n)_{n \in \mathbb{N}}$ be a sequence of $\mathcal{F}_{\tau_n}$-measurable random variables.
Then the predictable step process for that sequence is the process $V : \mathbb{R}_+ \to \Omega \to E$ defined by
\begin{align*}
  V_t = \sum_{n=0}^\infty \eta_n \mathbb{1}_{(\tau_n, \tau_{n+1}]}(t)
  \: .
\end{align*}
\end{definition}


\begin{definition}\label{def:elementaryStochasticIntegral}
  \uses{def:predictableStepProcess}
Let $V$ be a predictable step process and let $X$ be a stochastic process.
The \emph{elementary stochastic integral} process $V \cdot X : \mathbb{R}_+ \to \Omega \to E$ is defined by
\begin{align*}
  (V \cdot X)_t
  &= \sum_{n=0}^\infty \eta_n (X^t_{\tau_{n+1}} - X^t_{\tau_n})
  \: .
\end{align*}
\end{definition}


\bibliographystyle{amsalpha}
\bibliography{bib}
